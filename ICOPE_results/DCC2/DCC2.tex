% Subfigure (a) ape table    
    \begin{table}[!h]
            \centering
            % Table
            
\begin{tabular}{|l|r|r|r|r|r|r|}
\hline
    \textbf{} & \textbf{RMSE} & \textbf{Ort.} & \textbf{Medyan} & \textbf{Std.} & \textbf{Min} & \textbf{Max} \\ \hline
\textbf{Optimal Ortalama} & 21.49 & 16.62 & 11.37 & 13.62 & 0.0 & 60.19 \\ \hline
\textbf{Global Ortalama} & 23.21 & 18.94 & 14.53 & 13.42 & 1.07 & 61.82 \\ \hline
\textbf{Kalman Filtresi} & 23.11 & 17.61 & 10.68 & 14.96 & 0.89 & 64.12 \\ \hline
\textbf{Düzeltilmiş Ortalama} & 22.9 & 18.65 & 14.29 & 13.3 & 0.87 & 61.61 \\ \hline
\textbf{LiDAR} & 232.64 & 190.95 & 147.88 & 132.89 & 11.27 & 482.57 \\ \hline
\textbf{Dönüşüm Ortalama} & 319.64 & 294.61 & 291.13 & 124.01 & 5.4 & 503.77 \\ \hline
\end{tabular}
\caption{DCC2 Veriseti için Mutlak Hata Tablosu}\label{tab:d2_ape}
    \end{table}

    \vspace{1em} % Add vertical space between subfigures

    % Subfigure (b) ape table    
    \begin{table}[!h]
            \centering
            % Table
            
\begin{tabular}{|l|r|r|r|r|r|r|}
\hline
    \textbf{} & \textbf{RMSE} & \textbf{Ort.} & \textbf{Medyan} & \textbf{Std.} & \textbf{Min} & \textbf{Max} \\ \hline
\textbf{Optimal Ortalama} & 1.92 & 1.11 & 0.63 & 1.57 & 0.0 & 17.7 \\ \hline
\textbf{Global Ortalama} & 2.85 & 1.83 & 1.14 & 2.19 & 0.0 & 17.61 \\ \hline
\textbf{Kalman Filtresi} & 2.18 & 1.23 & 0.66 & 1.79 & 0.0 & 17.7 \\ \hline
\textbf{Düzeltilmiş Ortalama} & 2.14 & 1.37 & 0.82 & 1.64 & 0.0 & 15.8 \\ \hline
\textbf{LiDAR} & 1.36 & 0.6 & 0.07 & 1.22 & 0.0 & 25.08 \\ \hline
\textbf{Dönüşüm Ortalama} & 1.96 & 1.17 & 0.72 & 1.57 & 0.0 & 25.28 \\ \hline
\end{tabular}
\caption{DCC2 Veriseti için Görece Hata Tablosu}\label{tab:d2_rpe}
    \end{table}

DCC2 veriseti için mutlak hata tablosuna (Tablo (\ref{tab:d2_ape})) bakıldığında, Optimal Ortalama yöntemi beklendiği üzere tüm metriklerde (RMSE, ortalama, medyan, vb.) en düşük hatalara sahiptir ve bu açıdan referans görevi görmektedir. Onu takiben, Düzeltilmiş Ortalama ve Kalman Filtresi görece yakın sonuçlar sunmaktadır. RMSE değerinde Düzeltilmiş Ortalama (22.9 m), hem Global Ortalama (23.21 m) hem de Kalman Filtresi (23.11 m) yöntemlerine kıyasla daha iyi performans göstermiştir. Ancak ortalama değere bakıldığında Kalman Filtresi (17.61 m), Düzeltilmiş Ortalama (18.65 m) yöntemine göre daha düşük bir hataya sahiptir. Bu da her bir metrikte yöntemlerin farklı avantajlar sunduğunu göstermektedir.
LiDAR ve Dönüşüm Ortalama yöntemleri ise ciddi miktarda hata birikimi sergilemiş, en yüksek RMSE değerlerine (sırasıyla 232.64 m ve 319.64 m) sahip olmuşlardır. Özellikle Dönüşüm Ortalama yöntemi, ortalama ve medyan hatalarda da oldukça yüksek değerlere sahip olup, tek başına uygulanmasının sağlıklı bir kestirim sağlamayacağı açıkça görülmektedir.

Görece hata tablosunda (Tablo (\ref{tab:d2_rpe})) ise Optimal Ortalama, birçok metrikte yine güçlü bir performans sergilemekle birlikte, LiDAR dikkat çekici şekilde bazı metriklerde (RMSE ve ortalama gibi) daha düşük hatalara sahiptir. Örneğin LiDAR’ın RMSE değeri (1.36) Optimal Ortalamadan (1.92) daha düşüktür ve ortalama hatası da (0.60) yine en düşük seviyededir. Ancak maksimum hata değerine bakıldığında (25.08), LiDAR yöntemi zaman zaman yüksek sapmalara neden olabilmektedir.

Kalman Filtresi (2.18 RMSE) ve Dönüşüm Ortalama (1.96 RMSE) yöntemleri görece benzer sonuçlar sunarken, Düzeltilmiş Ortalama (2.14 RMSE) bu ikilinin gerisinde kalmaktadır. Global Ortalama yöntemi (2.85 RMSE) ise bu tabloda en yüksek görece hataya sahiptir. Sonuç olarak, görece hata açısından LiDAR beklenenden daha iyi ortalama ve RMSE değerleri sunsa da zaman zaman çok yüksek maksimum hatalar ürettiği, Optimal Ortalama yöntemi ise tutarlı bir şekilde en iyi genel performansı sağladığı görülmektedir.

Kestirim sonuçları ve zamana bağlı kestirim hataları Şekil (\ref{fig:d2_map}) ve (\ref{fig:d2_error})’de görülebilir.

\begin{figure}[!htb]
    \centering
     % Subfigure (c) pozisyon plots
        \begin{minipage}{0.85\textwidth}
            \centering
            \fbox{\includegraphics[trim=2cm 6.7cm 2cm 6.2cm, clip,width=1\linewidth]{ICOPE_results/DCC2/2BT_plot.pdf}}
        \end{minipage}
        \caption{DCC 2 Veriseti 3B Odometri Sonucu ve Sonucun Kuşbakışı Görüntüsü}\label{fig:d2_map}

\end{figure}
\begin{figure}[!htb]

     % Subfigure (d) pozisyon error plots
    \begin{subfigure}
        \centering
        \begin{minipage}{0.45\textwidth}
            \centering
        \includegraphics[width=1\linewidth]{ICOPE_results/DCC2/traj_out/ape_comparison_plot_raw.png}
        \end{minipage}%
        \hspace{0.05\textwidth}
        \begin{minipage}{0.45\textwidth}
            \centering
            \includegraphics[width=1\linewidth]{ICOPE_results/DCC2/traj_out/rpe_comparison_plot_raw.png}
        \end{minipage}
        \caption{DCC 2 Veriseti için Mutlak ve Görece Pozisyon Hatası Sonuçları}\label{fig:d2_error}
    \end{subfigure}

    \vspace{1em} % Add vertical space between subfigures
\begin{comment}
     % Subfigure (e) gain plots
    \begin{subfigure}
        \centering
        \begin{minipage}{0.45\textwidth}
            \centering
        \includegraphics[width=1\linewidth]{ICOPE_results/DCC2/Global_weights.pdf}
        \end{minipage}%
        \hspace{0.05\textwidth}
        \begin{minipage}{0.45\textwidth}
            \centering
            \includegraphics[width=1\linewidth]{ICOPE_results/DCC2/Dönüşüm_weights.pdf}
        \end{minipage}
        \caption{DCC 2 Veriseti için Global ve Dönüşüm Ortalama Metotlarında kullanılan LiDAR Sensörü Ağırlıkları}
    \end{subfigure}
    
\end{comment}

\end{figure}
\begin{comment}
\begin{figure}[!htb]

     % Subfigure (f) gain plots
    \begin{subfigure}
        \centering
        \begin{minipage}{0.45\textwidth}
            \centering
        \includegraphics[width=1\linewidth]{ICOPE_results/DCC2/traj_out/ape_comparison_plot_violin_histogram.png}
        \end{minipage}%
        \hspace{0.05\textwidth}
        \begin{minipage}{0.45\textwidth}
            \centering
            \includegraphics[width=1\linewidth]{ICOPE_results/DCC2/traj_out/rpe_comparison_plot_violin_histogram.png}
        \end{minipage}
        \caption{DCC 2 Veriseti için Mutlak ve Görece Pozisyon Hata Dağılımları}
    \end{subfigure}

    \vspace{1em} % Add vertical space between subfigures
     % Subfigure (f) gain plots
    \begin{subfigure}
        \centering
        \begin{minipage}{0.45\textwidth}
            \centering
        \includegraphics[width=1\linewidth]{ICOPE_results/DCC2/Global_spectrum.pdf}
        \end{minipage}%
        \hspace{0.05\textwidth}
        \begin{minipage}{0.45\textwidth}
            \centering
            \includegraphics[width=1\linewidth]{ICOPE_results/DCC2/Dönüşüm_spectrum.pdf}
        \end{minipage}
        \caption{DCC 2 Veriseti için Global ve Dönüşüm Ortalama Metotlarında kullanılan LiDAR Sensörü Ağırlıklarının Spektrumu}
        \label{fig:dcc_spec}
    \end{subfigure}
    
\end{figure}
\end{comment}

Düzeltilmiş ortalama metodu ise riverside 3 ve DCC 2 verisetlerinde kaynak aldığı sensör verilerinden daha başarılı sonuç verebilmiştir. Bu başarının tüm verisetlerinde tekrarlanamamış olması kesin bir yargıya varmanın önüne geçmekle beraber, Hausdorff Mesafesi metriğinin bir potansiyeli olduğunu göstermektedir.  