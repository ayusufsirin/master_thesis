\chapter{SONUÇ} \label{Conclusion}
\label{sec:conclusion}

Bu çalışmada, Papoulis--Gerchberg (PG) tabanlı LiDAR--stereo derinlik füzyon yönteminin odometrik doğruluk, zamansal kararlılık ve sistematik hata bastırma üzerindeki etkileri hem matematiksel hem de deneysel olarak incelenmiştir. Geliştirilen matematiksel deneysel çerçeve doğrultusunda gerçekleştirilen kapsamlı parametrik analizler, önerilen yöntemin yalnızca ortalama hata değerlerini düşürmekle kalmadığını, aynı zamanda büyük ölçekli sapmaları ve birikimli sürüklenmeyi anlamlı biçimde azalttığını ortaya koymuştur. Mutlak Poz Hatası (APE) metrikleri, dilim tabanlı analizler ve yörünge karşılaştırmaları birlikte değerlendirildiğinde, PG iterasyonlarının frekans alanı kısıtları yoluyla geometrik tutarlılığı güçlendirdiği ve bu sayede özellikle sistematik dönel sapmaların dolaylı olarak telafi edildiği gösterilmiştir. GPU hızlandırmalı uygulama sonuçları, yöntemin gerçek zamana yakın çalışmaya uygun olduğunu doğrulayarak pratik robotik sistemlerde kullanılabilirliğini desteklemiştir. Elde edilen bulgular, derinlik tutarlılığının odometrik performans üzerindeki belirleyici rolünü vurgulamakta ve iteratif, kısıt tabanlı derinlik füzyon yaklaşımlarının çoklu algılayıcı sistemlerde etkili bir çözüm sunduğunu göstermektedir.

\pagebreak



