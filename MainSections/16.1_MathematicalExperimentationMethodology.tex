\section{Deneysel Analiz İçin Matematiksel Çerçeve}
\label{sec:math_analysis_framework}

Bu çalışma, önerilen Papoulis--Gerchberg (PG) tabanlı LiDAR--stereo füzyon hattını, algoritmik konfigürasyonlar üzerinde yapılandırılmış bir tarama (sweep) kapsamında değerlendirmektedir. Amaç yalnızca tek bir ``en iyi'' parametre kümesini seçmek değil; RTK tabanlı GNSS referansı (yer gerçeği) ile karşılaştırılan yörüngelerde, yöntemin iç parametrelerinin odometrik doğruluğa \emph{bağımsız} ve \emph{birlikte} nasıl katkı sunduğunu nicel olarak karakterize etmektir.

\subsection{Problem Tanımı ve Değerlendirme Kurulumu}

Önerilen yöntem iki bağımsız kontrol parametresi tarafından belirlenmektedir:

\begin{itemize}
    \item $I \in \mathbb{N}$: PG iyileştirme iterasyon sayısı,
    \item $H \in \mathbb{N}$: zamansal koşullandırma için kullanılan LiDAR upsampling (LU) geçmiş (history) boyutu.
\end{itemize}

Her bir $(I,H)$ konfigürasyonu için bir kestirim yörüngesi üretilir ve \texttt{evo} kullanılarak RTK referans yörüngesine karşı değerlendirilir. Bu karşılaştırmadan elde edilen skaler bir performans metriği
\begin{equation}
M(I,H)
\end{equation}
ile gösterilsin. Değerlendirme bağlamına bağlı olarak $M$, APE (ör. RMSE), belirli bir uzamsal/zamansal ölçekte RPE, yaw ile ilişkili bir hata ölçütü veya maksimum sapma ya da varyans gibi türetilmiş bir istatistik olabilir. Tüm durumlarda \emph{$M$ değerinin daha küçük olması daha iyi performansı ifade eder}.

\paragraph*{Yörünge eşleştirme ve hizalama politikası.}
Gerçek dünya yörüngeleri eşzamanlı olmayan örneklemeyle elde edilebilir ve farklı koordinat çerçevelerinde ifade edilebilir. Bu nedenle her karşılaştırma; (i) kestirim ve referans yörüngeleri arasında poz eşleştirmesi (association) ve (ii) sabit bir hizalama politikası uygulandıktan sonra yapılır. Metriğin hesaplanmasından önce kestirim yörüngesine uygulanan hizalama operatörü $\mathcal{A}(\cdot)$ ile gösterilsin; burada:
\begin{itemize}
    \item $\mathcal{A}=\mathrm{raw}$: hizalama yok,
    \item $\mathcal{A}=\mathrm{SE(3)}$: rijit hizalama (dönme + öteleme),
    \item $\mathcal{A}=\mathrm{Sim(3)}$: benzerlik hizalaması (dönme + öteleme + ölçek).
\end{itemize}
Bu analiz boyunca $M(I,H)$, \emph{sabit} bir hizalama seçimi $\mathcal{A}$ altında hesaplanır (her metrik grubu için açıkça raporlanır); böylece tüm konfigürasyonlar doğrudan karşılaştırılabilir kalır.

\paragraph*{Temel (baseline) konfigürasyon.}
Temel konfigürasyon
\begin{equation}
M(0,1),
\end{equation}
olarak tanımlanır; bu durum minimal PG iyileştirmesine ($I=0$) ve minimal zamansal koşullandırmaya ($H=1$) karşılık gelir. Bu temel durum, normalize edilmiş iyileştirmeler ve etkileşim analizleri için referans noktası olarak kullanılır.

Birincil amaç, $I$ ve $H$ parametrelerindeki değişimlerin $M$ üzerindeki etkisini nicel olarak belirlemek ve performans kazanımlarının iterasyon derinliği, zamansal koşullandırma veya bu ikisinin etkileşiminden kaynaklanıp kaynaklanmadığını ortaya koymaktır.

\subsection{İki Faktörlü Toplamsal Ayrıştırma Modeli}

Performans yüzeyi $M(I,H)$'yi yorumlamak için iki faktörlü toplamsal bir ayrıştırma benimsenir:
\begin{equation}
\boxed{
M(I,H) \;=\; \mu \;+\; \alpha(I) \;+\; \beta(H) \;+\; \gamma(I,H)
}
\label{eq:additive_model}
\end{equation}
burada:
\begin{itemize}
    \item $\mu$ tüm test edilen konfigürasyonlar üzerindeki küresel ortalama performanstır,
    \item $\alpha(I)$ PG iterasyon sayısına atfedilen \emph{ana etki}yi ifade eder,
    \item $\beta(H)$ LU geçmiş boyutuna atfedilen \emph{ana etki}yi ifade eder,
    \item $\gamma(I,H)$ toplamsal olmayan davranışı yakalayan etkileşim terimidir.
\end{itemize}

Bu model, $I$ ve $H$'nin ortalama katkılarını ayırırken artık (residual) etkileşim yapısını açıkça korur. Önemle belirtmek gerekir ki bu yaklaşım, ölçülen ızgara değerlerinin \emph{etki ayrıştırması} olarak kullanılmaktadır; olasılıksal (probabilistik) bir iddia değildir.

\subsection{Model Bileşenlerinin Kestirimi}

$\mathcal{I}$ ve $\mathcal{H}$ sırasıyla değerlendirilen iterasyon sayıları ve geçmiş boyutlarının ayrık kümelerini göstersin.

\paragraph*{Küresel Ortalama}
\begin{equation}
\mu \;=\; \frac{1}{|\mathcal{I}||\mathcal{H}|}
\sum_{i \in \mathcal{I}} \sum_{h \in \mathcal{H}} M(i,h).
\end{equation}

\paragraph*{PG İterasyonlarının Ana Etkisi}
\begin{equation}
\alpha(i) \;=\; \frac{1}{|\mathcal{H}|}
\sum_{h \in \mathcal{H}} M(i,h) \;-\; \mu.
\end{equation}

\paragraph*{LU Geçmişinin Ana Etkisi}
\begin{equation}
\beta(h) \;=\; \frac{1}{|\mathcal{I}|}
\sum_{i \in \mathcal{I}} M(i,h) \;-\; \mu.
\end{equation}

\paragraph*{Etkileşim Terimi}
\begin{equation}
\gamma(i,h) \;=\; M(i,h) \;-\; \Big(\mu + \alpha(i) + \beta(h)\Big).
\end{equation}

\paragraph*{Tanımlanabilirlik kısıtları (tanım gereği).}
Yukarıdaki tanımlarla ayrıştırma tekildir ve şu koşulları sağlar:
\begin{equation}
\sum_{i\in\mathcal{I}}\alpha(i)=0,
\qquad
\sum_{h\in\mathcal{H}}\beta(h)=0,
\qquad
\sum_{i\in\mathcal{I}}\gamma(i,h)=0,
\qquad
\sum_{h\in\mathcal{H}}\gamma(i,h)=0.
\end{equation}

$\gamma(i,h)$ teriminin büyük mutlak değerlere sahip olması, toplamsal olmayan bir davranışa işaret eder; yani bir parametrenin etkisi, diğer parametrenin seviyesine bağlıdır.

\subsection{Marjinal (Ayrık) Duyarlılık Analizi}

Test edilen ızgara üzerinde artımsal değişimleri incelemek amacıyla marjinal sonlu farklar analiz edilir.

\paragraph*{İterasyon Artışı (Sabit $H$)}
Sabit bir geçmiş boyutu $H$ için, PG iterasyon sayısının $I_1$'den $I_2$'ye arttırılmasının artımsal etkisi
\begin{equation}
\Delta_I M(I_1\!\to\! I_2 \,;\, H) \;=\; M(I_2,H) \;-\; M(I_1,H)
\end{equation}
olarak tanımlanır. $I$ arttıkça $\Delta_I M$ büyüklüğünün azalması, iyileştirme sürecinde yakınsama doygunluğuna (saturation) işaret eder.

\paragraph*{Geçmiş Artışı (Sabit $I$)}
Benzer şekilde, sabit iterasyon sayısı $I$ için LU geçmiş boyutunun $H_1$'den $H_2$'ye arttırılmasının artımsal etkisi
\begin{equation}
\Delta_H M(H_1\!\to\! H_2 \,;\, I) \;=\; M(I,H_2) \;-\; M(I,H_1)
\end{equation}
şeklindedir. Bu analiz, LU geçmişinin öncelikle bir optimizasyon itici gücü mü yoksa daha çok bir kararlılık (stabilizasyon) mekanizması mı olduğunu ortaya koyar.

\subsection{Normalize Edilmiş İyileştirme Oranları}

İyileştirmeleri ölçekten bağımsız bir biçimde ifade etmek için, temel konfigürasyona göre normalize edilmiş kazanımlar tanımlanır:
\begin{equation}
\Delta M_{\text{total}}(I,H) \;=\; M(0,1) \;-\; M(I,H).
\end{equation}
$\Delta M_{\text{total}}(I,H)>0$ olduğunda $(I,H)$ konfigürasyonu temel duruma göre iyileşme sağlamaktadır.

İterasyon derinliğine atfedilen ortalama normalize iyileştirme
\begin{equation}
C_I(i) \;=\; \frac{1}{|\mathcal{H}|}
\sum_{h \in \mathcal{H}}
\frac{M(0,1) - M(i,h)}{M(0,1)}
\end{equation}
ile özetlenir; LU geçmiş boyutuna karşılık gelen özet ise
\begin{equation}
C_H(h) \;=\; \frac{1}{|\mathcal{I}|}
\sum_{i \in \mathcal{I}}
\frac{M(0,1) - M(i,h)}{M(0,1)}
\end{equation}
olarak tanımlanır. Bu nicelikler, $M$'nin mutlak ölçeği metrikler arasında farklılık gösterse bile $I$ ve $H$'nin göreli öneminin doğrudan karşılaştırılmasını mümkün kılar.

\subsection{Etkileşim İçin Süperpozisyon Testi}

$I$ ve $H$ etkilerinin toplamsal olarak birleşip birleşmediğini değerlendirmek amacıyla, dört kanonik konfigürasyon üzerinden bir süperpozisyon hipotezi test edilir:
\begin{itemize}
    \item temel: $(0,1)$,
    \item yalnızca iterasyon: $(I,1)$,
    \item yalnızca geçmiş: $(0,H)$,
    \item birleşik: $(I,H)$.
\end{itemize}

Toplamsal davranış altında beklenen performans
\begin{equation}
M_{\text{exp}}(I,H) \;=\;
M(0,1)
+ \Big[M(I,1) - M(0,1)\Big]
+ \Big[M(0,H) - M(0,1)\Big]
\end{equation}
şeklindedir.

Etkileşim sapması
\begin{equation}
\boxed{
\Gamma(I,H) \;=\; M(I,H) \;-\; M_{\text{exp}}(I,H)
}
\end{equation}
olarak tanımlanır; burada:
\begin{itemize}
    \item $\Gamma(I,H)\approx 0$ toplamsal davranışı,
    \item $\Gamma(I,H)<0$ sinerjik etkileşimi (birleşik etkinin toplamsal beklentiden daha iyi olmasını),
    \item $\Gamma(I,H)>0$ azalan getiriyi (birleşik etkinin toplamsal beklentiden daha kötü olmasını)
\end{itemize}
ifade eder.

\subsection{Metrik Bağımlı Yorum}

Farklı metriklerin $I$ ve $H$'ye karşı farklı duyarlılık profilleri sergilemesi beklenir. APE gibi küresel ölçüler birikimli sürüklenmeyi (drift) yansıtır ve iterasyon derinliği $\alpha(I)$'ye güçlü tepki verebilir; buna karşın yerel tutarlılık ölçüleri (ör. kısa menzilli RPE veya varyans temelli istatistikler) zamansal koşullandırma $\beta(H)$'ye daha duyarlı olabilir. Yönelim odaklı hatalar (ör. yaw) ise çoğu durumda aşağı akıştaki SLAM arka uç kısıtları ve sahne gözlenebilirliği tarafından baskın biçimde belirlenebilir; bu nedenle ön-işleme parametreleriyle doğrusal biçimde ölçeklenmeyebilir.

Bu çerçeve, bir sonraki bölümde raporlanan deneysel ızgaraların yorumlanması için ilkeli bir temel sağlar; böylece PG tabanlı füzyon çıktıları (PG$\to$GT) ile stereo-only temel çıktı (ZED/RTAB-Map$\to$GT) tutarlı değerlendirme koşulları altında sistematik olarak karşılaştırılabilir.
