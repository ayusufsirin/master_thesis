{\section*{\large EKLER}}

\subsection*{A. \quad 1B Sensör Füzyonu Sonuçları}\label{sec:1d_results}

\begin{table}[h]
	\centering
	\begin{tabular}{|l|r|r|r|r|}
\hline
		\textbf{} & \textbf{Ort.} & \textbf{Medyan} & \textbf{Std.} & \textbf{RMSE} \\ \hline
		\textbf{Sabit} & 0.5588 & 0.6080 & 0.0344 & 0.5886 \\ \hline
		\textbf{Dinamik} & 0.4805 & 0.5608 & 0.0520 & 0.5316 \\ \hline
		\textbf{Global} & 0.5518 & 0.6445 & 0.0515 & 0.5965 \\ \hline
		\textbf{Dönüşüm} & 0.4761 & 0.5624 & 0.0479 & 0.5238 \\ \hline
	\end{tabular}
	\caption*{Kapalı Alan Veriseti Deney 07\_02 için Mutlak Hata Tablosu}
\end{table}

\begin{table}[h]
	\centering
	\begin{tabular}{|l|r|r|r|r|}
\hline
		\textbf{} & \textbf{Ort.} & \textbf{Medyan} & \textbf{Std.} & \textbf{RMSE} \\ \hline
		\textbf{Sabit} & 0.2475 & 0.2577 & 0.0148 & 0.2749 \\ \hline
		\textbf{Dinamik} & 0.2450 & 0.1860 & 0.0301 & 0.2985 \\ \hline
		\textbf{Global} & 0.2238 & 0.1771 & 0.0178 & 0.2594 \\ \hline
		\textbf{Dönüşüm} & 0.2630 & 0.2084 & 0.0250 & 0.3054 \\ \hline
	\end{tabular}
	\caption*{Kapalı Alan Veriseti Deney 07\_02 için Görece Hata Tablosu}
\end{table}

\begin{figure}[h]
	\centering
	\includegraphics[trim=0.9cm 7.3cm 1cm 7cm, clip,width=0.8\linewidth]{Figures/1d_odom/07\_02_result.pdf}
	\caption*{Kapalı Alan Veriseti Deney 07\_02 1B Odometri Sonucu}
\end{figure} \pagebreak

\begin{table}[h]
	\centering
	\begin{tabular}{|l|r|r|r|r|}
\hline
		\textbf{} & \textbf{Ort.} & \textbf{Medyan} & \textbf{Std.} & \textbf{RMSE} \\ \hline
		\textbf{Sabit} & 0.4662 & 0.4905 & 0.0390 & 0.5061 \\ \hline
		\textbf{Dinamik} & 0.4393 & 0.4844 & 0.0478 & 0.4904 \\ \hline
		\textbf{Global} & 0.5340 & 0.5975 & 0.0649 & 0.5913 \\ \hline
		\textbf{Dönüşüm} & 0.4669 & 0.5298 & 0.0553 & 0.5224 \\ \hline
	\end{tabular}
	\caption*{Kapalı Alan Veriseti Deney 07\_03 için Mutlak Hata Tablosu}
\end{table}

\begin{table}[h]
	\centering
	\begin{tabular}{|l|r|r|r|r|}
\hline
		\textbf{} & \textbf{Ort.} & \textbf{Medyan} & \textbf{Std.} & \textbf{RMSE} \\ \hline
		\textbf{Sabit} & 0.3691 & 0.3562 & 0.0196 & 0.3937 \\ \hline
		\textbf{Dinamik} & 0.4387 & 0.4337 & 0.0140 & 0.4538 \\ \hline
		\textbf{Global} & 0.5195 & 0.5307 & 0.0172 & 0.5352 \\ \hline
		\textbf{Dönüşüm} & 0.4576 & 0.4760 & 0.0180 & 0.4761 \\ \hline
	\end{tabular}
	\caption*{Kapalı Alan Veriseti Deney 07\_03 için Görece Hata Tablosu}
\end{table}

\begin{figure}[h]
	\centering
	\includegraphics[trim=0.9cm 7.3cm 1cm 7cm, clip,width=0.8\linewidth]{Figures/1d_odom/07\_03_result.pdf}
	\caption*{Kapalı Alan Veriseti Deney 07\_03 1B Odometri Sonucu}
\end{figure} \pagebreak

\begin{table}[h]
	\centering
	\begin{tabular}{|l|r|r|r|r|}
\hline
		\textbf{} & \textbf{Ort.} & \textbf{Medyan} & \textbf{Std.} & \textbf{RMSE} \\ \hline
		\textbf{Sabit} & 0.3185 & 0.3705 & 0.0341 & 0.3680 \\ \hline
		\textbf{Dinamik} & 0.3156 & 0.3705 & 0.0360 & 0.3679 \\ \hline
		\textbf{Global} & 0.4346 & 0.4900 & 0.0709 & 0.5093 \\ \hline
		\textbf{Dönüşüm} & 0.4289 & 0.4815 & 0.0695 & 0.5031 \\ \hline
	\end{tabular}
	\caption*{Kapalı Alan Veriseti Deney 07\_04 için Mutlak Hata Tablosu}
\end{table}

\begin{table}[h]
	\centering
	\begin{tabular}{|l|r|r|r|r|}
\hline
		\textbf{} & \textbf{Ort.} & \textbf{Medyan} & \textbf{Std.} & \textbf{RMSE} \\ \hline
		\textbf{Sabit} & 0.6804 & 0.7136 & 0.0334 & 0.7035 \\ \hline
		\textbf{Dinamik} & 0.6937 & 0.7256 & 0.0368 & 0.7186 \\ \hline
		\textbf{Global} & 0.5788 & 0.5899 & 0.0468 & 0.6162 \\ \hline
		\textbf{Dönüşüm} & 0.5822 & 0.5915 & 0.0488 & 0.6209 \\ \hline
	\end{tabular}
	\caption*{Kapalı Alan Veriseti Deney 07\_04 için Görece Hata Tablosu}
\end{table}

\begin{figure}[h]
	\centering
	\includegraphics[trim=0.9cm 7.3cm 1cm 7cm, clip,width=0.8\linewidth]{Figures/1d_odom/07\_04_result.pdf}
	\caption*{Kapalı Alan Veriseti Deney 07\_04 1B Odometri Sonucu}
\end{figure} \pagebreak

\begin{table}[h]
	\centering
	\begin{tabular}{|l|r|r|r|r|}
\hline
		\textbf{} & \textbf{Ort.} & \textbf{Medyan} & \textbf{Std.} & \textbf{RMSE} \\ \hline
		\textbf{Sabit} & 0.3329 & 0.3936 & 0.0402 & 0.3883 \\ \hline
		\textbf{Dinamik} & 0.3279 & 0.3936 & 0.0428 & 0.3875 \\ \hline
		\textbf{Global} & 0.4267 & 0.4750 & 0.0755 & 0.5072 \\ \hline
		\textbf{Dönüşüm} & 0.4350 & 0.4839 & 0.0750 & 0.5137 \\ \hline
	\end{tabular}
	\caption*{Kapalı Alan Veriseti Deney 07\_05 için Mutlak Hata Tablosu}
\end{table}

\begin{table}[h]
	\centering
	\begin{tabular}{|l|r|r|r|r|}
\hline
		\textbf{} & \textbf{Ort.} & \textbf{Medyan} & \textbf{Std.} & \textbf{RMSE} \\ \hline
		\textbf{Sabit} & 0.3047 & 0.3339 & 0.0122 & 0.3235 \\ \hline
		\textbf{Dinamik} & 0.2872 & 0.3198 & 0.0133 & 0.3087 \\ \hline
		\textbf{Global} & 0.1756 & 0.1569 & 0.0131 & 0.2086 \\ \hline
		\textbf{Dönüşüm} & 0.1743 & 0.1558 & 0.0123 & 0.2057 \\ \hline
	\end{tabular}
	\caption*{Kapalı Alan Veriseti Deney 07\_05 için Görece Hata Tablosu}
\end{table}

\begin{figure}[h]
	\centering
	\includegraphics[trim=0.9cm 7.3cm 1cm 7cm, clip,width=0.8\linewidth]{Figures/1d_odom/07\_05_result.pdf}
	\caption*{Kapalı Alan Veriseti Deney 07\_05 1B Odometri Sonucu}
\end{figure} \pagebreak

\begin{table}[h]
	\centering
	\begin{tabular}{|l|r|r|r|r|}
\hline
		\textbf{} & \textbf{Ort.} & \textbf{Medyan} & \textbf{Std.} & \textbf{RMSE} \\ \hline
		\textbf{Sabit} & 0.2421 & 0.2589 & 0.0105 & 0.2629 \\ \hline
		\textbf{Dinamik} & 0.2402 & 0.2589 & 0.0114 & 0.2627 \\ \hline
		\textbf{Global} & 0.3908 & 0.4398 & 0.0340 & 0.4320 \\ \hline
		\textbf{Dönüşüm} & 0.3806 & 0.4299 & 0.0325 & 0.4210 \\ \hline
	\end{tabular}
	\caption*{Kapalı Alan Veriseti Deney 07\_06 için Mutlak Hata Tablosu}
\end{table}

\begin{table}[h]
	\centering
	\begin{tabular}{|l|r|r|r|r|}
\hline
		\textbf{} & \textbf{Ort.} & \textbf{Medyan} & \textbf{Std.} & \textbf{RMSE} \\ \hline
		\textbf{Sabit} & 0.3089 & 0.3377 & 0.0322 & 0.3558 \\ \hline
		\textbf{Dinamik} & 0.2986 & 0.3024 & 0.0295 & 0.3431 \\ \hline
		\textbf{Global} & 0.2459 & 0.2444 & 0.0226 & 0.2870 \\ \hline
		\textbf{Dönüşüm} & 0.2455 & 0.2475 & 0.0235 & 0.2882 \\ \hline
	\end{tabular}
	\caption*{Kapalı Alan Veriseti Deney 07\_06 için Görece Hata Tablosu}
\end{table}

\begin{figure}[h]
	\centering
	\includegraphics[trim=0.9cm 7.3cm 1cm 7cm, clip,width=0.8\linewidth]{Figures/1d_odom/07\_06_result.pdf}
	\caption*{Kapalı Alan Veriseti Deney 07\_06 1B Odometri Sonucu}
\end{figure} \pagebreak

\begin{table}[h]
	\centering
	\begin{tabular}{|l|r|r|r|r|}
\hline
		\textbf{} & \textbf{Ort.} & \textbf{Medyan} & \textbf{Std.} & \textbf{RMSE} \\ \hline
		\textbf{Sabit} & 0.4545 & 0.4866 & 0.0607 & 0.5166 \\ \hline
		\textbf{Dinamik} & 0.4533 & 0.4866 & 0.0617 & 0.5165 \\ \hline
		\textbf{Global} & 0.5592 & 0.5667 & 0.0991 & 0.6412 \\ \hline
		\textbf{Dönüşüm} & 0.5714 & 0.5731 & 0.1038 & 0.6555 \\ \hline
	\end{tabular}
	\caption*{Kapalı Alan Veriseti Deney 07\_07 için Mutlak Hata Tablosu}
\end{table}

\begin{table}[h]
	\centering
	\begin{tabular}{|l|r|r|r|r|}
\hline
		\textbf{} & \textbf{Ort.} & \textbf{Medyan} & \textbf{Std.} & \textbf{RMSE} \\ \hline
		\textbf{Sabit} & 0.1103 & 0.1023 & 0.0024 & 0.1204 \\ \hline
		\textbf{Dinamik} & 0.1084 & 0.1096 & 0.0022 & 0.1179 \\ \hline
		\textbf{Global} & 0.1503 & 0.1079 & 0.0111 & 0.1822 \\ \hline
		\textbf{Dönüşüm} & 0.1565 & 0.1035 & 0.0129 & 0.1921 \\ \hline
	\end{tabular}
	\caption*{Kapalı Alan Veriseti Deney 07\_07 için Görece Hata Tablosu}
\end{table}

\begin{figure}[h]
	\centering
	\includegraphics[trim=0.9cm 7.3cm 1cm 7cm, clip,width=0.8\linewidth]{Figures/1d_odom/07\_07\_result.pdf}
	\caption*{Kapalı Alan Veriseti Deney 07\_07 1B Odometri Sonucu}
\end{figure} \pagebreak

\begin{table}[h]
	\centering
	\begin{tabular}{|l|r|r|r|r|}
\hline
		\textbf{} & \textbf{Ort.} & \textbf{Medyan} & \textbf{Std.} & \textbf{RMSE} \\ \hline
		\textbf{Sabit} & 0.3744 & 0.4203 & 0.0317 & 0.4143 \\ \hline
		\textbf{Dinamik} & 0.3725 & 0.4203 & 0.0331 & 0.4143 \\ \hline
		\textbf{Global} & 0.5406 & 0.6397 & 0.0573 & 0.5909 \\ \hline
		\textbf{Dönüşüm} & 0.6065 & 0.7177 & 0.0732 & 0.6638 \\ \hline
	\end{tabular}
	\caption*{Kapalı Alan Veriseti Deney 07\_08 için Mutlak Hata Tablosu}
\end{table}

\begin{table}[h]
	\centering
	\begin{tabular}{|l|r|r|r|r|}
\hline
		\textbf{} & \textbf{Ort.} & \textbf{Medyan} & \textbf{Std.} & \textbf{RMSE} \\ \hline
		\textbf{Sabit} & 0.4528 & 0.4439 & 0.0706 & 0.5228 \\ \hline
		\textbf{Dinamik} & 0.4622 & 0.4431 & 0.0714 & 0.5318 \\ \hline
		\textbf{Global} & 0.4730 & 0.3523 & 0.0917 & 0.5590 \\ \hline
		\textbf{Dönüşüm} & 0.4891 & 0.4016 & 0.0994 & 0.5792 \\ \hline
	\end{tabular}
	\caption*{Kapalı Alan Veriseti Deney 07\_08 için Görece Hata Tablosu}
\end{table}

\begin{figure}[h]
	\centering
	\includegraphics[trim=0.9cm 7.3cm 1cm 7cm, clip,width=0.8\linewidth]{Figures/1d_odom/07\_08_result.pdf}
	\caption*{Kapalı Alan Veriseti Deney 07\_08 1B Odometri Sonucu}
\end{figure} \pagebreak

\begin{table}[h]
	\centering
	\begin{tabular}{|l|r|r|r|r|}
\hline
		\textbf{} & \textbf{Ort.} & \textbf{Medyan} & \textbf{Std.} & \textbf{RMSE} \\ \hline
		\textbf{Sabit} & 0.3350 & 0.3558 & 0.0153 & 0.3570 \\ \hline
		\textbf{Dinamik} & 0.3339 & 0.3558 & 0.0160 & 0.3569 \\ \hline
		\textbf{Global} & 0.4901 & 0.5537 & 0.0409 & 0.5301 \\ \hline
		\textbf{Dönüşüm} & 0.4867 & 0.5552 & 0.0421 & 0.5280 \\ \hline
	\end{tabular}
	\caption*{Kapalı Alan Veriseti Deney 07\_09 için Mutlak Hata Tablosu}
\end{table}

\begin{table}[h]
	\centering
	\begin{tabular}{|l|r|r|r|r|}
\hline
		\textbf{} & \textbf{Ort.} & \textbf{Medyan} & \textbf{Std.} & \textbf{RMSE} \\ \hline
		\textbf{Sabit} & 0.2348 & 0.2503 & 0.0161 & 0.2660 \\ \hline
		\textbf{Dinamik} & 0.2321 & 0.2503 & 0.0145 & 0.2606 \\ \hline
		\textbf{Global} & 0.4004 & 0.3870 & 0.0168 & 0.4202 \\ \hline
		\textbf{Dönüşüm} & 0.4094 & 0.4059 & 0.0181 & 0.4303 \\ \hline
	\end{tabular}
	\caption*{Kapalı Alan Veriseti Deney 07\_09 için Görece Hata Tablosu}
\end{table}

\begin{figure}[h]
	\centering
	\includegraphics[trim=0.9cm 7.3cm 1cm 7cm, clip,width=0.8\linewidth]{Figures/1d_odom/07\_09_result.pdf}
	\caption*{Kapalı Alan Veriseti Deney 07\_09 1B Odometri Sonucu}
\end{figure} \pagebreak

\begin{table}[h]
	\centering
	\begin{tabular}{|l|r|r|r|r|}
\hline
		\textbf{} & \textbf{Ort.} & \textbf{Medyan} & \textbf{Std.} & \textbf{RMSE} \\ \hline
		\textbf{Sabit} & 0.3826 & 0.4455 & 0.0329 & 0.4232 \\ \hline
		\textbf{Dinamik} & 0.3814 & 0.4455 & 0.0338 & 0.4232 \\ \hline
		\textbf{Global} & 0.5187 & 0.6206 & 0.0622 & 0.5753 \\ \hline
		\textbf{Dönüşüm} & 0.5184 & 0.6206 & 0.0626 & 0.5754 \\ \hline
	\end{tabular}
	\caption*{Kapalı Alan Veriseti Deney 07\_10 için Mutlak Hata Tablosu}
\end{table}

\begin{table}[h]
	\centering
	\begin{tabular}{|l|r|r|r|r|}
\hline
		\textbf{} & \textbf{Ort.} & \textbf{Medyan} & \textbf{Std.} & \textbf{RMSE} \\ \hline
		\textbf{Sabit} & 0.1995 & 0.1502 & 0.0212 & 0.2455 \\ \hline
		\textbf{Dinamik} & 0.2046 & 0.1682 & 0.0206 & 0.2485 \\ \hline
		\textbf{Global} & 0.1577 & 0.1115 & 0.0222 & 0.2153 \\ \hline
		\textbf{Dönüşüm} & 0.1585 & 0.1035 & 0.0230 & 0.2176 \\ \hline
	\end{tabular}
	\caption*{Kapalı Alan Veriseti Deney 07\_10 için Görece Hata Tablosu}
\end{table}

\begin{figure}[h]
	\centering
	\includegraphics[trim=0.9cm 7.3cm 1cm 7cm, clip,width=0.8\linewidth]{Figures/1d_odom/07\_10_result.pdf}
	\caption*{Kapalı Alan Veriseti Deney 07\_10 1B Odometri Sonucu}
\end{figure} \pagebreak

\begin{table}[h]
	\centering
	\begin{tabular}{|l|r|r|r|r|}
\hline
		\textbf{} & \textbf{Ort.} & \textbf{Medyan} & \textbf{Std.} & \textbf{RMSE} \\ \hline
		\textbf{Sabit} & 0.0564 & 0.0433 & 0.0019 & 0.0712 \\ \hline
		\textbf{Dinamik} & 0.0000 & 0.0000 & 0.0000 & 0.0003 \\ \hline
		\textbf{Global} & 0.0648 & 0.0496 & 0.0030 & 0.0849 \\ \hline
		\textbf{Dönüşüm} & 0.1003 & 0.0944 & 0.0039 & 0.1183 \\ \hline
	\end{tabular}
	\caption*{Kapalı Alan Veriseti Deney 11\_01 için Mutlak Hata Tablosu}
\end{table}

\begin{table}[h]
	\centering
	\begin{tabular}{|l|r|r|r|r|}
\hline
		\textbf{} & \textbf{Ort.} & \textbf{Medyan} & \textbf{Std.} & \textbf{RMSE} \\ \hline
		\textbf{Sabit} & 0.0616 & 0.0658 & 0.0021 & 0.0762 \\ \hline
		\textbf{Dinamik} & 0.0004 & 0.0000 & 0.0000 & 0.0012 \\ \hline
		\textbf{Global} & 0.0839 & 0.0808 & 0.0027 & 0.0982 \\ \hline
		\textbf{Dönüşüm} & 0.0972 & 0.0856 & 0.0036 & 0.1134 \\ \hline
	\end{tabular}
	\caption*{Kapalı Alan Veriseti Deney 11\_01 için Görece Hata Tablosu}
\end{table}

\begin{figure}[h]
	\centering
	\includegraphics[trim=0.9cm 7.3cm 1cm 7cm, clip,width=0.8\linewidth]{Figures/1d_odom/11\_01_result.pdf}
	\caption*{Kapalı Alan Veriseti Deney 11\_01 1B Odometri Sonucu}
\end{figure} \pagebreak

\begin{table}[h]
	\centering
	\begin{tabular}{|l|r|r|r|r|}
\hline
		\textbf{} & \textbf{Ort.} & \textbf{Medyan} & \textbf{Std.} & \textbf{RMSE} \\ \hline
		\textbf{Sabit} & 0.0631 & 0.0513 & 0.0032 & 0.0849 \\ \hline
		\textbf{Dinamik} & 0.0000 & 0.0000 & 0.0000 & 0.0003 \\ \hline
		\textbf{Global} & 0.1479 & 0.1383 & 0.0080 & 0.1726 \\ \hline
		\textbf{Dönüşüm} & 0.1105 & 0.1156 & 0.0027 & 0.1220 \\ \hline
	\end{tabular}
	\caption*{Kapalı Alan Veriseti Deney 11\_02 için Mutlak Hata Tablosu}
\end{table}

\begin{table}[h]
	\centering
	\begin{tabular}{|l|r|r|r|r|}
\hline
		\textbf{} & \textbf{Ort.} & \textbf{Medyan} & \textbf{Std.} & \textbf{RMSE} \\ \hline
		\textbf{Sabit} & 0.1063 & 0.0947 & 0.0045 & 0.1251 \\ \hline
		\textbf{Dinamik} & 0.0438 & 0.0358 & 0.0015 & 0.0582 \\ \hline
		\textbf{Global} & 0.1913 & 0.1914 & 0.0100 & 0.2150 \\ \hline
		\textbf{Dönüşüm} & 0.1627 & 0.1652 & 0.0041 & 0.1744 \\ \hline
	\end{tabular}
	\caption*{Kapalı Alan Veriseti Deney 11\_02 için Görece Hata Tablosu}
\end{table}

\begin{figure}[h]
	\centering
	\includegraphics[trim=0.9cm 7.3cm 1cm 7cm, clip,width=0.8\linewidth]{Figures/1d_odom/11\_02_result.pdf}
	\caption*{Kapalı Alan Veriseti Deney 11\_02 1B Odometri Sonucu}
\end{figure} \pagebreak

\begin{table}[h]
	\centering
	\begin{tabular}{|l|r|r|r|r|}
\hline
		\textbf{} & \textbf{Ort.} & \textbf{Medyan} & \textbf{Std.} & \textbf{RMSE} \\ \hline
		\textbf{Sabit} & 0.2524 & 0.2570 & 0.0045 & 0.2610 \\ \hline
		\textbf{Dinamik} & 0.2516 & 0.2570 & 0.0045 & 0.2605 \\ \hline
		\textbf{Global} & 0.3412 & 0.3718 & 0.0100 & 0.3554 \\ \hline
		\textbf{Dönüşüm} & 0.3383 & 0.3688 & 0.0083 & 0.3502 \\ \hline
	\end{tabular}
	\caption*{Kapalı Alan Veriseti Deney 11\_03 için Mutlak Hata Tablosu}
\end{table}

\begin{table}[h]
	\centering
	\begin{tabular}{|l|r|r|r|r|}
\hline
		\textbf{} & \textbf{Ort.} & \textbf{Medyan} & \textbf{Std.} & \textbf{RMSE} \\ \hline
		\textbf{Sabit} & 0.1155 & 0.1137 & 0.0044 & 0.1327 \\ \hline
		\textbf{Dinamik} & 0.1155 & 0.1137 & 0.0044 & 0.1326 \\ \hline
		\textbf{Global} & 0.1978 & 0.2026 & 0.0088 & 0.2180 \\ \hline
		\textbf{Dönüşüm} & 0.1897 & 0.2008 & 0.0071 & 0.2069 \\ \hline
	\end{tabular}
	\caption*{Kapalı Alan Veriseti Deney 11\_03 için Görece Hata Tablosu}
\end{table}

\begin{figure}[h]
	\centering
	\includegraphics[trim=0.9cm 7.3cm 1cm 7cm, clip,width=0.8\linewidth]{Figures/1d_odom/11\_03_result.pdf}
	\caption*{Kapalı Alan Veriseti Deney 11\_03 1B Odometri Sonucu}
\end{figure} \pagebreak

\begin{table}[h]
	\centering
	\begin{tabular}{|l|r|r|r|r|}
\hline
		\textbf{} & \textbf{Ort.} & \textbf{Medyan} & \textbf{Std.} & \textbf{RMSE} \\ \hline
		\textbf{Sabit} & 0.1515 & 0.1592 & 0.0034 & 0.1623 \\ \hline
		\textbf{Dinamik} & 0.1503 & 0.1592 & 0.0037 & 0.1621 \\ \hline
		\textbf{Global} & 0.2748 & 0.2885 & 0.0120 & 0.2958 \\ \hline
		\textbf{Dönüşüm} & 0.3937 & 0.4263 & 0.0205 & 0.4189 \\ \hline
	\end{tabular}
	\caption*{Kapalı Alan Veriseti Deney 11\_04 için Mutlak Hata Tablosu}
\end{table}

\begin{table}[h]
	\centering
	\begin{tabular}{|l|r|r|r|r|}
\hline
		\textbf{} & \textbf{Ort.} & \textbf{Medyan} & \textbf{Std.} & \textbf{RMSE} \\ \hline
		\textbf{Sabit} & 0.1330 & 0.1360 & 0.0045 & 0.1484 \\ \hline
		\textbf{Dinamik} & 0.1377 & 0.1439 & 0.0048 & 0.1535 \\ \hline
		\textbf{Global} & 0.2207 & 0.1990 & 0.0149 & 0.2511 \\ \hline
		\textbf{Dönüşüm} & 0.2973 & 0.3076 & 0.0250 & 0.3352 \\ \hline
	\end{tabular}
	\caption*{Kapalı Alan Veriseti Deney 11\_04 için Görece Hata Tablosu}
\end{table}

\begin{figure}[h]
	\centering
	\includegraphics[trim=0.9cm 7.3cm 1cm 7cm, clip,width=0.8\linewidth]{Figures/1d_odom/11\_04_result.pdf}
	\caption*{Kapalı Alan Veriseti Deney 11\_04 1B Odometri Sonucu}
\end{figure} \pagebreak

\begin{table}[h]
	\centering
	\begin{tabular}{|l|r|r|r|r|}
\hline
		\textbf{} & \textbf{Ort.} & \textbf{Medyan} & \textbf{Std.} & \textbf{RMSE} \\ \hline
		\textbf{Sabit} & 0.1218 & 0.1174 & 0.0022 & 0.1305 \\ \hline
		\textbf{Dinamik} & 0.1204 & 0.1174 & 0.0021 & 0.1289 \\ \hline
		\textbf{Global} & 0.2632 & 0.2644 & 0.0057 & 0.2738 \\ \hline
		\textbf{Dönüşüm} & 0.2095 & 0.2152 & 0.0055 & 0.2222 \\ \hline
	\end{tabular}
	\caption*{Kapalı Alan Veriseti Deney 11\_05 için Mutlak Hata Tablosu}
\end{table}

\begin{table}[h]
	\centering
	\begin{tabular}{|l|r|r|r|r|}
\hline
		\textbf{} & \textbf{Ort.} & \textbf{Medyan} & \textbf{Std.} & \textbf{RMSE} \\ \hline
		\textbf{Sabit} & 0.0855 & 0.0847 & 0.0029 & 0.1004 \\ \hline
		\textbf{Dinamik} & 0.0718 & 0.0647 & 0.0019 & 0.0833 \\ \hline
		\textbf{Global} & 0.1063 & 0.1084 & 0.0053 & 0.1278 \\ \hline
		\textbf{Dönüşüm} & 0.0940 & 0.0840 & 0.0042 & 0.1133 \\ \hline
	\end{tabular}
	\caption*{Kapalı Alan Veriseti Deney 11\_05 için Görece Hata Tablosu}
\end{table}

\begin{figure}[h]
	\centering
	\includegraphics[trim=0.9cm 7.3cm 1cm 7cm, clip,width=0.8\linewidth]{Figures/1d_odom/11\_05_result.pdf}
	\caption*{Kapalı Alan Veriseti Deney 11\_05 1B Odometri Sonucu}
\end{figure} \pagebreak

\begin{table}[h]
	\centering
	\begin{tabular}{|l|r|r|r|r|}
\hline
		\textbf{} & \textbf{Ort.} & \textbf{Medyan} & \textbf{Std.} & \textbf{RMSE} \\ \hline
		\textbf{Sabit} & 0.1031 & 0.0825 & 0.0078 & 0.1358 \\ \hline
		\textbf{Dinamik} & 0.1023 & 0.0825 & 0.0080 & 0.1357 \\ \hline
		\textbf{Global} & 0.2678 & 0.2776 & 0.0144 & 0.2934 \\ \hline
		\textbf{Dönüşüm} & 0.3240 & 0.3344 & 0.0148 & 0.3460 \\ \hline
	\end{tabular}
	\caption*{Kapalı Alan Veriseti Deney 11\_06 için Mutlak Hata Tablosu}
\end{table}

\begin{table}[h]
	\centering
	\begin{tabular}{|l|r|r|r|r|}
\hline
		\textbf{} & \textbf{Ort.} & \textbf{Medyan} & \textbf{Std.} & \textbf{RMSE} \\ \hline
		\textbf{Sabit} & 0.0614 & 0.0513 & 0.0039 & 0.0869 \\ \hline
		\textbf{Dinamik} & 0.0606 & 0.0420 & 0.0039 & 0.0860 \\ \hline
		\textbf{Global} & 0.1456 & 0.1411 & 0.0113 & 0.1791 \\ \hline
		\textbf{Dönüşüm} & 0.1625 & 0.1540 & 0.0085 & 0.1860 \\ \hline
	\end{tabular}
	\caption*{Kapalı Alan Veriseti Deney 11\_06 için Görece Hata Tablosu}
\end{table}

\begin{figure}[h]
	\centering
	\includegraphics[trim=0.9cm 7.3cm 1cm 7cm, clip,width=0.8\linewidth]{Figures/1d_odom/11\_06_result.pdf}
	\caption*{Kapalı Alan Veriseti Deney 11\_06 1B Odometri Sonucu}
\end{figure} \pagebreak


\clearpage

\subsection*{B. \quad Kesin Pozitif Matris ile Yarı-Kesin Pozitif Matris Toplamının Daima Pozitif Matris Olduğunun Kanıtlanması}\label{sec:psd_proof}

\subsubsection*{Problemin Kurulumu}
Aşağıdaki şartları sağlayan \( A \), \( B \) ve \( C \) matrislerini varsayalım:
\begin{itemize}
    \item \( C = A + B \),
    \item \( A \) simetrik yarı-kesin pozitif matris,
    \item \( B \) simetrik kesin pozitif matris,
    \item \( C \) simetrik yarı-kesin pozitif matris.
\end{itemize}
Amacımız bu şartları sağlayan \( A \) ve \( B \) matrislerini bulmaktır.

\subsubsection*{Özdeğer Ayrışımı}
Simetrik \( A \), \( B \) ve \( C \) matrislerini aşağıdaki gibi yazabiliriz:
\begin{align}
A &= Q_A \Lambda_A Q_A^T, \\
B &= Q_B \Lambda_B Q_B^T, \\
C &= Q_C \Lambda_C Q_C^T,
\end{align}
\noindent burada:
\begin{itemize}
    \item \( Q_A \), \( Q_B \) ve \( Q_C \) ortogonal matrislerdir \((Q_A^T Q_A = Q_B^T Q_B = Q_C^T Q_C = I)\),
    \item \( \Lambda_A = \operatorname{diag}(\lambda_1^A, \lambda_2^A) \), \( \Lambda_B = \operatorname{diag}(\lambda_1^B, \lambda_2^B) \), \( \Lambda_C = \operatorname{diag}(\lambda_1^C, \lambda_2^C) \) özdeğer matrisleridir.
\end{itemize}
\noindent Bu özdeğerler şu şartları sağlar:
\begin{align}
\lambda_1^A &> 0, \quad \lambda_2^A = 0, \quad \text{(yarı-kesin pozitif)}, \\
\lambda_1^B &> 0, \quad \lambda_2^B > 0, \quad \text{(kesin pozitif)}, \\
\lambda_1^C &> 0, \quad \lambda_2^C = 0, \quad \text{(yarı-kesin pozitif)}.
\end{align}

\subsubsection*{\( C = A + B \) Eşitliğinin Hesaplanması}
Daha genel bir çözüm yapabilmek için, \( A \) ve \( B \)'nin aynı özvektör tabanını paylaşmadığını  varsayalım. Hesaplamayı kolaylaştırmak için, \( Q_A \) ve \( Q_B \) şu şekilde olsun (\(\theta\)'nın 0 olması durumunda aynı özvektör tabanını elde ederiz.):
\begin{align}
Q_A &= \begin{bmatrix}
\cos\theta & \sin\theta \\
-\sin\theta & \cos\theta
\end{bmatrix}, \quad Q_B = I.
\end{align}
Bu durumda:
\begin{align}
A &= Q_A \Lambda_A Q_A^T = \begin{bmatrix}
x \cos^2\theta & x \cos\theta \sin\theta \\
x \cos\theta \sin\theta & x \sin^2\theta
\end{bmatrix}, \quad \Lambda_A = \begin{bmatrix} x & 0 \\ 0 & 0 \end{bmatrix}, \quad x > 0,
\end{align}
ve
\begin{align}
B &= \Lambda_B = \begin{bmatrix}
y & 0 \\
0 & z
\end{bmatrix}, \quad y > 0, \; z > 0.
\end{align}
\( C \) matrisi şu şekilde olur:
\begin{align}
C = A + B = \begin{bmatrix}
x \cos^2\theta + y & x \cos\theta \sin\theta \\
x \cos\theta \sin\theta & x \sin^2\theta + z
\end{bmatrix}.
\end{align}

\subsubsection*{\( C \)'nin Özdeğerleri}
\( C \)'nin özdeğerlerini bulmak için aşağıdaki denklemi çözeriz:
\begin{align}
\det(C - \lambda I) = 0.
\end{align}
\( C - \lambda I \)'yi yerine koyarsak:
\begin{align}
C - \lambda I = \begin{bmatrix}
x \cos^2\theta + y - \lambda & x \cos\theta \sin\theta \\
x \cos\theta \sin\theta & x \sin^2\theta + z - \lambda
\end{bmatrix}.
\end{align}
Determinant şu şekilde olur:
\begin{align}
\det(C - \lambda I) = \left(x \cos^2\theta + y - \lambda\right)\left(x \sin^2\theta + z - \lambda\right) - \left(x \cos\theta \sin\theta\right)^2. \label{determinant_eq}
\end{align}

\subsubsection*{0 Özdeğer Durumu}
\( \lambda = 0 \) durumunda Denklem (\ref{determinant_eq}) şu hale gelir:
\begin{align}
\left(x \cos^2\theta + y\right)\left(x \sin^2\theta + z\right) - \left(x \cos\theta \sin\theta\right)^2 = 0. \label{zero_eigenvalue_eq}
\end{align}
Terimleri genişletirsek:
\begin{align}
\left(x \cos^2\theta + y\right)\left(x \sin^2\theta + z\right) &= x^2 \cos^2\theta \sin^2\theta + x z \cos^2\theta + y x \sin^2\theta + y z, \\
\left(x \cos\theta \sin\theta\right)^2 &= x^2 \cos^2\theta \sin^2\theta.
\end{align}
Bu ifadeleri çıkarırsak:
\begin{align}
x z \cos^2\theta + y x \sin^2\theta + y z = 0. \label{final_eq}
\end{align}

\subsubsection*{Denklemin Analizi}
Denklem (\ref{final_eq})'deki ilk iki terim \( x, y, z > 0 \) ve \( \cos^2\theta, \sin^2\theta \geq 0 \) şartları sağlandığında negatif olmayan bir değer alacaktır. Üçüncü terim ise aynı şartlar altında daima pozitiftir. Bu nedenle, toplamları daima pozitiftir:
\begin{align}
x z \cos^2\theta + y x \sin^2\theta + y z > 0.
\end{align}
Bu durum, denklemin sağlanamayacağı anlamına gelir.

\pagebreak

\subsection*{C. \quad 3B Sensör Füzyonu Sonuçları}

\begin{figure}[!htb]
    \centering
    % Subfigure (a) ape table    
    \begin{subfigure}
            \centering
            % Table
\begin{tabular}{|l|r|r|r|r|r|r|}
\hline
    \textbf{} & \textbf{RMSE} & \textbf{Ort.} & \textbf{Medyan} & \textbf{Std.} & \textbf{Min} & \textbf{Max} \\ \hline
\textbf{Düzeltilmiş Ortalama} & 6.24 & 5.15 & 4.3 & 3.52 & 0.25 & 30.0 \\ \hline
\textbf{LiDAR} & 330.58 & 246.28 & 187.94 & 220.53 & 1.8 & 738.87 \\ \hline
\textbf{Optimal Ortalama} & 19.07 & 16.69 & 19.35 & 9.22 & 0.0 & 38.66 \\ \hline
\textbf{Dönüşüm Ortalama} & 262.18 & 218.94 & 140.1 & 144.22 & 4.56 & 493.65 \\ \hline
\textbf{Global Ortalama} & 6.34 & 5.24 & 4.34 & 3.57 & 0.17 & 28.93 \\ \hline
\textbf{Kalman Filtresi} & 5.88 & 4.74 & 3.7 & 3.48 & 0.19 & 29.73 \\ \hline
\end{tabular}
\caption*{riverside1 Veriseti için Mutlak Hata Tablosu}
    \end{subfigure}

    \vspace{1em} % Add vertical space between subfigures

    % Subfigure (b) ape table    
    \begin{subfigure}
            \centering
            % Table
\begin{tabular}{|l|r|r|r|r|r|r|}
\hline
    \textbf{} & \textbf{RMSE} & \textbf{Ort.} & \textbf{Medyan} & \textbf{Std.} & \textbf{Min} & \textbf{Max} \\ \hline
\textbf{Düzeltilmiş Ortalama} & 1.25 & 0.6 & 0.26 & 1.1 & 0.0 & 16.94 \\ \hline
\textbf{LiDAR} & 0.05 & 0.03 & 0.02 & 0.04 & 0.0 & 0.72 \\ \hline
\textbf{Optimal Ortalama} & 0.93 & 0.4 & 0.16 & 0.84 & 0.0 & 10.86 \\ \hline
\textbf{Dönüşüm Ortalama} & 0.98 & 0.47 & 0.25 & 0.85 & 0.0 & 15.13 \\ \hline
\textbf{Global Ortalama} & 1.09 & 0.5 & 0.19 & 0.97 & 0.0 & 13.18 \\ \hline
\textbf{Kalman Filtresi} & 1.0 & 0.44 & 0.18 & 0.9 & 0.0 & 16.11 \\ \hline
\end{tabular}
\caption*{riverside1 Veriseti için Görece Hata Tablosu}
    \end{subfigure}

    \vspace{1em} % Add vertical space between subfigures
    \begin{minipage}{0.75\textwidth}
        \centering
        \fbox{\includegraphics[trim=0.85cm 6.2cm 0.85cm 6.2cm, clip,width=1\linewidth]{ICOPE_results/riverside1/2BT_plot.pdf}}
        \caption*{Riverside 1 Veriseti 3B Odometri Sonucu ve Sonucun Kuşbakışı Görüntüsü}
    \end{minipage}

\end{figure}
\begin{figure}[!htb]
            \centering

     % Subfigure (d) pozisyon error plots
        \begin{minipage}{0.65\textwidth}
            \centering
        \includegraphics[width=1\linewidth]{ICOPE_results/riverside1/traj_out/ape_comparison_plot_raw.png}
        \caption*{Riverside 1 Veriseti için Mutlak Pozisyon Hatası Sonuçları}
        \end{minipage}%
        
        \begin{minipage}{0.65\textwidth}
            \centering
            \includegraphics[width=1\linewidth]{ICOPE_results/riverside1/traj_out/rpe_comparison_plot_raw.png}
        \caption*{Riverside 1 Veriseti için Görece Pozisyon Hatası Sonuçları}
        \end{minipage}

    \vspace{1em} % Add vertical space between subfigures

\end{figure}

\begin{figure}[!htb]

     % Subfigure (e) gain plots
    \begin{subfigure}
        \centering
        \begin{minipage}{0.45\textwidth}
            \centering
        \includegraphics[trim=0.85cm 6.2cm 0.85cm 6.2cm, clip,width=1\linewidth]{ICOPE_results/riverside1/Global_weights.pdf}
        \end{minipage}%
        \hspace{0.05\textwidth}
        \begin{minipage}{0.45\textwidth}
            \centering
            \includegraphics[trim=0.85cm 6.2cm 0.85cm 6.2cm, clip,width=1\linewidth]{ICOPE_results/riverside1/Dönüşüm_weights.pdf}
        \end{minipage}
        \caption*{Riverside 1 Veriseti için Global ve Dönüşüm Ortalama Metotlarında kullanılan LiDAR Sensörü Ağırlıkları}
    \end{subfigure}

    \vspace{3em} % Add vertical space between subfigures
     % Subfigure (f) gain plots
    \begin{subfigure}
        \centering
        \begin{minipage}{0.45\textwidth}
            \centering
        \includegraphics[trim=0.85cm 6.2cm 0.85cm 6.2cm, clip,width=1\linewidth]{ICOPE_results/riverside1/Global_spectrum.pdf}
        \end{minipage}%
        \hspace{0.05\textwidth}
        \begin{minipage}{0.45\textwidth}
            \centering
            \includegraphics[trim=0.85cm 6.2cm 0.85cm 6.2cm, clip,width=1\linewidth]{ICOPE_results/riverside1/Dönüşüm_spectrum.pdf}
        \end{minipage}
        \caption*{Riverside 1 Veriseti için Global ve Dönüşüm Ortalama Metotlarında kullanılan LiDAR Sensörü Ağırlıklarının Spektrumu}
        \label{fig:r1_spec}
    \end{subfigure}
\end{figure}
\begin{figure}[!htb]
    \centering
    % Subfigure (a) ape table    
    \begin{subfigure}
        \centering
        \begin{tabular}{|l|r|r|r|r|r|r|}
        \hline
            \textbf{} & \textbf{RMSE} & \textbf{Ort.} & \textbf{Medyan} & \textbf{Std.} & \textbf{Min} & \textbf{Max} \\ \hline
        \textbf{Düzeltilmiş Ortalama} & 5.7 & 4.93 & 5.03 & 2.87 & 0.13 & 32.87 \\ \hline
        \textbf{LiDAR} & 380.55 & 274.96 & 190.7 & 263.09 & 8.61 & 846.19 \\ \hline
        \textbf{Optimal Ortalama} & 11.57 & 9.71 & 7.89 & 6.28 & 0.0 & 34.83 \\ \hline
        \textbf{Dönüşüm Ortalama} & 131.82 & 119.23 & 107.85 & 56.23 & 28.42 & 321.65 \\ \hline
        \textbf{Global Ortalama} & 6.35 & 5.38 & 5.64 & 3.36 & 0.07 & 35.21 \\ \hline
        \textbf{Kalman Filtresi} & 4.68 & 3.85 & 3.51 & 2.66 & 0.11 & 32.83 \\ \hline
        \end{tabular}
        \caption*{riverside2 Veriseti için Mutlak Hata Tablosu}
    \end{subfigure}

    \vspace{1em} % Add vertical space between subfigures

    % Subfigure (b) ape table    
    \begin{subfigure}
        \centering
        \begin{tabular}{|l|r|r|r|r|r|r|}
        \hline
            \textbf{} & \textbf{RMSE} & \textbf{Ort.} & \textbf{Medyan} & \textbf{Std.} & \textbf{Min} & \textbf{Max} \\ \hline
        \textbf{Düzeltilmiş Ortalama} & 1.13 & 0.63 & 0.31 & 0.95 & 0.0 & 12.4 \\ \hline
        \textbf{LiDAR} & 3.57 & 0.32 & 0.03 & 3.55 & 0.0 & 201.97 \\ \hline
        \textbf{Optimal Ortalama} & 0.75 & 0.29 & 0.11 & 0.69 & 0.0 & 13.89 \\ \hline
        \textbf{Dönüşüm Ortalama} & 3.53 & 0.57 & 0.28 & 3.49 & 0.0 & 195.9 \\ \hline
        \textbf{Global Ortalama} & 1.2 & 0.55 & 0.19 & 1.06 & 0.0 & 14.4 \\ \hline
        \textbf{Kalman Filtresi} & 0.85 & 0.34 & 0.12 & 0.78 & 0.0 & 13.89 \\ \hline
        \end{tabular}
        \caption*{riverside2 Veriseti için Görece Hata Tablosu}
    \end{subfigure}

    \vspace{1em} % Add vertical space between subfigures
    \begin{minipage}{0.85\textwidth}
        \centering
        \fbox{\includegraphics[trim=0.85cm 6.2cm 0.85cm 6.2cm, clip,width=1\linewidth]{ICOPE_results/riverside2/2BT_plot.pdf}}
        \caption*{Riverside 2 Veriseti 3B Odometri Sonucu ve Sonucun Kuşbakışı Görüntüsü}
    \end{minipage}
\end{figure}

\begin{figure}[!htb]
            \centering

     % Subfigure (d) pozisyon error plots
        \begin{minipage}{0.65\textwidth}
            \centering
        \includegraphics[width=1\linewidth]{ICOPE_results/riverside2/traj_out/ape_comparison_plot_raw.png}
        \caption*{Riverside 2 Veriseti için Mutlak Pozisyon Hatası Sonuçları}
        \end{minipage}%
        
        \begin{minipage}{0.65\textwidth}
            \centering
            \includegraphics[width=1\linewidth]{ICOPE_results/riverside2/traj_out/rpe_comparison_plot_raw.png}
        \caption*{Riverside 2 Veriseti için Görece Pozisyon Hatası Sonuçları}
        \end{minipage}

    \vspace{1em} % Add vertical space between subfigures

\end{figure}

\begin{figure}[!htb]

     % Subfigure (e) gain plots
    \begin{subfigure}
        \centering
        \begin{minipage}{0.45\textwidth}
            \centering
        \includegraphics[trim=0.85cm 6.2cm 0.85cm 6.2cm, clip,width=1\linewidth]{ICOPE_results/riverside2/Global_weights.pdf}
        \end{minipage}%
        \hspace{0.05\textwidth}
        \begin{minipage}{0.45\textwidth}
            \centering
            \includegraphics[trim=0.85cm 6.2cm 0.85cm 6.2cm, clip,width=1\linewidth]{ICOPE_results/riverside2/Dönüşüm_weights.pdf}
        \end{minipage}
        \caption*{Riverside 2 Veriseti için Global ve Dönüşüm Ortalama Metotlarında kullanılan LiDAR Sensörü Ağırlıkları}
    \end{subfigure}

    \vspace{3em} % Add vertical space between subfigures
     % Subfigure (f) gain plots
    \begin{subfigure}
        \centering
        \begin{minipage}{0.45\textwidth}
            \centering
        \includegraphics[trim=0.85cm 6.2cm 0.85cm 6.2cm, clip,width=1\linewidth]{ICOPE_results/riverside2/Global_spectrum.pdf}
        \end{minipage}%
        \hspace{0.05\textwidth}
        \begin{minipage}{0.45\textwidth}
            \centering
            \includegraphics[trim=0.85cm 6.2cm 0.85cm 6.2cm, clip,width=1\linewidth]{ICOPE_results/riverside2/Dönüşüm_spectrum.pdf}
        \end{minipage}
        \caption*{Riverside 2 Veriseti için Global ve Dönüşüm Ortalama Metotlarında kullanılan LiDAR Sensörü Ağırlıklarının Spektrumu}
        \label{fig:r1_spec}
    \end{subfigure}
\end{figure}