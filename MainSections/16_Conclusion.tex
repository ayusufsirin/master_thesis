
\section{SONUÇ} \label{Conclusion}

%Bu tez çalışmasında, doğası gereği yaptığı kestirim ile birlikte varyans bilgisi vermeyen LOAM algoritması için ortalama maliyet ve Hausdorff Mesafesi hata metrikleri ile kestirimin varyans değerini bulmaya çalıştık ve bu değerlerin başarımını, yalın bir yöntem olan ters varyans ortalaması temelli metotlar ile test ettik. Ayrıca bu konudaki testleri kendi verisetimizde yapabilmek amacıyla kendi sensör sistemimizi oluşturduk. Bu sistemle kapalı alan da 16 adet deney aldık ve sistemi açık alanda veri toplayacak şekilde düzenleme çalışmalarına başladık.

%Ortalama maliyet metriği ilk denediğimiz metriktir ve bu metrik ile yaptığımız sensör füzyonunun sonuçları çok stabil sonuçlar vermemiştir. Ayrıca ortalamada, hem mutlak hem de görece hata açısından, sabit bir ağırlık ile yapılabilinecek sensör füzyonunun altında kalmıştır. Bu metriğin bir diğer dezavantajı ise LOAM algoritmasının doğruluğunu büyük oranda etkileyen haritalama adımının varyansa etkisini göz ardı etmesidir. Tüm bunları göz önünde bulundurduğumuzda, bu metrik üzerinden yapılan varyans kestirimi sonuçları başarısız olmuştur.

%Hausdorff Mesafesi metriği ise, varyans kestiriminde haritalama adımını da göz önünde bulundurması nedeniyle bir sonraki tercihimiz olmuştur. Ayrıca çalışmanın bu kısmında, kalman filtresinin iki tahmini arasındaki dönüşümün varyansını elde etmek için özgün bir yol önerilmiştir. Hausdorff Mesafesi metriği, test ettiğimiz dört verisetinin ikisinde (riverside 3 ve DCC2) pozisyonların, ikisinde (riverside 1 ve DCC 2) ise dönüşümlerin ters varyans ortalaması sonucunda Kalman Filtresinin sonucundan daha başarılı sonuç vermiştir. Ayrıca önerdiğimiz Düzeltilmiş Ortalama metodu tüm verisetlerinde, P\(^3\)-LOAM\cite{p3loam} makalesinde referans olarak önerilen Optimum Ortalama sonucunda elde edilen mutlak hata değerinden \%6.6 dan düşük sapma göstermiştir.


Bu tez çalışmasında, LOAM algoritmasının eksikliklerini gidererek Hausdorff Mesafesi tabanlı bir varyans kestirim yöntemi geliştirilmiş ve bu yöntemin sensör füzyonu üzerindeki etkisi incelenmiştir. Çalışmanın temel amacı, LiDAR, IMU ve GPS gibi farklı sensörlerden elde edilen verilerin stokastik bir düzlemde birleştirilmesiyle, güvenilir ve hassas bir pozisyon kestirimi gerçekleştirmektir. LOAM algoritmasının temel problemi olan varyans bilgisi eksikliği, Hausdorff Mesafesi ile başarılı bir şekilde giderilmiştir. Bu yöntem, haritalama adımının varyansa etkisini göz önünde bulundurarak, LOAM algoritmasının çıktılarındaki güvenilirliği artırmış ve sensör füzyonunda daha sağlıklı bir temel oluşturmuştur. Ayrıca, Kalman filtresi ile entegre edilen bu yaklaşım, kısa ve uzun vadeli hata birikimlerini azaltarak LOAM algoritmasının performansını önemli ölçüde iyileştirmiştir.

Ortalama maliyet metriği ilk denediğimiz metriktir ve bu metrik ile yaptığımız sensör füzyonunun sonuçları çok stabil sonuçlar vermemiştir. Ayrıca ortalamada, hem mutlak hem de görece hata açısından, sabit bir ağırlık ile yapılabilinecek sensör füzyonunun altında kalmıştır. Bu metriğin bir diğer dezavantajı ise LOAM algoritmasının doğruluğunu büyük oranda etkileyen haritalama adımının varyansa etkisini göz ardı etmesidir. Tüm bunları göz önünde bulundurduğumuzda, bu metrik üzerinden yapılan varyans kestirimi sonuçlarında kaydadeğer bir başarı elde edilememiştir.

Hausdorff Mesafesi tabanlı varyans kestirimi, farklı verisetleri üzerinde test edilmiş ve önerilen yöntemin doğruluğu çeşitli deney ortamlarında değerlendirilmiştir. Çalışmada önerilen yöntemin sensör verilerini hatayı azaltacak şekilde birleştirebildiği ve daha stabil pozisyon kestirimi sunduğu gözlemlenmiştir. Test ettiğimiz dört verisetinin ikisinde (riverside 3 ve DCC2) pozisyonların, ikisinde (riverside 1 ve DCC 2) ise dönüşümlerin ters varyans ortalaması sonucunda Kalman Filtresinin sonucundan daha başarılı sonuç vermiştir. Ayrıca önerdiğimiz Düzeltilmiş Ortalama metodu tüm verisetlerinde, P\(^3\)-LOAM\cite{p3loam} makalesinde referans olarak önerilen Optimal Ortalama sonucunda elde edilen mutlak hata değerinden \%6.6'dan daha düşük sapma göstermiştir. Bunun sonucunda Hausdorf Mesafesi ile varyans kestirimini kullanabilecek yeni algoritmalar önerilmiş ve bu konudaki çalışmalardan ilk sonuçlar alınmıştır. Ayrıca, bu bağlamda, Kalman filtresinin son iki tahmini arasındaki dönüşümün varyansını hesaplamak için özgün bir yöntem önerilmiş ve bu yöntem sensör füzyonunda kullanılmıştır. 

Donanım tarafında ise LiDAR, IMU, GPS ve stereo kamera gibi çeşitli sensörlerden veri toplayabilen bir sistem geliştirilmiştir. Bu sistem, hem kapalı hem de açık alanlarda veri toplamaya uygun olacak şekilde tasarlanmış ve genişletilebilir bir altyapı sunmuştur. Bu sistem ile kapalı alanda LiDAR ve stereo kamera odometri verilerini içeren, mutlak referansa sahip bir veriseti oluşturulmuştur.

Sonuç olarak, bu tez çalışması, LOAM algoritmasının eksikliklerini gidererek, sensör füzyonu ve varyans kestirimi alanında katkı sağlamaktadır. Önerilen Hausdorff Mesafesi tabanlı varyans kestirim yöntemi, SLAM algoritmalarında başarılı sensör füzyonu için yeterli bir araç olarak değerlendirilmekle birlikte, gelecekte bu yöntemin geniş veri setleri üzerinde daha kapsamlı bir şekilde test edilmesi ve alternatif metriklerle karşılaştırılması önerilmiştir.


\pagebreak



