\section{LiDAR Nokta Yoğunluğu Artırımı}
\label{lidar_upsample}

LiDAR verisi doğası gereği seyrek bir yapıya sahiptir. Seyrek yapıdaki LiDAR verisinin doğrudan işlem hattına alınarak stereo görüntü yakınsatmasında kullanılması durumunda, kılavuz görevi gören LiDAR noktalarının yetersizliği nedeniyle Papoulis--Gerchberg (PG) sonrası elde edilen stereo derinlik haritasının daha da bozulduğu gözlemlenmiştir. Bu problemi gidermek amacıyla, LiDAR noktalarına bir ön işleme uygulanmış ve odometri verisi ile füzyon gerçekleştirilmiştir. Bu füzyon işlemi, temelde odometri verileri kullanılarak LiDAR noktalarının öteleme ve dönme işlemlerine tabi tutulması ve böylece kümülatif bir nokta kümesi elde edilmesi esasına dayanmaktadır.

Yapı ilk bakışta bir LiDAR--Ataletsel Odometri (LIO) işlemini andırsa da, burada çok daha basit bir yaklaşım önerilmektedir. Düşük boyutlu zaman pencereleri (window) içerisinde gerçekleştirilen bu işlem, temel bir harita üretmekte ve algoritmayı bu aşamada tam kapsamlı bir SLAM çözümü gereksiniminden kurtarmaktadır.

\begin{figure}[H]
    \centering
    \begin{adjustbox}{width=\textwidth * 10 /12}
        \includegraphics{mermaid/lidar_upsample.png}
    \end{adjustbox}
    \caption{LiDAR nokta yoğunluğu artırımı (LU) akış şeması}
    \label{fig:lidar_upsample}
\end{figure}

\subsection{Nokta bulutu ve odometri gösterimi}

Her zaman adımı \(t_k\) için robot aşağıdaki mesajları yayınlamaktadır:
\begin{itemize}
    \item \texttt{/odom} başlığı altında bir odometri mesajı,
    \item \texttt{/velodyne\_points} başlığı altında bir 3B nokta bulutu mesajı.
\end{itemize}

Zaman \(t_k\) anındaki odometri pozu, odometri referans çerçevesi
\(\mathcal{F}_\text{odom}\) içerisinde
\begin{equation}
    \mathbf{p}_k \in \mathbb{R}^3,
    \qquad
    \mathbf{q}_k \in \mathbb{S}^3,
\end{equation}
şeklinde tanımlanır. Burada \(\mathbf{p}_k\) konumu, \(\mathbf{q}_k\) ise birim kuaterniyon biçimindeki yönelimi temsil etmektedir.
Buna karşılık gelen dönme matrisi
\begin{equation}
    \mathbf{R}_k = \mathcal{R}(\mathbf{q}_k) \in SO(3)
\end{equation}
olarak ifade edilir.

LiDAR algılayıcı referans çerçevesi \(\mathcal{F}_\text{lidar}\) ile odometri
referans çerçevesi \(\mathcal{F}_\text{odom}\) arasındaki homojen rijit cisim dönüşümü
zaman \(t_k\) için
\begin{equation}
    \mathbf{T}_k =
    \begin{bmatrix}
        \mathbf{R}_k & \mathbf{p}_k \\
        \mathbf{0}^\top & 1
    \end{bmatrix}
    \in SE(3)
\end{equation}
şeklinde tanımlanır.

Zaman \(t_k\) anındaki nokta bulutu \(N_k\) adet noktadan oluşur:
\begin{equation}
    \mathcal{X}_k = \left\{ \mathbf{x}_{k,j} \in \mathbb{R}^3 \;\middle|\;
                        j = 1,\dots,N_k \right\},
\end{equation}
burada \(\mathbf{x}_{k,j}\) noktaları \(\mathcal{F}_\text{lidar}\) çerçevesinde ifade edilmektedir.
Homojen koordinatlar
\begin{equation}
    \tilde{\mathbf{x}}_{k,j} =
    \begin{bmatrix}
        \mathbf{x}_{k,j} \\
        1
    \end{bmatrix}
    \in \mathbb{R}^4
\end{equation}
olarak tanımlanır.

\subsection{Nokta bulutunun rijit cisim dönüşümü}

Her bir nokta, odometriden elde edilen poz kullanılarak LiDAR referans çerçevesinden
odometri referans çerçevesine dönüştürülür:
\begin{equation}
    \tilde{\mathbf{y}}_{k,j}
    =
    \mathbf{T}_k \, \tilde{\mathbf{x}}_{k,j}
    =
    \begin{bmatrix}
        \mathbf{R}_k & \mathbf{p}_k \\
        \mathbf{0}^\top & 1
    \end{bmatrix}
    \begin{bmatrix}
        \mathbf{x}_{k,j} \\[2pt]
        1
    \end{bmatrix}
    =
    \begin{bmatrix}
        \mathbf{R}_k \mathbf{x}_{k,j} + \mathbf{p}_k \\[2pt]
        1
    \end{bmatrix}.
\end{equation}

\(\mathcal{F}_\text{odom}\) çerçevesindeki dönüştürülmüş nokta,
ilk üç bileşen alınarak
\begin{equation}
    \mathbf{y}_{k,j} = \mathbf{R}_k \mathbf{x}_{k,j} + \mathbf{p}_k
\end{equation}
şeklinde elde edilir.

\texttt{/transformed\_point\_cloud} başlığı altında yayınlanan dönüştürülmüş nokta bulutu
\begin{equation}
    \mathcal{Y}_k
    =
    \left\{ \mathbf{y}_{k,j} \in \mathbb{R}^3 \;\middle|\;
        j = 1,\dots,N_k \right\}
\end{equation}
olarak tanımlanır.

\noindent
\textit{Uygulama notu.} Mevcut uygulamada dönme matrisi
\(\mathbf{R}_k = \mathbf{I}_3\)
olarak alınmıştır. Bu durumda dönüşüm
\(\mathbf{y}_{k,j} = \mathbf{x}_{k,j} + \mathbf{p}_k\)
şeklinde yalnızca öteleme içermektedir.

\subsection{Kayan pencere tabanlı kümülatif nokta bulutu}

\(H \in \mathbb{N}\) değeri geçmiş uzunluğunu
(uygulamada \texttt{PC\_HISTORY\_SIZE}) temsil etsin.
Zaman adımı \(k\) için düğüm, son \(H\) adet dönüştürülmüş nokta bulutunu
\(\mathcal{Y}_{k-H+1}, \dots, \mathcal{Y}_k\)
şeklinde kayan bir pencere içerisinde tutar.
Kümülatif dönüştürülmüş nokta bulutu
\begin{equation}
    \mathcal{Y}_k^{\text{cum}}
    =
    \bigcup_{i = \max(1,\,k-H+1)}^{k}
    \mathcal{Y}_i
\end{equation}
şeklinde tanımlanır ve bu küme tek bir \texttt{PointCloud2} mesajı olarak
\texttt{/cumulative\_point\_cloud} başlığı altında yayınlanır.

\paragraph{Başlık eşlemelerinin özeti.}
\texttt{/velodyne\_points} başlığı altında gelen \(\mathcal{X}_k\) nokta bulutları ve
\texttt{/odom} başlığı altında gelen \((\mathbf{R}_k, \mathbf{p}_k)\) odometri pozları kullanılarak:
\begin{align}
    \mathcal{X}_k
    &\xrightarrow{(\mathbf{R}_k, \mathbf{p}_k)}
    \mathcal{Y}_k
    &&\text{\texttt{/transformed\_point\_cloud} üzerinde yayınlanır,} \\
    \{\mathcal{Y}_i\}_{i=k-H+1}^k
    &\mapsto
    \mathcal{Y}_k^{\text{cum}}
    &&\text{\texttt{/cumulative\_point\_cloud} üzerinde yayınlanır,} \\
    \{\mathcal{Z}_i\}_{i=k-H+1}^k
    &\mapsto
    \mathcal{Z}_k^{\text{cum}}
    &&\text{\texttt{/cumulative\_origin\_point\_cloud} üzerinde yayınlanır.}
\end{align}

\subsection{Zamanlama ve gerçek zamanlılık metrikleri}

Geometrik dönüşümlere ek olarak, düğüm gerçek zamanlı davranışı değerlendirmek amacıyla
çeşitli zamanlama metriklerini de izlemektedir.

\subsubsection{Zaman damgaları ve gecikme}

Her senkronize mesaj çifti için:
\begin{itemize}
    \item LiDAR zaman damgası \(\tau^{\text{pc}}_k\),
    \item geri çağırımın alındığı ROS zamanı \(\tau^{\text{in}}_k\),
    \item işlemenin tamamlandığı ROS zamanı \(\tau^{\text{out}}_k\)
\end{itemize}
olarak tanımlanır.

Uçtan uca gecikme
\begin{equation}
    L_k = \tau^{\text{out}}_k - \tau^{\text{pc}}_k
\end{equation}
şeklinde hesaplanır ve milisaniye cinsinden kaydedilir.

\subsubsection{İşlem süresi}

Zaman adımı \(k\) için toplam işlem süresi
\begin{equation}
    T^{\text{proc}}_k
    =
    \tau^{\text{end}}_k - \tau^{\text{start}}_k
\end{equation}
olarak tanımlanır ve alt aşamalara ayrılarak kaydedilir.

\subsubsection{Giriş ve işlem oranları}

Senkronize veri çiftlerinin giriş hızı
\(\lambda_{\text{in}}\),
işleme hızı ise
\(\lambda_{\text{proc}}\)
olarak hesaplanır.
Bu iki büyüklüğün oranı
\begin{equation}
    \rho = \frac{\lambda_{\text{proc}}}{\lambda_{\text{in}}}
\end{equation}
olarak tanımlanır.

\subsubsection{Gerçek zamanlılık koşulu}

Bir zaman aralığında gerçek zamanlı çalışmanın gerekli koşulu
\begin{equation}
    \rho \gtrsim 1
\end{equation}
ve gecikmenin üst sınır altında kalmasıdır.

\begin{table}[h!]
    \centering
    \caption{GPU hızlandırmalı nokta bulutu dönüşüm hattına ait özet istatistikler}
    \begin{tabular}{lccc}
        \hline
        \textbf{Metrik}                       & \textbf{Ortalama} & \textbf{Medyan} & \textbf{Maksimum} \\
        \hline
        Gecikme (ms)                          & 1.43              & 0.91            & 448.05            \\
        Kare başına işlem süresi (ms)         & 43.97             & 38.14           & 933.23            \\
        Nokta bulutu geri çağırım süresi (ms) & 43.81             & 38.13           & 933.17            \\
        Dönüşüm adımı süresi (ms)             & 8.90              & 8.73            & 54.73             \\
        PC oluşturma süresi (ms)              & 0.62              & 0.61            & 30.89             \\
        Giriş hızı (Hz)                       & 9.80              & 10.02           & 14.29             \\
        İşleme hızı (Hz)                      & 9.57              & 10.00           & 11.24             \\
        Aktarım oranı                         & 0.98              & 1.00            & 1.39              \\
        Kümülatif bulut boyutu                & 9.99              & 10.00           & 10.00             \\
        Kuyruk boyutu                         & 0.00              & 0.00            & 5.00              \\
        \hline
    \end{tabular}
\end{table}

Deneysel sonuçlar, önerilen GPU hızlandırmalı dönüşüm hattının hedef algılayıcı
konfigürasyonu için gerçek zamanlı performans sağladığını göstermektedir.
Kare başına ortalama işlem süresi $43.97\,\text{ms}$ olup, bu değer yaklaşık
$9.57\,\text{Hz}$’lik bir işleme hızına karşılık gelmektedir ve
$9.80$–$10.02\,\text{Hz}$ aralığındaki LiDAR giriş hızıyla uyumludur.
Ortalama aktarım oranı $\rho = 0.98$ olup, medyan değerin bire çok yakın olması
($0.9999$), sistemin nokta bulutlarını birikimli gecikme veya kuyruk
oluşumu olmaksızın tutarlı biçimde işlediğini göstermektedir.
Bu durum, kuyruk doluluk değerlerinin neredeyse tüm karelerde sıfır kalmasıyla
da doğrulanmaktadır.

Dönüşüm adımı GPU hızlandırmasından önemli ölçüde fayda sağlamaktadır.
CuPy tabanlı rijit cisim dönüşümü ortalama yalnızca $8.9\,\text{ms}$ sürmekte,
uçtan uca gecikmelerin medyan değeri $0.9\,\text{ms}$ seviyesine kadar
düşmektedir. Kümülatif nokta bulutu oluşturma ve orijin hizalama adımları da
gerçek zamanlı sınırlar içerisinde çalışmakta ve maksimum geçmiş uzunluğu
($H=10$ kare) seviyesinde kararlı hale gelmektedir.

Sonuç olarak, ölçümler sistemin
$\lambda_{\text{proc}} \approx \lambda_{\text{in}}$
koşulunu sağladığını ve gecikmelerin tipik robotik algılama bütçelerinin
($<100\,\text{ms}$) oldukça altında kaldığını göstermektedir.
Bu bulgular, GPU hızlandırmalı nokta bulutu işlemenin sürekli çalışma koşulları
altında hem verimli hem de dayanıklı olduğunu ortaya koymaktadır.
