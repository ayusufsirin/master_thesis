\chapter{METODOLOJİ}  \label{sec:metodology}

\section{PG Tabanlı Stereo–LiDAR Füzyonu İçin İşlem Hattı}

Bu çalışmada önerilen yöntem, PG algoritmasının klasik formundan yola çıkılarak aşağıdaki temel aşamalar etrafında kurgulanmıştır:

\begin{enumerate}
    \item \textbf{Ön İşleme:}
    Stereo kameradan elde edilen disparite haritası derinlik haritasına dönüştürülür ve uygunsa filtrelenir. LiDAR nokta bulutu, kamera koordinat sistemine taşınır ve pinhole kamera modeli ile derinlik çerçevesine projeksiyon yapılarak tek kanallı bir LiDAR derinlik imgesi elde edilir.
    % TODO: Burada kullandığın kalibrasyon ve projeksiyon denklemlerini (iç/dış parametre matrisleri) ekleyebilirsin.

    \item \textbf{Maskeleme ve Veri Zenginleştirme:}
    Stereo derinlik haritasındaki boş veya düşük güvenilirlikli bölgeler için stereo maske oluşturulur; LiDAR projeksiyonunda veri olmayan noktalar için de LiDAR null maskesi tanımlanır. İstenirse önceki LiDAR çerçeveleri kullanılarak tarihçeye dayalı bir doldurma (history-based infill) yapılabilir.

    \item \textbf{Papoulis–Gerchberg Yinelemeleri:}
    Stereo derinlik haritası yoğun arka plan, LiDAR derinlik imgesi ise yüksek güvenilirlikli nokta kümesi olarak kabul edilir. Her yinelemede:
    \begin{itemize}
        \item Uzay alanında LiDAR ölçümleri sabitlenir, stereo derinlik değerleri LiDAR ankrajları ile tutarlı olacak şekilde güncellenir,
        \item Frekans alanında, yüksek frekans bileşenlerini mümkün olduğunca koruyup orta frekans bölgesini bastıran bir filtreleme uygulanır (özellikle kenar keskinliği için önemlidir).
        % TODO: Kullandığın filtreyi (örneğin bant-reject veya low-pass filtre) matematiksel olarak tanımlayabilirsin.
    \end{itemize}

    \item \textbf{Art İşleme:}
    Yinelemeler sonucunda elde edilen derinlik haritası, stereo ve LiDAR maskeleri kullanılarak tutarlılık açısından kontrol edilir; düşük güvenilirlikli bölgeler elenir, gerekli durumlarda yumuşatma (smoothing) ve kenar keskinleştirme adımları uygulanır.
\end{enumerate}

Bu işlem hattı, derin öğrenme tabanlı modellere kıyasla daha düşük hesaplama maliyetiyle çalışabilen, fakat LiDAR doğruluğunu stereo derinlik haritalarına etkin bir biçimde aktarabilen deterministik bir füzyon çerçevesi sunmaktadır. İlerleyen bölümlerde, bu genel yapı ayrıntılandırılacak ve hem KITTI veri seti hem de kurulan mobil sensör düzeneği üzerinden elde edilen sonuçlarla desteklenecektir.

Uyanık ve Ozbayın calismalarindan esinlenilerek lidar noktalarinin coznurlugu arttirilarak stero kamera verisi cozunurlugune yaklastirilmaya calisilmistir. Ancak bu tez kapsaminda lidar yogunluk arttirimi icin PG yerine daha sade bir yontem kullanilmis ve PG multimodal fusion asamasinda kullanilmistir. Methodolojinin genel bir akis semasi Sekil~\ref{methodology_0}'de verildigi gibidir.

\begin{figure}[H]
    \centering
    \begin{adjustbox}{width=\textwidth * 10 /12}
        \includegraphics{mermaid/fusion_flow.png}
    \end{adjustbox}
    \caption{Füzyon algoritması akış diyagramı 0}
    \label{methodology_0}
\end{figure}

\begin{figure}[H]
    \centering
    \begin{adjustbox}{width=\textwidth * 10 /12}
        \includegraphics{mermaid/methodology.png}
    \end{adjustbox}
    \caption{Füzyon algoritması akış diyagramı 1}
    \label{methodology_1}
\end{figure}

\subsection{LiDAR Nokta Yoğunluğu Arttırımı} \label{lidar_upsample}

Lidar verisi doğası gereği seyrek bir yapıya sahiptir. Seyrek yapıdaki Lidar verisini işlem hattına sokarak stereo imaj yakınsatmasında kullanıldığı takdirde kılavuz görevi gören lidar noktalarının yetersizliği sebebiyle PG sonrası stereo derinlik haritasının daha da kötüleşmesi gibi bir problem görülmüştür. Bundan dolayı lidar noktalarına bir ön işleme uygulanarak odometri verisi ile fuse edilmiştir. Bu füsyon, temelde odometry verileri kullanılarak lidar noktalarının translate ve rotate ettirilmesi ve kümülatif nokta kümesi elde edilmesidir.

Bakıldığında LIO (lidar inertial odometry) operasyonu gibi görünse de çok daha basit bir yapı önerilmiştir. Düşük boyutlu window'larda bu işlemi yapmak çok temel bir harita elde ederken algoritmayı bu aşamada SLAM yapma gereksiniminden kurtarmıştır.

\begin{figure}[H]
    \centering
    \begin{adjustbox}{width=\textwidth * 10 /12}
        \includegraphics{mermaid/lidar_upsample.png}
    \end{adjustbox}
    \caption{LU akış şeması}
    \label{fig:lidar_upsample}
\end{figure}

\subsubsection{Point cloud and odometry notation}

At each time step \(t_k\), the robot publishes:
\begin{itemize}
  \item an odometry message on \texttt{/odom}, and
  \item a 3D point cloud message on \texttt{/velodyne\_points}.
\end{itemize}

Let the odometry pose at time \(t_k\) in the odometry frame \(\mathcal{F}_\text{odom}\) be
\begin{equation}
  \mathbf{p}_k \in \mathbb{R}^3,
  \qquad
  \mathbf{q}_k \in \mathbb{S}^3,
\end{equation}
where \(\mathbf{p}_k\) is the position and \(\mathbf{q}_k\) is the unit quaternion orientation.
The corresponding rotation matrix is
\begin{equation}
  \mathbf{R}_k = \mathcal{R}(\mathbf{q}_k) \in SO(3).
\end{equation}

The homogeneous rigid-body transform from the LiDAR sensor frame
\(\mathcal{F}_\text{lidar}\) to the odometry frame \(\mathcal{F}_\text{odom}\) at
time \(t_k\) is
\begin{equation}
  \mathbf{T}_k =
  \begin{bmatrix}
    \mathbf{R}_k & \mathbf{p}_k \\
    \mathbf{0}^\top & 1
  \end{bmatrix}
  \in SE(3).
\end{equation}

The point cloud at time \(t_k\) contains \(N_k\) points
\begin{equation}
  \mathcal{X}_k = \left\{ \mathbf{x}_{k,j} \in \mathbb{R}^3 \;\middle|\;
  j = 1,\dots,N_k \right\},
\end{equation}
where \(\mathbf{x}_{k,j}\) is expressed in \(\mathcal{F}_\text{lidar}\).
We denote the homogeneous coordinates by
\begin{equation}
  \tilde{\mathbf{x}}_{k,j} =
  \begin{bmatrix}
    \mathbf{x}_{k,j} \\
    1
  \end{bmatrix}
  \in \mathbb{R}^4.
\end{equation}

\subsubsection{Rigid-body transformation of the point cloud}

Each point is transformed from the LiDAR frame to the odometry frame using
the odometry-derived pose:
\begin{equation}
  \tilde{\mathbf{y}}_{k,j}
  =
  \mathbf{T}_k \, \tilde{\mathbf{x}}_{k,j}
  =
  \begin{bmatrix}
    \mathbf{R}_k & \mathbf{p}_k \\
    \mathbf{0}^\top & 1
  \end{bmatrix}
  \begin{bmatrix}
    \mathbf{x}_{k,j} \\[2pt]
    1
  \end{bmatrix}
  =
  \begin{bmatrix}
    \mathbf{R}_k \mathbf{x}_{k,j} + \mathbf{p}_k \\[2pt]
    1
  \end{bmatrix}.
\end{equation}

The transformed point in \(\mathcal{F}_\text{odom}\) is the first three components:
\begin{equation}
  \mathbf{y}_{k,j} = \mathbf{R}_k \mathbf{x}_{k,j} + \mathbf{p}_k.
\end{equation}

The transformed point cloud published on \texttt{/transformed\_point\_cloud} is
\begin{equation}
  \mathcal{Y}_k
  =
  \left\{ \mathbf{y}_{k,j} \in \mathbb{R}^3 \;\middle|\;
  j = 1,\dots,N_k \right\}.
\end{equation}

\noindent
\textit{Implementation note.} In the current implementation, the rotation matrix
is set to identity,
\(\mathbf{R}_k = \mathbf{I}_3\),
so that
\(\mathbf{y}_{k,j} = \mathbf{x}_{k,j} + \mathbf{p}_k\),
i.e.\ only a translation is applied.

\subsubsection{Sliding-window cumulative point cloud}

Let \(H \in \mathbb{N}\) denote the history size
(\texttt{PC\_HISTORY\_SIZE} in the implementation).
At time step \(k\), the node maintains a sliding window of the last
\(H\) transformed point clouds
\(\mathcal{Y}_{k-H+1}, \dots, \mathcal{Y}_k\).
The cumulative transformed cloud is defined as
\begin{equation}
  \mathcal{Y}_k^{\text{cum}}
  =
  \bigcup_{i = \max(1,\,k-H+1)}^{k}
  \mathcal{Y}_i
  =
  \bigcup_{i = \max(1,\,k-H+1)}^{k}
  \left\{
    \mathbf{y}_{i,j} \in \mathbb{R}^3
    \;\middle|\;
    j = 1,\dots,N_i
  \right\}.
\end{equation}

This cumulative set \(\mathcal{Y}_k^{\text{cum}}\) is published as a single
\texttt{PointCloud2} message on the topic \texttt{/cumulative\_point\_cloud}.

\paragraph{Summary of topic mappings.}
Given input point clouds \(\mathcal{X}_k\) on \texttt{/velodyne\_points}
and odometry poses \((\mathbf{R}_k, \mathbf{p}_k)\) on
\texttt{/odom}, the node computes:
\begin{align}
  \mathcal{X}_k
  &\xrightarrow{(\mathbf{R}_k, \mathbf{p}_k)}
  \mathcal{Y}_k
  &&\text{published on \texttt{/transformed\_point\_cloud},}
  \\
  \{\mathcal{Y}_i\}_{i=k-H+1}^k
  &\mapsto
  \mathcal{Y}_k^{\text{cum}}
  &&\text{published on \texttt{/cumulative\_point\_cloud},}
  \\
  \{\mathcal{Z}_i\}_{i=k-H+1}^k
  &\mapsto
  \mathcal{Z}_k^{\text{cum}}
  &&\text{published on \texttt{/cumulative\_origin\_point\_cloud}.}
\end{align}

\subsubsection{Timing and real-time metrics}

In addition to the geometric transformations, the node monitors several
timing-related quantities in order to assess its real-time behavior.

\subsubsubsection{Timestamps and latency}

For each synchronized pair of messages at step \(k\), we denote:
\begin{itemize}
  \item the LiDAR timestamp (from the \texttt{PointCloud2} header) by
  \(\tau^{\text{pc}}_k \in \mathbb{R}\),
  \item the ROS time when the synchronized callback receives the pair by
  \(\tau^{\text{in}}_k \in \mathbb{R}\),
  \item the ROS time when processing is completed by
  \(\tau^{\text{out}}_k \in \mathbb{R}\).
\end{itemize}

The end-to-end latency from measurement time to completion of processing
is then approximated as
\begin{equation}
  L_k
  =
  \tau^{\text{out}}_k - \tau^{\text{pc}}_k
  \quad [\text{s}],
\end{equation}
which in the implementation is logged as
\(\texttt{latency\_ms} = 1000 \, L_k\).

\subsubsubsection{Processing time}

Let \(\tau^{\text{start}}_k\) and \(\tau^{\text{end}}_k\) denote the wall-clock
times at the beginning and at the end of the processing loop for step \(k\).
The total processing time for this frame is
\begin{equation}
  T^{\text{proc}}_k
  =
  \tau^{\text{end}}_k - \tau^{\text{start}}_k
  \quad [\text{s}],
\end{equation}
which is logged as
\(\texttt{processing\_time\_ms} = 1000 \, T^{\text{proc}}_k\).

Inside this interval, we record the durations of individual stages:
\begin{align}
  T^{\text{pc}\rightarrow\text{points}}_k &\quad\text{: conversion from \texttt{PointCloud2} to an }(N_k \times 3)\text{ array}, \\
  T^{\text{transform}}_k &\quad\text{: CuPy-based rigid-body transformation of all points}, \\
  T^{\text{pc\_create}}_k &\quad\text{: creation of the transformed point cloud message}, \\
  T^{\text{cum}}_k &\quad\text{: update and stacking of the cumulative transformed cloud}, \\
  T^{\text{cum\_create}}_k &\quad\text{: creation of the cumulative transformed cloud message}, \\
  T^{\text{translate}}_k &\quad\text{: translation of points back to the origin-aligned frame}, \\
  T^{\text{cum-origin}}_k &\quad\text{: update and stacking of the cumulative origin-aligned cloud}, \\
  T^{\text{cum-origin\_create}}_k &\quad\text{: creation of the cumulative origin-aligned cloud message}.
\end{align}

Collectively,
\begin{equation}
  T^{\text{proc}}_k
  \approx
  T^{\text{pc}\rightarrow\text{points}}_k
  +
  T^{\text{transform}}_k
  +
  T^{\text{pc\_create}}_k
  +
  T^{\text{cum}}_k
  +
  T^{\text{cum\_create}}_k
  +
  T^{\text{translate}}_k
  +
  T^{\text{cum-origin}}_k
  +
  T^{\text{cum-origin\_create}}_k.
\end{equation}

\subsubsubsection{Input and processing rates}

To estimate the input rate of synchronized point cloud--odometry pairs,
we use the sequence of arrival timestamps
\(\{\tau^{\text{in}}_k\}\). For a window of indices
\(k = k_0, \dots, k_1\) with \(k_1 > k_0\), the average input rate is
\begin{equation}
  \lambda_{\text{in}}
  \approx
  \frac{k_1 - k_0}{\tau^{\text{in}}_{k_1} - \tau^{\text{in}}_{k_0}}
  \quad [\text{Hz}].
\end{equation}

Similarly, using a sequence of processed timestamps
\(\{\tau^{\text{pc}}_k\}\) associated with the processed frames, we estimate
the processing rate as
\begin{equation}
  \lambda_{\text{proc}}
  \approx
  \frac{k_1 - k_0}{\tau^{\text{pc}}_{k_1} - \tau^{\text{pc}}_{k_0}}
  \quad [\text{Hz}].
\end{equation}

These quantities are logged as \(\texttt{input\_rate\_Hz}\) and
\(\texttt{processing\_rate\_Hz}\) respectively. The throughput ratio is then
defined as
\begin{equation}
  \rho
  =
  \frac{\lambda_{\text{proc}}}{\lambda_{\text{in}}},
\end{equation}
and is logged as \(\texttt{throughput\_ratio}\).

\subsubsubsection{Queue occupancy and history size}

The internal processing queue has a finite capacity \(Q_{\max}\). Let
\(q_k \in \{0, 1, \dots, Q_{\max}\}\) denote the number of messages waiting
in the queue just before processing frame \(k\). This is logged as
\(\texttt{input\_queue\_size}\). The number of point clouds stored in the
sliding window for the cumulative clouds is bounded by
\(H = \texttt{PC\_HISTORY\_SIZE}\), and the current number of stored point
clouds at time step \(k\) is logged as \(\texttt{cumulative\_points}\).

\subsubsubsection{Real-time condition}

A necessary condition for real-time operation over a given interval is that
the system is able to keep up with the input rate, i.e.
\begin{equation}
  \rho = \frac{\lambda_{\text{proc}}}{\lambda_{\text{in}}} \gtrsim 1,
\end{equation}
so that the processing queue does not grow unbounded, and the latency
\(\{L_k\}\) remains within an application-dependent bound
\(L_{\max}\):
\begin{equation}
  L_k \leq L_{\max} \quad \forall k \text{ in the interval of interest.}
\end{equation}

In this way, the node not only applies consistent geometric transformations
to the input point clouds, but also quantitatively monitors its real-time
performance in terms of latency, processing time, throughput, and queue
occupancy.

\begin{table}[h!]
\centering
\caption{Summary statistics of GPU-accelerated point cloud transformation pipeline.}
\begin{tabular}{lccc}
\hline
\textbf{Metric} & \textbf{Mean} & \textbf{Median} & \textbf{Max} \\
\hline
Latency (ms) & 1.43 & 0.91 & 448.05 \\
Processing time per frame (ms) & 43.97 & 38.14 & 933.23 \\
Point cloud callback time (ms) & 43.81 & 38.13 & 933.17 \\
Transform step time (ms) & 8.90 & 8.73 & 54.73 \\
PC creation time (ms) & 0.62 & 0.61 & 30.89 \\
Input rate (Hz) & 9.80 & 10.02 & 14.29 \\
Processing rate (Hz) & 9.57 & 10.00 & 11.24 \\
Throughput ratio & 0.98 & 1.00 & 1.39 \\
Cumulative cloud size & 9.99 & 10.00 & 10.00 \\
Queue size & 0.00 & 0.00 & 5.00 \\
\hline
\end{tabular}
\end{table}


The experimental results demonstrate that the proposed GPU-accelerated
transformation pipeline achieves real-time performance for the target
sensor configuration. The average processing time per frame is
$43.97\,\text{ms}$, corresponding to a processing rate of approximately
$9.57\,\text{Hz}$, which closely matches the LiDAR input rate of
$9.80$–$10.02\,\text{Hz}$. The throughput ratio is $\rho = 0.98$ on
average, with a median value extremely close to unity ($0.9999$),
indicating that the system consistently processes point clouds as fast
as they arrive, without accumulating delay or queue backlog. This is
further confirmed by the queue occupancy remaining at zero for almost
all frames.

The transformation step itself benefits significantly from GPU
acceleration: the CuPy-based rigid-body conversion requires only
$8.9\,\text{ms}$ on average, enabling end-to-end latencies as low as
$0.9\,\text{ms}$ (median) and keeping mean latency at only
$1.43\,\text{ms}$. The cumulative point cloud construction and
origin-alignment steps also operate well within real-time constraints,
stabilizing at the maximum history length ($H=10$ frames) without
impacting throughput.

Overall, the measurements confirm that the system satisfies the real-time
condition $\lambda_{\text{proc}} \approx \lambda_{\text{in}}$ and maintains
latencies far below typical robotic perception budgets ($<100\,\text{ms}$),
demonstrating that GPU-accelerated point cloud processing is both
efficient and robust for continuous operation.



\section{Derinlik Haritası Uyumlama} \label{sec:depth_map_fitting}

Bu adımın amacı, odometri ile yoğunlaştırılmış LiDAR nokta bulutunu kamera düzlemine projekte ederek bir LiDAR derinlik imgesi üretmek ve stereo (ZED) derinlik haritası ile aynı piksel uzayında \textit{örtüşen (overlap)} bölgede tutarlı bir derinlik haritası elde etmektir. Bu işlem, PG yinelemelerinden bağımsız olarak, iki sensörün aynı geometri üzerinde hizalanmış tek bir derinlik alanı üretmesini sağlar.

\subsubsection{Gösterim ve Girdiler}

ZED stereo kameradan elde edilen derinlik görüntüsü aşağıdaki gibi tanımlanmaktadır:
\begin{equation}
    D_{\text{ZED}}^{\text{orig}}(v,u) \in \mathbb{R} \cup \{\text{NaN}\},
\end{equation}
burada piksel koordinatları
\begin{equation}
    u \in \{0,\dots,W-1\}, \quad v \in \{0,\dots,H-1\}
\end{equation}
şeklindedir.

LiDAR sensörü tarafından döndürülen üç boyutlu nokta bulutu
\begin{equation}
    \mathcal{P}_{\text{L}}
    =
    \left\{
        \mathbf{p}_k
    =
        \begin{bmatrix}
            x_k \\ y_k \\ z_k
        \end{bmatrix}
        \in \mathbb{R}^3
    \right\}_{k=1}^{N}
\end{equation}
olarak tanımlanmaktadır.

Kamera içsel parametreleri (ROS \texttt{CameraInfo} mesajından elde edilmektedir) şu şekildedir:
\begin{equation}
    f_x, f_y \quad \text{(odak uzaklıkları)}, \qquad
    c_x, c_y \quad \text{(ana nokta koordinatları)}.
\end{equation}

Derinlik işlemlerinde kullanılan kırpma bölgesi
\begin{equation}
    \Omega_c = \left\{ (v,u) \;\middle|\;
                   t \le v < H - b,\
                   \ell \le u < W - r
    \right\}
\end{equation}
olarak tanımlanır. Burada
\begin{equation}
\begin{aligned}
    t &= \texttt{MORTAL\_ROWS\_TOP} \\
    b &= \texttt{MORTAL\_ROWS\_BOTTOM} \\
    \ell &= \texttt{MORTAL\_COLUMNS\_LEFT} \\
    r &= \texttt{MORTAL\_COLUMNS\_RIGHT}
\end{aligned}
\end{equation}
parametreleri görüntüden çıkarılacak bölgeleri ifade etmektedir.

ZED ve LiDAR derinlikleri arasındaki tutarlılık eşiği ise
\begin{equation}
    \tau = \texttt{ZED\_VLP\_DIFF\_MAX}
\end{equation}
olarak tanımlanmaktadır.

\subsubsection{LiDAR Kartezyen Gösteriminden Küresel Gösterime Dönüşüm}

Her bir LiDAR noktası $\mathbf{p}_k = (x_k, y_k, z_k)^\top$ için küresel koordinatlar aşağıdaki şekilde tanımlanmaktadır:
\begin{align}
    r_k     &= \sqrt{x_k^2 + y_k^2 + z_k^2}, \\
    \theta_k &= \arctan\!\left( \frac{z_k}{\sqrt{x_k^2 + y_k^2}} \right), \\
    \phi_k   &= \arctan\!\left( \frac{y_k}{x_k} \right),
\end{align}
ve bu gösterim
\begin{equation}
    \mathbf{s}_k =
    \begin{bmatrix}
        r_k \\ \theta_k \\ \phi_k
    \end{bmatrix}
\end{equation}
şeklinde ifade edilir.

Bu küresel gösterim, uygulama kapsamında açısal filtreleme ve tanılama (diagnostics) amaçlı olarak kullanılmaktadır.

\subsubsection{LiDAR Noktalarının Kamera Düzlemine Yansıtılması}

\paragraph{Eksen Yeniden Eşlemesi.}
LiDAR noktaları öncelikle kamera benzeri bir koordinat sistemine dönüştürülmektedir:
\begin{equation}
    \begin{bmatrix}
        X_k \\ Y_k \\ Z_k
    \end{bmatrix}
    =
    R
    \begin{bmatrix}
        x_k \\ y_k \\ z_k
    \end{bmatrix},
    \qquad
    R =
    \begin{bmatrix}
        0 & -1 & 0 \\
        0 &  0 & -1 \\
        1 &  0 & 0
    \end{bmatrix}.
\end{equation}
Bu dönüşüm sonucunda $Z_k$ ekseni kameraya doğru olan ileri yönü, $X_k$ yatay ekseni ve $Y_k$ ise dikey ekseni temsil etmektedir.

\paragraph{İğne Deliği (Pinhole) Kamera Modeli.}
Kamera izdüşümü aşağıdaki denklemlerle elde edilmektedir:
\begin{align}
    \tilde{u}_k &= \frac{X_k f_x}{Z_k} + c_x, \\
    \tilde{v}_k &= \frac{Y_k f_y}{Z_k} + c_y.
\end{align}
Bu değerler, piksel indislerine yuvarlanarak dönüştürülür:
\begin{equation}
    u_k = \operatorname{round}(\tilde{u}_k), \qquad
    v_k = \operatorname{round}(\tilde{v}_k).
\end{equation}

Yalnızca kamera görüş alanında kalan ve kameranın önünde bulunan noktalar korunur:
\begin{equation}
    \mathcal{K}
    =
    \left\{ k \;\middle|\;
        0 \le u_k < W,\
        0 \le v_k < H,\
        Z_k > 0,\
        Z_k \text{ sonlu}
    \right\}.
\end{equation}

\paragraph{LiDAR Derinlik Görüntüsünün Oluşturulması.}
LiDAR derinlik görüntüsü, aynı piksele düşen noktalar arasından en yakın yüzeyin seçilmesiyle elde edilir:
\begin{equation}
    D_{\text{L}}(v,u) =
    \begin{cases}
        \displaystyle
        \min\limits_{k \in \mathcal{K} : (v_k,u_k) = (v,u)} Z_k,
        & \text{eğer böyle bir } k \text{ varsa}, \\[1.2ex]
        \text{NaN}, & \text{aksi halde}.
    \end{cases}
\end{equation}

Bu işlem, LiDAR nokta bulutundan elde edilen en yakın yüzeye karşılık gelen bir derinlik görüntüsü üretmektedir.

\subsubsection{ZED Derinlik Haritasının Doldurulması (Inpainting)}

Ham ZED derinlik haritası $D_{\text{ZED}}^{\text{orig}}(v,u)$ geçersiz değerler (NaN veya $\pm\infty$) içerebilmektedir. Bu nedenle bir geçersizlik maskesi şu şekilde tanımlanır:
\begin{equation}
    M(v,u) =
    \begin{cases}
        1, & \text{eğer } D_{\text{ZED}}^{\text{orig}}(v,u) \text{ NaN veya } \pm\infty \text{ ise}, \\
        0, & \text{aksi halde}.
    \end{cases}
\end{equation}

Başlangıç derinlik haritası
\begin{equation}
    D^{(0)}(v,u) =
    \begin{cases}
        D_{\text{ZED}}^{\text{orig}}(v,u), & M(v,u) = 0, \\
        0, & M(v,u) = 1
    \end{cases}
\end{equation}
şeklinde tanımlanır.

Geçersiz pikseller için, dört bağlantılı komşuluk
\begin{equation}
    \mathcal{N}(v,u) =
    \{(v-1,u), (v+1,u), (v,u-1), (v,u+1)\}
\end{equation}
kullanılarak yinelemeli bir doldurma işlemi uygulanır.

$t+1$ iterasyonunda güncelleme
\begin{equation}
    D^{(t+1)}(v,u)
    =
    \frac{
        \displaystyle
        \sum\limits_{(p,q) \in \mathcal{N}(v,u)}
        D^{(t)}(p,q)\,
        \mathbf{1}_{\{M(p,q) = 0\}}
    }{
        \displaystyle
        \sum\limits_{(p,q) \in \mathcal{N}(v,u)}
        \mathbf{1}_{\{M(p,q) = 0\}}
        + \varepsilon
    }
\end{equation}
şeklindedir. Burada $\varepsilon \ll 1$ sıfıra bölmeyi önlemek amacıyla eklenmiştir.

$T$ iterasyon sonunda doldurulmuş ZED derinlik haritası
\begin{equation}
    D_{\text{ZED}}(v,u) = D^{(T)}(v,u)
\end{equation}
olarak elde edilir.

\subsubsection{Kırpma ve ZED--LiDAR Tutarlılık Filtresi}

Derinlik haritaları kırpma bölgesi ile sınırlandırılır:
\begin{align}
    D_{\text{ZED}}^{c}(v,u) &= D_{\text{ZED}}(v,u), \\
    D_{\text{L}}^{c}(v,u)   &= D_{\text{L}}(v,u),
\end{align}
\quad $(v,u) \in \Omega_c$.

ZED ve LiDAR arasındaki tutarsız ölçümler şu koşula göre elenir:
\begin{equation}
    D_{\text{L}}^{c}(v,u) \leftarrow
    \begin{cases}
        D_{\text{L}}^{c}(v,u),
        & \left| D_{\text{ZED}}^{c}(v,u) - D_{\text{L}}^{c}(v,u) \right| \le \tau, \\
        \text{NaN},
        & \text{aksi halde}.
    \end{cases}
\end{equation}

Bu çalışmada, kırpma bölgesi içerisindeki DMF derinliği doğrudan filtrelenmiş LiDAR derinliği olarak alınmaktadır:
\begin{equation}
    D_{\text{fit}}^{c}(v,u) = D_{\text{L}}^{c}(v,u).
\end{equation}

Bu çıktı, bir sonraki füzyon adımına (örneğin Bölüm~\ref{sec:dynamic_pg} altında tanımlanan yayınlama/sonlandırma kuralı veya iteratif bir iyileştirme yöntemi) giriş olarak kullanılır.

\begin{figure}[H]
    \centering
    \begin{adjustbox}{width=\textwidth * 10 /12}
        \includegraphics{mermaid/depth_map_fitting.png}
    \end{adjustbox}
    \caption{DMF akış şeması}
    \label{fig:depth_map_fitting_flow}
\end{figure}
\section{Dinamik Papoulis-Gerchberg}  \label{sec:dynamic_pg}

Dinamik PG yapısı,

\subsection{Mathematical Formulation of the Depth Fusion Pipeline}

In this section, we formalize the mathematical model underlying our ROS--CuPy based ZED--LiDAR fusion node. The implementation operates on synchronized depth images from a stereo camera (ZED) and point clouds from a LiDAR sensor and produces a fused depth image and a corresponding point cloud in real time.

\begin{figure}[H]
    \centering
    \begin{adjustbox}{width=\textwidth}
        \includegraphics[width=\textwidth]{mermaid/pg.png}
    \end{adjustbox}
    \caption{Füzyon algoritması akış diyagramı 0}
    \label{methodology_2}
\end{figure}

\subsubsection*{Notation and Inputs}

Let the ZED depth image be
\begin{equation}
    D_{\text{ZED}}^{\text{orig}}(v,u) \in \mathbb{R} \cup \{\text{NaN}\},
\end{equation}
with pixel coordinates
\begin{equation}
    u \in \{0,\dots,W-1\}, \quad v \in \{0,\dots,H-1\}.
\end{equation}

The LiDAR returns a 3D point cloud
\begin{equation}
    \mathcal{P}_{\text{L}}
    =
    \left\{
        \mathbf{p}_k
    =
        \begin{bmatrix}
            x_k \\ y_k \\ z_k
        \end{bmatrix}
        \in \mathbb{R}^3
    \right\}_{k=1}^{N}.
\end{equation}

Camera intrinsics (from the ROS \texttt{CameraInfo} message) are given by
\begin{equation}
    f_x, f_y \quad \text{(focal lengths)}, \qquad
    c_x, c_y \quad \text{(principal point)}.
\end{equation}

We define a crop region
\begin{equation}
    \Omega_c = \left\{ (v,u) \;\middle|\;
                   t \le v < H - b,\
                   \ell \le u < W - r
    \right\},
\end{equation}
where
\begin{equation}
    t = \texttt{MORTAL\_ROWS\_TOP}, \quad
    b = \texttt{MORTAL\_ROWS\_BOTTOM}, \quad
    \ell = \texttt{MORTAL\_COLUMNS\_LEFT}, \quad
    r = \texttt{MORTAL\_COLUMNS\_RIGHT}.
\end{equation}
The consistency threshold between ZED and LiDAR depth is denoted by
\begin{equation}
    \tau = \texttt{ZED\_VLP\_DIFF\_MAX}.
\end{equation}

\subsubsection*{LiDAR Cartesian to Spherical Representation}

For each LiDAR point $\mathbf{p}_k = (x_k, y_k, z_k)^\top$, we define a spherical representation
\begin{align}
    r_k     &= \sqrt{x_k^2 + y_k^2 + z_k^2}, \\
    \theta_k &= \arctan\!\left( \frac{z_k}{\sqrt{x_k^2 + y_k^2}} \right), \\
    \phi_k   &= \arctan\!\left( \frac{y_k}{x_k} \right),
\end{align}
so that
\begin{equation}
    \mathbf{s}_k =
    \begin{bmatrix}
        r_k \\ \theta_k \\ \phi_k
    \end{bmatrix}.
\end{equation}
This representation is used for angular filtering and diagnostics in the implementation.

\subsubsection*{LiDAR to Camera Projection and Depth Image}

\paragraph{Axis Remapping.}
The LiDAR points are first remapped into a camera-like coordinate system via
\begin{equation}
    \begin{bmatrix}
        X_k \\ Y_k \\ Z_k
    \end{bmatrix}
    =
    R
    \begin{bmatrix}
        x_k \\ y_k \\ z_k
    \end{bmatrix},
    \qquad
    R =
    \begin{bmatrix}
        0 & -1 & 0 \\
        0 &  0 & -1 \\
        1 &  0 & 0
    \end{bmatrix},
\end{equation}
so that $Z_k$ is forward, $X_k$ is horizontal and $Y_k$ is vertical in the camera frame.

\paragraph{Pinhole Projection.}
Using the pinhole camera model, the projected image coordinates are
\begin{align}
    \tilde{u}_k &= \frac{X_k f_x}{Z_k} + c_x, \\
    \tilde{v}_k &= \frac{Y_k f_y}{Z_k} + c_y.
\end{align}
Discretization to pixel indices is performed by rounding:
\begin{equation}
    u_k = \operatorname{round}(\tilde{u}_k), \qquad
    v_k = \operatorname{round}(\tilde{v}_k).
\end{equation}

We only keep points that project to a valid pixel and lie in front of the camera:
\begin{equation}
    \mathcal{K}
    =
    \left\{ k \;\middle|\;
        0 \le u_k < W,\
    0 \le v_k < H,\
    Z_k > 0,\
    Z_k \text{ finite} \right\}.
\end{equation}

\paragraph{LiDAR Depth Image Construction.}
The LiDAR depth image $D_{\text{L}}(v,u)$ is defined as the minimum range $Z_k$ over all points that project to the corresponding pixel:
\begin{equation}
    D_{\text{L}}(v,u) =
    \begin{cases}
        \displaystyle
        \min\limits_{k \in \mathcal{K} : (v_k,u_k) = (v,u)} Z_k,
        & \text{if such } k \text{ exists}, \\[1.2ex]
        \text{NaN}, & \text{otherwise}.
    \end{cases}
\end{equation}
This corresponds to a nearest-surface depth image derived from the LiDAR point cloud.

\subsubsection*{ZED Depth Inpainting}

The raw ZED depth map $D_{\text{ZED}}^{\text{orig}}(v,u)$ may contain invalid values (NaN or $\pm\infty$). We define the invalidity mask
\begin{equation}
    M(v,u) =
    \begin{cases}
        1, & \text{if } D_{\text{ZED}}^{\text{orig}}(v,u) \text{ is NaN or } \pm\infty, \\
        0, & \text{otherwise}.
    \end{cases}
\end{equation}

We initialize an inpainted depth map by
\begin{equation}
    D^{(0)}(v,u) =
    \begin{cases}
        D_{\text{ZED}}^{\text{orig}}(v,u), & M(v,u) = 0, \\
        0, & M(v,u) = 1.
    \end{cases}
\end{equation}

Then we iteratively propagate valid neighbouring depths into invalid regions using a 4-connected neighbourhood
\begin{equation}
    \mathcal{N}(v,u) =
    \{(v-1,u), (v+1,u), (v,u-1), (v,u+1)\}.
\end{equation}
At iteration $t+1$, for each pixel with $M(v,u) = 1$, we update
\begin{equation}
    D^{(t+1)}(v,u)
    =
    \frac{
        \displaystyle
        \sum\limits_{(p,q) \in \mathcal{N}(v,u)}
        D^{(t)}(p,q)\,
        \mathbf{1}_{\{M(p,q) = 0\}}
    }{
        \displaystyle
        \sum\limits_{(p,q) \in \mathcal{N}(v,u)}
        \mathbf{1}_{\{M(p,q) = 0\}}
        + \varepsilon
    },
\end{equation}
where $\mathbf{1}_{\{\cdot\}}$ is the indicator function and $\varepsilon \ll 1$ is a small constant to prevent division by zero. Valid pixels ($M(v,u)=0$) remain unchanged in this iteration.

After a fixed number $T$ of iterations, we obtain the inpainted ZED depth map
\begin{equation}
    D_{\text{ZED}}(v,u) = D^{(T)}(v,u).
\end{equation}
The original invalid pixels are retained by reapplying the mask at the end.

\subsubsection*{Cropping and ZED--LiDAR Consistency Filtering}

We restrict the inpainted ZED depth and LiDAR depth to the crop region $\Omega_c$:
\begin{align}
    D_{\text{ZED}}^{c}(v,u) &= D_{\text{ZED}}(v,u),
    & (v,u) &\in \Omega_c, \\
    D_{\text{L}}^{c}(v,u)   &= D_{\text{L}}(v,u),
    & (v,u) &\in \Omega_c.
\end{align}

To reject inconsistent LiDAR measurements, we apply a consistency filter based on the ZED depth:
\begin{equation}
    D_{\text{L}}^{c}(v,u) \leftarrow
    \begin{cases}
        D_{\text{L}}^{c}(v,u),
        & \text{if } \left| D_{\text{ZED}}^{c}(v,u) - D_{\text{L}}^{c}(v,u) \right| \le \tau, \\[0.8ex]
        \text{NaN},
        & \text{otherwise},
    \end{cases}
    \qquad (v,u) \in \Omega_c.
\end{equation}

In the current implementation, the PG depth in the crop region is taken directly from the filtered LiDAR depth:
\begin{equation}
    D_{\text{PG}}^{c}(v,u) = D_{\text{L}}^{c}(v,u),
    \qquad (v,u) \in \Omega_c.
\end{equation}

\subsubsection*{Final Fused Depth Map}

The final fused depth map $D_{\text{PG}}(v,u)$, which is published as \texttt{PG\_DEPTH\_TOPIC}, is constructed as follows:

\begin{enumerate}
    \item Initialize with the original ZED depth:
    \begin{equation}
        D_{\text{PG}}(v,u) \leftarrow D_{\text{ZED}}^{\text{orig}}(v,u).
    \end{equation}

    \item In the crop region, overwrite with LiDAR depth wherever it is valid:
    \begin{equation}
        D_{\text{PG}}(v,u) =
        \begin{cases}
            D_{\text{L}}^{c}(v,u),
            & (v,u) \in \Omega_c,\ D_{\text{L}}^{c}(v,u) \text{ finite}, \\[0.8ex]
            D_{\text{ZED}}^{\text{orig}}(v,u),
            & \text{otherwise}.
        \end{cases}
    \end{equation}

    \item Enforce the original ZED invalidity mask:
    \begin{equation}
        D_{\text{PG}}(v,u) \leftarrow \text{NaN}
        \quad \text{for all } (v,u) \text{ such that } M(v,u) = 1.
    \end{equation}
\end{enumerate}

Combining these steps, we can concisely express the fused depth as
\begin{equation}
    D_{\text{PG}}(v,u) =
    \begin{cases}
        \text{NaN},
        & M(v,u) = 1, \\[0.8ex]
        D_{\text{L}}^{c}(v,u),
        & (v,u) \in \Omega_c,\ D_{\text{L}}^{c}(v,u) \text{ finite}, \\[0.8ex]
        D_{\text{ZED}}^{\text{orig}}(v,u),
        & \text{otherwise}.
    \end{cases}
\end{equation}

\subsubsection*{Back-Projection to 3D and Output Point Cloud}

The fused depth map $D_{\text{PG}}(v,u)$ is converted back to 3D using the camera intrinsics. For each pixel $(v,u)$ with a finite depth
\begin{equation}
    d = D_{\text{PG}}(v,u),
\end{equation}
the point in the camera frame is
\begin{align}
    X_{\text{cam}}(v,u) &= \frac{(u - c_x)\, d}{f_x}, \\
    Y_{\text{cam}}(v,u) &= \frac{(v - c_y)\, d}{f_y}, \\
    Z_{\text{cam}}(v,u) &= d.
\end{align}

The implementation applies an axis remapping so that the output frame uses
\begin{align}
    x'(v,u) &= Z_{\text{cam}}(v,u) = d, \\
    y'(v,u) &= -X_{\text{cam}}(v,u) = -\frac{(u - c_x)\, d}{f_x}, \\
    z'(v,u) &= -Y_{\text{cam}}(v,u) = -\frac{(v - c_y)\, d}{f_y}.
\end{align}
Thus, the resulting 3D point corresponding to pixel $(v,u)$ is
\begin{equation}
    \mathbf{p}'(v,u)
    =
    \begin{bmatrix}
        x'(v,u) \\
        y'(v,u) \\
        z'(v,u)
    \end{bmatrix}
    =
    \begin{bmatrix}
        d \\[0.4ex]
        -\dfrac{(u - c_x)\, d}{f_x} \\[1.0ex]
        -\dfrac{(v - c_y)\, d}{f_y}
    \end{bmatrix}.
\end{equation}

Collecting all valid pixels yields the fused point cloud
\begin{equation}
    \mathcal{P}_{\text{PG}}
    =
    \left\{
        \mathbf{p}'(v,u)
        \;\middle|\;
        D_{\text{PG}}(v,u) \text{ is finite}
    \right\},
\end{equation}
which is serialized as a \texttt{sensor\_msgs/PointCloud2} message and published on the fused point cloud topic.

\subsection{Sentetik Veri ile Algoritmanın Doğrulanması}
Gerçek sensör verisiyle yapılan iyileştirme denemelerinde gözlemlenen artefaktların (özellikle ``spike'' benzeri geometrilerin ve yakınsama kararsızlıklarının) kaynağını daha net ayırt edebilmek amacıyla, algoritmanın doğrulama aşamasında kontrollü bir sentetik veri seti kullanılmıştır. Bu doğrulama yaklaşımının temel motivasyonu; sensör gürültüsü, görüş alanı (FOV) uyuşmazlığı, eşzamanlılık sapmaları, hareket kaynaklı bozulmalar ve eksik örneklem gibi gerçek dünya etkilerini geçici olarak devre dışı bırakarak, Papoulis--Gerchberg (PG) tabanlı iyileştirme sürecinin ``beklenen'' davranışını gözlemleyebilmektir.

Bu kapsamda dama tahtası (satranç tahtası) formunda üretilmiş sentetik desen üzerinden iki ayrı kaynak veri hazırlanmıştır: (i) ZED stereo kamera temsili nokta/derinlik verisi ve (ii) VLP-16 LiDAR temsili nokta bulutu. Şekil~\ref{fig:synthetic_validation_triplet} içerisinde sırasıyla ZED temsili veri (Şekil~\ref{fig:dama_zed}), VLP temsili veri (Şekil~\ref{fig:dama_vlp}) ve PG çıktısı (Şekil~\ref{fig:dama_pg}) birlikte sunulmuştur. Bu deneyde, desenin yüksek kontrastlı ve düzenli geometrik yapısı sayesinde, PG algoritmasının frekans alanındaki kısıtlama ve uzamsal alandaki yeniden yapılandırma adımlarının ürettiği etki kolayca izlenebilmiştir.

\begin{figure}[H]
    \centering

    \begin{subfigure}[t]{0.6\textwidth}
        \centering
        \includegraphics[width=\linewidth]{./Figures/pg_synthetic_validation/dama_zed.png}
        \caption{ZED stereo kameradan elde edilen sentetik dama tahtası verisi}
        \label{fig:dama_zed}
    \end{subfigure}

    \vspace{0.6em}

    \begin{subfigure}[t]{0.6\textwidth}
        \centering
        \includegraphics[width=\linewidth]{./Figures/pg_synthetic_validation/dama_vlp.png}
        \caption{VLP-16 LiDAR’dan elde edilen sentetik dama tahtası verisi}
        \label{fig:dama_vlp}
    \end{subfigure}

    \vspace{0.6em}

    \begin{subfigure}[t]{0.6\textwidth}
        \centering
        \includegraphics[width=\linewidth]{./Figures/pg_synthetic_validation/dama_pg.png}
        \caption{Papoulis--Gerchberg (PG) algoritması sonrası elde edilen çıktı}
        \label{fig:dama_pg}
    \end{subfigure}

    \caption{Sentetik dama tahtası verisi ile ZED, VLP-16 ve PG çıktılarının karşılaştırılması. Bu deney, gerçek veri kaynaklı belirsizliklerden bağımsız olarak algoritmanın temel yakınsama davranışını doğrulamak amacıyla gerçekleştirilmiştir.}
    \label{fig:synthetic_validation_triplet}
\end{figure}

Görsellerde görüldüğü üzere LiDAR verisine karşılık üretilen sentetik veri satranç tahtası üzerinde ölçüm hassasiyeti çok yüksek ancak seyrek noktalardan oluşmaktadır. Benzer mantıkla stereo sentetik verisi ise yüksek çözünürlüklü ancak gürültülüdür.

Elde edilen sonuçlar, algoritmanın sentetik veri üzerinde kararlı bir yakınsama sergilediğini ve iki veri kaynağı arasında beklenen yönde bir uyum üretebildiğini göstermektedir. Ayrıca danışman değerlendirmesi kapsamında, elde edilen çıktının görsel olarak tutarlı bulunduğu ve algoritmanın temel çalışma prensibinin doğrulandığı not edilmiştir. Bu bulgu, gerçek veri ile benzer bir görsel çıktının elde edilememesi durumunda problemin doğrudan algoritmanın çekirdek mantığından ziyade, gerçek sensör koşullarına (özellikle ZED ve VLP-16 görüş alanlarının PG sürecinde etkin kullanılan ortak FOV ile tam örtüşmemesi, senkronizasyon/örnekleme farklılıkları ve tarihsel LiDAR birikiminin gölgeleme etkileri) bağlı olabileceğine işaret etmektedir.

Dolayısıyla bu doğrulama adımı, çalışmanın devamında gerçekleştirilen filtre tipi seçimi ve parametre ayarlama sürecine metodolojik bir temel sağlamaktadır. Sentetik veri deneyinde algoritmanın ``ideal'' koşullarda çalıştığının gösterilmesi sayesinde, bir sonraki aşamada ele alınan filtre seçimi (Brick-wall, Gaussian, Butterworth) ve cutoff/iterasyon gibi parametrelerin belirlenmesi, gerçek veride ortaya çıkan artefaktları azaltmaya ve yakınsamayı daha kararlı hâle getirmeye yönelik sistematik bir optimizasyon problemi olarak ele alınabilmiştir. Bu nedenle, sentetik doğrulama sonrasında Bölüm~\ref{sec:pg_filter_selection} ve Bölüm~\ref{sec:pg_parameter_tuning} altında sunulan filtre/parametre analizlerine geçilmiştir.

\subsection{3B İçin Filtre Seçimi}

Bu çalışmada, LiDAR (VLP-16) ve stereo kamera (ZED) verilerinin birlikte kullanıldığı üç boyutlu veri füzyonu sürecinde, Papoulis--Gerchberg (PG) tabanlı iyileştirme algoritmasının filtreleme parametrelerine olan duyarlılığı incelenmiştir. PG algoritması, eksik veya örtüşmeyen frekans bileşenlerinin iteratif olarak yeniden yapılandırılmasına dayandığından, özellikle frekans alanında uygulanan filtreleme işlemleri, sonuç geometrisi üzerinde belirleyici bir etkiye sahiptir.

Gerçek veri üzerinde yapılan deneylerde, sensörlerin anlık olarak algılayamadığı ancak zamansal olarak biriktirilmiş (historical) LiDAR noktalarının, PG algoritmasını fiziksel olarak anlamlı olmayan yakınsama davranışlarına zorladığı gözlemlenmiştir. Bu durum, düşük geçiren filtrelerin (Low-Pass Filter, LPF) yanlış yapılandırılması halinde, uzaysal alanda orijine doğru uzanan ve “spike” benzeri yapay geometrilerin ortaya çıkmasına neden olmaktadır. Bu bağlamda, filtreleme süreci yalnızca gürültü bastırma amacıyla değil, aynı zamanda sensörler arası tutarsızlıkların etkisini sınırlamak için ele alınmıştır.

\subsubsection{Cutoff Frekansını Ayarlama}

Filtreleme sürecinde kullanılan cutoff frekansı, Papoulis--Gerchberg algoritmasının yakınsama karakteristiğini doğrudan etkilemektedir. Bu çalışmada, farklı normalize cutoff frekans değerleri için (0.01--0.16 aralığında) sabit iterasyon sayısı altında elde edilen çıktılar karşılaştırılmıştır. Düşük cutoff değerlerinde, yüksek frekans bileşenlerinin büyük ölçüde bastırıldığı ve bunun sonucunda özellikle ince geometrik detayların kaybolduğu gözlemlenmiştir. Buna karşılık, yüksek cutoff değerlerinde ise sensörler arası uyumsuzluklardan kaynaklanan yüksek frekanslı bileşenlerin yeterince bastırılamadığı görülmüştür.

Sentetik satranç tahtası (checkerboard) verisi üzerinde gerçekleştirilen deneyler (Şekil~\ref{fig:pg_cutoff_comparison}), cutoff frekansının uygun seçilmesi halinde PG algoritmasının teorik beklentilerle uyumlu şekilde çalıştığını doğrulamıştır. Ancak gerçek veri üzerinde, ZED kamera ve VLP LiDAR sensörlerinin görüş alanlarının (Field of View, FOV) tam olarak örtüşmemesi nedeniyle, aynı cutoff değerlerinin her durumda benzer sonuçlar üretmediği tespit edilmiştir. Bu durum, cutoff frekansının sabit bir parametre olarak değil, sensör geometrisi ve veri dağılımına bağlı olarak değerlendirilmesi gerektiğini göstermektedir.

\begin{figure}[H]
    \centering

    % -------- Row 1 --------
    \begin{subfigure}[t]{0.48\textwidth}
        \centering
        \includegraphics[height=6cm]{./Figures/pg_filter_selection/dama_0.01_100.png}
        \caption{Cutoff = 0.01, Iter = 100}
        \label{fig:pg_001}
    \end{subfigure}
    \hfill
    \begin{subfigure}[t]{0.48\textwidth}
        \centering
        \includegraphics[height=6cm]{./Figures/pg_filter_selection/dama_0.02_100.png}
        \caption{Cutoff = 0.02, Iter = 100}
        \label{fig:pg_002}
    \end{subfigure}

    \vspace{0.8em}

    % -------- Row 2 --------
    \begin{subfigure}[t]{0.48\textwidth}
        \centering
        \includegraphics[height=6cm]{./Figures/pg_filter_selection/dama_0.04_100.png}
        \caption{Cutoff = 0.04, Iter = 100}
        \label{fig:pg_004}
    \end{subfigure}
    \hfill
    \begin{subfigure}[t]{0.48\textwidth}
        \centering
        \includegraphics[height=6cm]{./Figures/pg_filter_selection/dama_0.08_100.png}
        \caption{Cutoff = 0.08, Iter = 100}
        \label{fig:pg_008}
    \end{subfigure}

    \vspace{0.8em}

    % -------- Row 3 --------
    \begin{subfigure}[t]{\textwidth}
        \centering
        \includegraphics[height=6cm]{./Figures/pg_filter_selection/dama_0.16_100.png}
        \caption{Cutoff = 0.16, Iter = 100}
        \label{fig:pg_016}
    \end{subfigure}

    \caption{Papoulis--Gerchberg algoritmasında farklı normalize cutoff frekansları için elde edilen çıktılar.
    Düşük cutoff değerlerinde yüksek frekans bileşenlerinin baskılandığı, yüksek cutoff değerlerinde ise detayların
    daha belirgin hâle geldiği gözlemlenmektedir.}
    \label{fig:pg_cutoff_comparison}
\end{figure}

Ayrıca, derinlik imajlarının kare olmayan boyutlara sahip olması durumunda, frekans alanında kullanılan dairesel maskelemenin hatalı sonuçlar ürettiği gözlemlenmiştir. Bu problem, eliptik maske kullanımı ile giderilmiş ve dikdörtgen ve kare imajlar arasında tutarlı sonuçlar elde edilmiştir.

\subsubsection{Filtre Tipi Seçimi}

Varsayılan olarak kullanılan brick-wall (ideal) düşük geçiren filtre, frekans alanında keskin bir kesim uygulaması nedeniyle uzaysal alanda salınımlara (ringing artefaktları) yol açmaktadır. Bu durum, özellikle sensörler arası mesafenin fazla olduğu bölgelerde, PG çıktısında yapay dalgalanmalara neden olmaktadır. Bu nedenle, çalışmada brick-wall filtreye alternatif olarak Gaussian ve Butterworth filtreler değerlendirilmiştir.

Farklı filtre tiplerinin aynı parametrelerdeki çıktıları Şekil~\ref{fig:pg_filter_comparison} içinde birlikte sunulmuştur. Gaussian filtre (Şekil~\ref{fig:pg_gaussian}), frekans alanında yumuşak bir geçiş sağladığından, yüksek frekans bileşenlerini kademeli olarak bastırmakta ve uzaysal alanda daha kararlı sonuçlar üretmektedir. Butterworth filtre (Şekil~\ref{fig:pg_butterworth}) ise geçiş bölgesinin eğimini kontrol edilebilir kılması sayesinde, brick-wall filtreden (Şekil~\ref{fig:pg_brickwall}) daha yumuşak, Gaussian filtreden ise daha seçici bir yapı sunmaktadır. Yapılan deneyler, her iki filtrenin de brick-wall filtreye kıyasla, özellikle spike oluşumlarını belirgin şekilde azalttığını göstermiştir.

Bununla birlikte, Gaussian ve Butterworth filtrelerin kullanımı sonucunda, zemin düzlemi gibi LiDAR verisinin yoğun olduğu bölgelerde bozulmalar meydana geldiği gözlemlenmiştir. Bu durum, filtreleme işleminin tüm sahneye homojen biçimde uygulanmasının her zaman ideal olmadığını ve baskın geometrik yapıların ayrı ele alınması gerektiğini ortaya koymaktadır.

\begin{figure}[H]
    \centering

    % ---------- Row 1 ----------
    \begin{subfigure}[t]{0.48\textwidth}
        \centering
        \includegraphics[height=7cm]{./Figures/pg_filter_selection/dama_gaussian_0.64_2000.png}
        \caption{Gaussian LPF çıktısı}
        \label{fig:pg_gaussian}
    \end{subfigure}
    \hfill
    \begin{subfigure}[t]{0.48\textwidth}
        \centering
        \includegraphics[height=7cm]{./Figures/pg_filter_selection/dama_butterworth_0.64_2000.png}
        \caption{Butterworth LPF çıktısı}
        \label{fig:pg_butterworth}
    \end{subfigure}

    \vspace{0.8em}

    % ---------- Row 2 ----------
    \begin{subfigure}[t]{0.48\textwidth}
        \centering
        \includegraphics[height=7cm]{./Figures/pg_filter_selection/dama_brick-wall_0.64_2000.png}
        \caption{Brick-wall LPF çıktısı}
        \label{fig:pg_brickwall}
    \end{subfigure}

    \vspace{0.8em}

    % ---------- Row 3 ----------
    \begin{subfigure}[t]{\textwidth}
        \centering

        \renewcommand{\arraystretch}{1.15}
        \begin{tabular}{lccc}
            \toprule
            \textbf{Filtre Tipi} & \textbf{Cutoff} & \textbf{Iterasyon} & \textbf{Notlar}            \\
            \midrule
            Gaussian             & 0.64            & 2000               & Yumuşak geçiş, düşük spike \\
            Butterworth          & 0.64            & 2000               & Dengeli yapı               \\
            Brick-wall           & 0.64            & 2000               & Keskin geçiş, artefakt     \\
            \bottomrule
        \end{tabular}

        \caption{Filtre parametreleri}
        \label{fig:pg_filter_table}
    \end{subfigure}

    \caption{Papoulis--Gerchberg algoritması için farklı düşük geçiren filtrelerin karşılaştırılması.
    Sarı noktalar ZED, yeşil noktalar VLP, uzaklığa göre geçişli renklerle gösterilen noktalar füzyon çıktısı verileri temsil etmektedir.}
    \label{fig:pg_filter_comparison}
\end{figure}

\subsubsection{Filtrelerin Çapraz Karşılaştırılması}

Filtre tiplerinin karşılaştırmalı analizi hem sentetik hem de gerçek veri üzerinde gerçekleştirilmiştir. Sentetik veride, ideal koşullar altında filtrelerin benzer yakınsama davranışı sergilediği görülürken, gerçek veri üzerinde belirgin farklar ortaya çıkmıştır. Brick-wall filtre, frekans alanında keskin bir kesim uyguladığı için PG çıktısında artefaktlara ve yer yer kararsız yakınsamaya neden olmuştur. Bu çalışmada tercih edilen Gaussian filtre ise yumuşak geçiş karakteristiği sayesinde sensörler arası uyumsuzlukları daha iyi tolere etmiş ve görsel olarak daha kararlı, daha tutarlı bir iyileştirme sağlamıştır. Butterworth filtre belirli senaryolarda dengeli bir alternatif sunsa da, yapılan deneylerde en tutarlı sonuçlar Gaussian filtre ile elde edilmiştir (Bakınız Şekil~\ref{fig:pg_gaussian_vs_brickwall_citrus}).

\begin{figure}[H]
    \centering

    \begin{subfigure}[t]{0.48\textwidth}
        \centering
        \includegraphics[height=5.5cm]{./Figures/pg_filter_selection/pg_nan_filled_1.png}
        \caption{Brick-wall LPF ile elde edilen PG çıktısı (Kırmızı noktalar PG, sarı noktalar ZED, yeşil noktalar VLP çıktısı)}
        \label{fig:pg_brickwall_not_variance_filtered}
    \end{subfigure}
    \hfill
    \begin{subfigure}[t]{0.48\textwidth}
        \centering
        \includegraphics[height=5.5cm]{./Figures/pg_filter_selection/variance_filtered.png}
        \caption{Brick-wall LPF ve Varyans Filtresi ile elde edilen PG çıktısı (Kırmızı noktalar PG, sarı noktalar ZED, yeşil noktalar VLP çıktısı)}
        \label{fig:pg_brickwall_variance_filtered}
    \end{subfigure}

    \vspace{0.8em}

    \begin{subfigure}[t]{\textwidth}
        \centering
        \includegraphics[width=0.5\linewidth]{./Figures/pg_filter_selection/citrus_gaussian_0.16_33.png}
        \caption{Gaussian LPF ve Varyans Filtresi ile elde edilen PG çıktısı (Beyaz noktalar ZED, kırmızı noktalar VLP, gökkuşağı noktalar PG çıktısı)}
        \label{fig:pg_gaussian_citrus}
    \end{subfigure}

    \caption{Citrus Farm verisinde benzer frame’ler üzerinde Papoulis--Gerchberg (PG) çıktılarının karşılaştırılması.
    Brick-wall düşük geçiren filtre kullanıldığında düzeltme süreci daha belirgin artefaktlar (örn. küçük spike yapıları)
    üretebilirken, Gaussian filtre kullanımı daha yumuşak geçiş sağlayarak yakınsamanın görsel kararlılığını artırmıştır.
    Bu nedenle, ilgili deney setinde Gaussian filtre, Brick-wall filtreye kıyasla daha tutarlı bir PG iyileştirmesi sunmuştur.}
    \label{fig:pg_gaussian_vs_brickwall_citrus}
\end{figure}

Ayrıca, ZED ve VLP referans çerçeveleri arasında hesaplanan varyans (covariance) temelli maskeleme yaklaşımı, filtreleme süreciyle birlikte kullanıldığında, yüksek varyanslı LiDAR noktalarının PG güncellemesine dahil edilmemesi sayesinde daha kararlı sonuçlar elde edilmesini sağlamıştır. Bu yaklaşım, algoritmanın zorlayıcı ve fiziksel olarak anlamsız düzeltmeler yapması yerine, yalnızca güvenilir bölgelerde iyileştirme gerçekleştirmesine olanak tanımaktadır.

Sonuç olarak, bu çalışmada elde edilen bulgular, üç boyutlu veri füzyonunda filtre seçiminin yalnızca matematiksel değil, aynı zamanda fiziksel sahne yapısı ve sensör karakteristikleriyle birlikte değerlendirilmesi gerektiğini göstermektedir. Filtre tipi, cutoff frekansı ve maskeleme stratejilerinin birlikte ele alındığı bir yapı, Papoulis--Gerchberg tabanlı iyileştirme algoritmasının gerçek veri üzerindeki başarımını anlamlı ölçüde artırmaktadır.

