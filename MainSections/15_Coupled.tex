
\section{BAĞLAŞIK METOTLAR ve İLERİ ÇALIŞMA} \label{coupled}

%\epigraph{All human things are subject to decay, and when fate summons, Monarchs must obey}{\textit{Mac Flecknoe \\ John Dryden}}

\epigraph{All that is gold does not glitter,\\Not all those who wander are lost.}{---Bilbo Baggins, \textit{Riddle of Strider}}


%\hfill 

Çalışma sürecinde, LOAM algoritması ile EKF'yi bağlaşık (coupled) olarak kullandığımız sensör füzyonu metotları da denedik. Bağlaşık metotların diğer metotlardan farkı, sensör füzyonunu ham veri ile yapmasıdır. Biz buradaki metotlarda haritalama adımında EKF sonucunu da optimizasyon sürecine dahil ederek doğruluğu artırmayı amaçladık. 

LiDAR odometrisini EKF'ye ölçüm olarak verirken yine Hausdorff Mesafesi ile elde ettiğimiz varyans değerini kullandık. Ayrıca EKF algoritması olarak robot\_localization ROS paketini\cite{robot_localization}, LiDAR ölçümü güncellemesi sonrası sonuç verecek şekilde modifiye ederek kullandık. 

Veriseti olarak ise, verileri .bag dosyası olarak kaydettiği için ROS ile kullanımı daha rahat olan M2DGR\cite{m2dgr} verisetinin street4 sekansını kullandık. Bu verisetinin MulRan verisetine göre bir diğer avantajı, ROS REP-105'i\cite{ros_rep_105} (Mobil Platformlar için Koordinat Sistemleri - Coordinate Frames for Mobile Platforms) IMU varyans bilgisi haricinde uygulamasıdır. Bu eksik bilgiyi ise IMU'nun veri kağıdından\cite{handsfree_imu} elde ettik.

\subsection{Kalman Filtresi ile İlklendirilmiş Haritalama}

İlk denediğimiz bağlaşık metot LOAM'ın haritalama adımında optimizasyon döngüsünü Kalman Filtresi'nin çıktısında elde edilen pozisyonda ilklendirerek (initialization) başlatmaktı. Ancak bu metotta elde ettiğimiz pozisyon değerleri referans pozisyona yakın olmakla beraber, harita orijinal algoritma ile karşılaştırıldığında çok fazla gürültü içermiştir. Haritadaki bu gürültü, zaman geçtikçe birikerek, algoritmanın yer noktası bulutunu yanlış eşlemesine ve sonuç olarak özellikle oryantasyon değerlerinde gerçek değerden sapmasına neden olmuştur.

ICP algoritmasının en kronik sorunlarından biri, doğru pozisyonda ilklendirilmemesi durumunda yerel minimumlara takılma eğilimi göstermesidir, hatta Brossard ve ark.\cite{icp_cov_2} bunun genellikle en dominant hata nedeni olduğunu belirtmekte ve bunu Şekil (\ref{fig:icp_dist})'deki gibi göstermektedir.

\begin{figure}
    \centering
    \includegraphics[width=0.75\linewidth]{Figures/icp_dist.png}
    \caption[Doğru ve Dağınık pozisyanlarda iklendirilmiş ICP kestirimleri]{Doğru (sol) ve Dağınık (sağ) pozisyanlarda iklendirilmiş ICP kestirimleri sonuçları. Siyah noktalar ilklendirme pozisyonlarını, Kırmızı noktalar ICP kestirimlerini, M0
    avi noktalar ise gerçek kestirim sonucunu göstermektedir\cite{icp_cov_2}.}
    \label{fig:icp_dist}
\end{figure}

LOAM kestirimlerini ICP kullanmadan yapmakla beraber aynı problemi benzer şekilde çözmesi nedeniyle benzer sorunlardan muzdarip olması beklenebilir. Öte yandan LiDAR Odometrisi adımında sadece önceki nokta bulutunun referans alınması, bu bulutta daha az nokta bulunması nedeniyle problemde bulunan yerel minimum sayısını azaltarak global minimuma ulaşmayı mümkün kılıyor olabilir. Örneklem sayısının azaltılmasının yerel minimumlara etkisi Şekil (\ref{fig:sampling_minima})'de gösterilmiştir.

\begin{figure}[!htb]
    \centering
    \includegraphics[width=0.95\linewidth]{Figures/samling_minima.png}
\caption[Örneklem Sayısının Yerel Minimumlara Etkisi]{
Örneklem sayısının azaltılması, yerel minimum sayısını azaltmakta ve uzaydaki bir noktadan global minimuma ulaşma ihtimalini arttırmaktadır. Şekilde \text{\(f(x, y) = \sin(3x)\cos(3y) + \frac{1}{\sqrt{(x + 2.5)^2 + y^2 + 1}} - \frac{1}{\sqrt{(x - 2.5)^2 + y^2 + 1}}\)}
 fonksiyonunun 0.1 (sol) ve 0.01 (sağ) birim aralıklarda örneklenmiş hali görülmektedir. Bu örneklem noktalarından gradyan vektörünün tersi yöndeki noktaya ilerleyerek global minimuma ulaşabilen noktalar mavi, ulaşamayanlar kırmızı renkte gösterilmiştir. Şekilden de anlaşılacağı üzere, daha küçük aralıklarla örnekleme yapıldığında, yerel minimuma takılma ihtimali artmaktadır.
}\label{fig:sampling_minima}
\end{figure}

Haritalama algoritmasının ilklendirmesini Kalman filtresinin çıktısı ile yapmak, algoritmayı bir yerel minimuma sokuyorken, bu ilklendirme pozisyonunu, minimumlara takılma ihtimali düşük dönüşümlerin sonucu olarak almak algoritmayı global minimuma yakınsayabileceği bir noktadan başlatıyor olabilir. Bu durumu, bir bilyeyi engebeli bir yüzeyde ilerletmek yerine, bir bilye makinasında ard arda gelen huniler ile hedef pozisyona yönlendirmek gibi düşünebiliriz. Burada optimizasyon probleminin global minimum değerinin, gerçek gerçek pozisyona yakın olmakla beraber, tam olarak aynı sonucu vermiyor olabileceğine belirtmemiz gerekir.

\textbf{Ancak bu konuda yapılan açıklama bir hipotez durumundadır ve bu konuda yapılabilinecek daha detaylı çalışmalar, tezin kapsamının dışına çıkmaktadır.}

\subsection{Kalman Filtresi Sonucu Destekli Maliyet Fonksiyonu}
Guadagnino ve ark. Kinematic-ICP\cite{kinematic_icp} makalesinde, LiDAR odometrisini teker odometrisinin sonucunda elde ettikleri pozisyonda ilklendirmekte ve LiDAR odometrisinin bu pozisyona göre hesapladığı yer değiştirme miktarını da maliyet fonksiyonuna eklemektedirler. Buradaki amaçları, 2 Boyutta çalışıyor olmaları nedeniyle yüksek yer değiştirme ve düşük oryantasyon değişimi doğruluğuna sahip tekerlek odometrisi bilgisinden istifade etmektir.

Bu metotta, bu çalışmadan esinlenerek, haritalama adımında çözülen optimizasyon problemine, Kalman Filtresi'nin sonucundan yapılan sapmayı ekleyerek, LOAM algoritmasında zamanla oluşacak hata birikimini azaltmayı amaçladık. Bu işlem sonucunda elde ettiğimiz yeni maliyet fonksiyonu Denklem (\ref{eq:opt_w_kalman_1}-\ref{eq:opt_w_kalman_4})'de verilmiştir.

\begin{align}
    \widehat{E}_k &= \widehat{P}_{\text{LiDAR},k}^0 - \widehat{P}_{\text{Kalman},k}^0 \label{eq:opt_w_kalman_1}\\
    W_k &= \frac{diag(cov(\widehat{P}_{\text{Kalman},k}^0))}{\sqrt[3]{det(cov(\widehat{P}_{\text{Kalman},k}^0))}} \label{eq:opt_w_kalman_2}\\
    d_\mathcal{G} &= \left| W_k^{-1} \times \widehat{E}_k \right| \label{eq:opt_w_kalman_3}\\
    \widehat{T}_k &= \min_{\mathbf{T}_{k}} \sum^N_{i=1} \Big( d_\mathcal{E}^2 + d_\mathcal{H}^2 \Big) + \sqrt{N}\times d_\mathcal{G}^2 \label{eq:opt_w_kalman_4}
\end{align}

Burada öncelikle LiDAR kestiriminin Kalman Filtresi kestiriminden ne kadar saptığını bulduk (Denklem (\ref{eq:opt_w_kalman_1})). Daha sonra ise Kalman filtresinin kovaryans matrisini normalize ederek, ağırlık matrisini (\(W_k\)) elde ettik (Denklem (\ref{eq:opt_w_kalman_2})). Bu ağırlık matrisi, ilk denklemde hesapladığımız hatanın hangi eksenine ne kadar güvenebileceğimizi göstermektedir. Denklem (\ref{eq:opt_w_kalman_3})'de hesapladığımız ağırlık matrisinin tersi ile çarpıp sonucun mutlak değerini aldık. Bu işlem aslında bir baz değişimi işlemi olarak görülebilir. Bir eksendeki varyans değeri ne kadar büyükse, o eksenin efektif uzunluğunu ve dolayısıyla maliyete etkisini o oranda azalttık. Son olarak elde ettiğimiz bu yeni maliyeti \(d_\mathcal{G}^2\), maliyet fonksiyonumuza ekledik. Ancak burada maliyet fonksiyonunda bulunan karşılıklılık sayısının kökü \(\sqrt{N}\) ile çarparak ağırlığını arttırdık. 
Burada \(N\) yerine \(\sqrt{N}\) kullanmamızın 2 nedeni vardır. Bunlardan birincisi, \(d_\mathcal{G}^2\)'nin \(d_\mathcal{H}^2\) ve \(d_\mathcal{E}^2\)'te göre daha büyük olması nedeniyle, optimizasyon problemine eşit ağırlık ile eklendiği durumda problemi domine etmesidir. İkincisi ise \(N\) değerinin büyük olması, bulunan öznitelik sayısının fazla olduğu anlamına gelmektedir ve bu da LOAM'ın performansını arttırmaktadır.

\begin{table}[!ht]
\centering
\begin{tabular}{|l|r|r|r|r|r|r|}
\hline
\textbf{} & \textbf{RMSE} & \textbf{Ort.} & \textbf{Medyan} & \textbf{Std.} & \textbf{Min} & \textbf{Max} \\ \hline
\textbf{Kalman\_Destekli}    & 14.7150 & 13.1506 & 13.5750 & 6.6025 & 0.7352 & 31.4388 \\ \hline
\textbf{LeGO\_LOAM}          & 14.6938 & 13.1271 & 13.5175 & 6.6019 & 0.6680 & 31.2379 \\ \hline
\textbf{LeGO\_LOAM\_w\_IMU}  & 0.9959  & 0.7999  & 0.5255  & 0.5934 & 0.0111 & 2.3625 \\ \hline
\end{tabular}
\caption{Mutlak Hata Değerleri}
\label{tab:mutlakHata}
\end{table}

\begin{table}[!ht]
\centering
\begin{tabular}{|l|r|r|r|r|r|r|}
\hline
\textbf{} & \textbf{RMSE} & \textbf{Ort.} & \textbf{Medyan} & \textbf{Std.} & \textbf{Min} & \textbf{Max}\\ \hline
\textbf{Kalman\_Destekli}    & 0.1831 & 0.1286 & 0.0951 & 0.1303 & 0.0000 & 1.1009 \\ \hline
\textbf{LeGO\_LOAM}          & 0.1882 & 0.1306 & 0.0905 & 0.1355 & 0.0002 & 1.1595 \\ \hline
\textbf{LeGO\_LOAM\_w\_IMU}  & 0.0588 & 0.0450 & 0.0350 & 0.0378 & 0.0000 & 0.2372 \\ \hline
\end{tabular}
\caption{Görece Hata Değerleri}
\label{tab:goreceHata}
\end{table}

Sonuçlar bir önceki kısımdaki gibi evo python paketi ile alınmıştır. Buradaki tek fark hizalamanın tüm veriyi hizalayacak şekilde yapılmasıdır. Önerilen yöntem, yalın LiDAR odometrisinde bir miktar gelişme sağlamıştır, ancak bu metodun başarım arttırımı, IMU verisinin kayma giderme sürecindeki başarım arttırımının altında kalmaktadır. Bu durum LiDAR odometrisi için kayma giderme sürecinin ne kadar kritik olduğunu bir kez daha göz önüne sermiştir.

Buradaki verisetindeki LiDAR odometri başarısının önceki kısımdan daha iyi olmasının bir nedeni ise senaryolar arasındaki farktır. MulRan verisetinde, LiDAR'ın arkasına konumlanan Radar, LiDAR'ın görüş açısının yaklaşık 70 derecelik bir bölümünü kapatmaktadır. Özellikle riverside verisetlerinde LiDAR'ın kalan görüş açısın yaklaşık yarısı çok fazla öznitelik içermeyen nehir bölgesine bakmaktadır. Ayrıca bu veri setinde bir otomobil üzerinde daha yüksek hız ve ivmelerde veri alınması da LiDAR odometrisini olumsuz etkilemektedir.

\subsection{İleri Çalışma}

Bu çalışmada, Hausdorff Mesafesinin varyans kestirimi için kullanılabilir sonuçlar sunduğu ortaya konulmuştur. Öte yandan Optimal Ortalama Metodunun sonucu ile arasında hala gelişmeye açık bir alan mevcuttur. İlerleyen çalışmalarda farklı hata metrikleri, bu çalışmada kullanılan metotlar kullanılarak denenebilir. Örneğin, P\(^3\) LOAM makalesinde SVD ile hatanın yayılımı hesaplanarak varyans değeri kestirimi yapılmıştır. 

Kullanabilinecek bir diğer yöntem, öznitelik noktalarının ait oldukları doğru yada düzleme olan normal vektörlerinin kovaryansı olabilir. Buna benzer bir diğer alternatif ise, KISS-ICP ve Kinematic-ICP makalelerinin maliyet fonksiyonunda kullandığı en yakın noktaya olan ortalama uzaklık yada uzaklık vektörlerinin kovaryansı olabilir. 

Bu konuda potansiyeli olabilecek bir diğer metrik ise, özellikle son yıllarda makina öğrenmesi alanında sıkça kullanılan MMD (Maximum Mean Discrepancy - Maksimum Ortalama Farklılık)\cite{MMD_orig} metriğidir. Ancak bu yöntem makina öğrenmesi dışındaki alanlarda da uygulanmış ve başarılı sonuçlar elde edilmiştir\cite{mmd_osman}. Kısaca, iki verinin dağılımlarının ortalaması arasındaki farkı ölçen bu yöntem, eğer iki nokta bulutu düzgün bir şekilde eşlenirse düşük bir değer vermelidir. Özellikle ortalama üzerinden ölçüm yapması nedeniyle ayrık verilere karşı düşük hassasiyete sahip olması bu metriği Hausdorff Mesafesi metriğinin önüne çıkarabilir. Ancak burada dikkat edilmesi gereken konu, özellikle harita bulutunda haritanın keşfedilmiş tarafındaki nokta yoğunluğunun, keşfedilmemiş taraflarından fazla olması nedeniyle dönme ekseni etrafında normalize edilmesi gerekliliğidir. Bu konuda ise Giseop Kim ve ark.'nın Scan Context\cite{scan_context} yönteminde kullandıkları bakış açısı faydalı olabilir.


\pagebreak