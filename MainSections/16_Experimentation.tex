\chapter{DENEY} \label{chap:experimentation}

\section{Yöntem}

Dinamik Papoulish-Gerchberg algoritmasının SLAM'e katkısını ölçülebilir kılmak için bu tez kapsamında oluşturulan SensorSuiteV2 ve benzer veri tiplerini içeren açık kaynak CitrusFarm versiseti kullanılmıştır. Bu verisetlerindeki veriler RTABMap aracı kullanılarak doğrudan veya algoritmayla ön işlemeden geçirilerek işlenmiştir.

Algoritma katkısının hangi algoritma adımlarıca majör olarak sağlandığını bulabilmek için deneyler superposition prensibine uygun olarak decompose edilmiştir. Ön işleme (PG algoritması) farklı adımları bypass geçilerek geriye kalan kısımlarının iyileşmeye katkısı değerlendirilmiştir. En sonunda kümülatif katkının süperpozisyon prensipine uygun olup olmadığı değerlendirilmiştir.

Deney esnasında decompose edilen parçalar şu şekildedir:

\begin{itemize}
    \item Bütünleşik algoritmanın çıktısı derinlik imajı
    \item PG iterason sayısını N adet discrete değer ile üretilen derinlik imajı (N çeşitlendirerek iyileşme lineerlinin test edilmesi amaçlanmıştır N=[0, 33])
    \item Lidar upsampleda kullanilan geriye dönük lidar frame sayısının çeşitlendirilmesi algoritma çıktısı derinlik imajı N[1,10]
\end{itemize}

Tüm bu deney çıktılarının SLAM performanlarının karşılaştırılması için odometri çıktıları birleştirilerek elde edilen güzergahlar RTK GPS verisi ile karşılaştırılmıştır.

Kendi verisetimizde loop-closure içeren kısımları da barındırdığımız için path örtüşmesi kıyasının yanı sıra aynı pozisyondan geçerkenki drift değerleri de kıyaslanmıştır.

SLAM çıktısının bir diğeri üç boyutlu harita olduğu için aynı zamanda üç boyutlu harita performansları da kıyaslanmıştır. Burada temel ölçü olarak harita çözünürlüğü, harita alignment hatası ve bir de gözle yapılan kıyasların görsellerine tez kapsamında yer verilmiştir.Bahsedilen son metrik öznel yargılar işerebileceği gibi ektra eklendiği için bilimsel bir kanıt amacı günülmemiş, yalnızca bir bakış açısı katılması amaçlanmıştır.

Deneyde yalnızca Lidar ve IMU tabanlı SLAM algoritmaları ile karşılaştırmaya girilmemiştir. Bunun temel sebebi Lidarın seyrek veri yapısının renksiz ve düşük çözünürlüklü SLAM çıktısının stereo depth ile elde edilen SLAM çıktısıyla karşılaştırılmasının çok fazla metrik değişiminden dolayı sağlıksız olmasıdır.

\subsection{Bütünleşik algoritma parametreleri ve akışı}

Algoritma baştan uca tüm parametrelerin en iyi sonucu vereceği tahmin edilen dahiliyle deneye sokulmuştur.

Bu başlık altında test edilen şeyin ne tür bir füsyon olduğu görsellerle anlatılmalıdır.

\subsection{Değişken PG iterasyon sayısı}

Bu başlık altında test edilen şeyin ne tür bir füsyon olduğu görsellerle anlatılmalıdır.

\subsection{Değişken Lidar upsample Lidar History sayısı}

Bu başlık altında test edilen şeyin ne tür bir füsyon olduğu görsellerle anlatılmalıdır.

Alt başlıklarda verilen deney yöntemleri tek bir tablo altına toplanmış ve matris yapı ile çaprazlanarak deneyler tamamlanmıştır.
Bu deneylerin sonuçları grafikler ve tablolar ile nicel olarak ele alınmalıdır.
APE ve benzeri yöntemler ile ilgili numerik çıktılar buraya eklenmelidir.
XYZ güzergah karşılaştırmaları eklenmelidir.

\section{Deneysel Analiz İçin Matematiksel Çerçeve}
\label{sec:math_analysis_framework}

Bu çalışma, önerilen Papoulis--Gerchberg (PG) tabanlı LiDAR--stereo füzyon hattını, algoritmik konfigürasyonlar üzerinde yapılandırılmış bir tarama (sweep) kapsamında değerlendirmektedir. Amaç yalnızca tek bir ``en iyi'' parametre kümesini seçmek değil; RTK tabanlı GNSS referansı (yer gerçeği) ile karşılaştırılan yörüngelerde, yöntemin iç parametrelerinin odometrik doğruluğa \emph{bağımsız} ve \emph{birlikte} nasıl katkı sunduğunu nicel olarak karakterize etmektir.

\subsection{Problem Tanımı ve Değerlendirme Kurulumu}

Önerilen yöntem iki bağımsız kontrol parametresi tarafından belirlenmektedir:

\begin{itemize}
    \item $I \in \mathbb{N}$: PG iyileştirme iterasyon sayısı,
    \item $H \in \mathbb{N}$: zamansal koşullandırma için kullanılan LiDAR upsampling (LU) geçmiş (history) boyutu.
\end{itemize}

Her bir $(I,H)$ konfigürasyonu için bir kestirim yörüngesi üretilir ve \texttt{evo} kullanılarak RTK referans yörüngesine karşı değerlendirilir. Bu karşılaştırmadan elde edilen skaler bir performans metriği
\begin{equation}
M(I,H)
\end{equation}
ile gösterilsin. Değerlendirme bağlamına bağlı olarak $M$, APE (ör. RMSE), belirli bir uzamsal/zamansal ölçekte RPE, yaw ile ilişkili bir hata ölçütü veya maksimum sapma ya da varyans gibi türetilmiş bir istatistik olabilir. Tüm durumlarda \emph{$M$ değerinin daha küçük olması daha iyi performansı ifade eder}.

\paragraph*{Yörünge eşleştirme ve hizalama politikası.}
Gerçek dünya yörüngeleri eşzamanlı olmayan örneklemeyle elde edilebilir ve farklı koordinat çerçevelerinde ifade edilebilir. Bu nedenle her karşılaştırma; (i) kestirim ve referans yörüngeleri arasında poz eşleştirmesi (association) ve (ii) sabit bir hizalama politikası uygulandıktan sonra yapılır. Metriğin hesaplanmasından önce kestirim yörüngesine uygulanan hizalama operatörü $\mathcal{A}(\cdot)$ ile gösterilsin; burada:
\begin{itemize}
    \item $\mathcal{A}=\mathrm{raw}$: hizalama yok,
    \item $\mathcal{A}=\mathrm{SE(3)}$: rijit hizalama (dönme + öteleme),
    \item $\mathcal{A}=\mathrm{Sim(3)}$: benzerlik hizalaması (dönme + öteleme + ölçek).
\end{itemize}
Bu analiz boyunca $M(I,H)$, \emph{sabit} bir hizalama seçimi $\mathcal{A}$ altında hesaplanır (her metrik grubu için açıkça raporlanır); böylece tüm konfigürasyonlar doğrudan karşılaştırılabilir kalır.

\paragraph*{Temel (baseline) konfigürasyon.}
Temel konfigürasyon
\begin{equation}
M(0,1),
\end{equation}
olarak tanımlanır; bu durum minimal PG iyileştirmesine ($I=0$) ve minimal zamansal koşullandırmaya ($H=1$) karşılık gelir. Bu temel durum, normalize edilmiş iyileştirmeler ve etkileşim analizleri için referans noktası olarak kullanılır.

Birincil amaç, $I$ ve $H$ parametrelerindeki değişimlerin $M$ üzerindeki etkisini nicel olarak belirlemek ve performans kazanımlarının iterasyon derinliği, zamansal koşullandırma veya bu ikisinin etkileşiminden kaynaklanıp kaynaklanmadığını ortaya koymaktır.

\subsection{İki Faktörlü Toplamsal Ayrıştırma Modeli}

Performans yüzeyi $M(I,H)$'yi yorumlamak için iki faktörlü toplamsal bir ayrıştırma benimsenir:
\begin{equation}
\boxed{
M(I,H) \;=\; \mu \;+\; \alpha(I) \;+\; \beta(H) \;+\; \gamma(I,H)
}
\label{eq:additive_model}
\end{equation}
burada:
\begin{itemize}
    \item $\mu$ tüm test edilen konfigürasyonlar üzerindeki küresel ortalama performanstır,
    \item $\alpha(I)$ PG iterasyon sayısına atfedilen \emph{ana etki}yi ifade eder,
    \item $\beta(H)$ LU geçmiş boyutuna atfedilen \emph{ana etki}yi ifade eder,
    \item $\gamma(I,H)$ toplamsal olmayan davranışı yakalayan etkileşim terimidir.
\end{itemize}

Bu model, $I$ ve $H$'nin ortalama katkılarını ayırırken artık (residual) etkileşim yapısını açıkça korur. Önemle belirtmek gerekir ki bu yaklaşım, ölçülen ızgara değerlerinin \emph{etki ayrıştırması} olarak kullanılmaktadır; olasılıksal (probabilistik) bir iddia değildir.

\subsection{Model Bileşenlerinin Kestirimi}

$\mathcal{I}$ ve $\mathcal{H}$ sırasıyla değerlendirilen iterasyon sayıları ve geçmiş boyutlarının ayrık kümelerini göstersin.

\paragraph*{Küresel Ortalama}
\begin{equation}
\mu \;=\; \frac{1}{|\mathcal{I}||\mathcal{H}|}
\sum_{i \in \mathcal{I}} \sum_{h \in \mathcal{H}} M(i,h).
\end{equation}

\paragraph*{PG İterasyonlarının Ana Etkisi}
\begin{equation}
\alpha(i) \;=\; \frac{1}{|\mathcal{H}|}
\sum_{h \in \mathcal{H}} M(i,h) \;-\; \mu.
\end{equation}

\paragraph*{LU Geçmişinin Ana Etkisi}
\begin{equation}
\beta(h) \;=\; \frac{1}{|\mathcal{I}|}
\sum_{i \in \mathcal{I}} M(i,h) \;-\; \mu.
\end{equation}

\paragraph*{Etkileşim Terimi}
\begin{equation}
\gamma(i,h) \;=\; M(i,h) \;-\; \Big(\mu + \alpha(i) + \beta(h)\Big).
\end{equation}

\paragraph*{Tanımlanabilirlik kısıtları (tanım gereği).}
Yukarıdaki tanımlarla ayrıştırma tekildir ve şu koşulları sağlar:
\begin{equation}
\sum_{i\in\mathcal{I}}\alpha(i)=0,
\qquad
\sum_{h\in\mathcal{H}}\beta(h)=0,
\qquad
\sum_{i\in\mathcal{I}}\gamma(i,h)=0,
\qquad
\sum_{h\in\mathcal{H}}\gamma(i,h)=0.
\end{equation}

$\gamma(i,h)$ teriminin büyük mutlak değerlere sahip olması, toplamsal olmayan bir davranışa işaret eder; yani bir parametrenin etkisi, diğer parametrenin seviyesine bağlıdır.

\subsection{Marjinal (Ayrık) Duyarlılık Analizi}

Test edilen ızgara üzerinde artımsal değişimleri incelemek amacıyla marjinal sonlu farklar analiz edilir.

\paragraph*{İterasyon Artışı (Sabit $H$)}
Sabit bir geçmiş boyutu $H$ için, PG iterasyon sayısının $I_1$'den $I_2$'ye arttırılmasının artımsal etkisi
\begin{equation}
\Delta_I M(I_1\!\to\! I_2 \,;\, H) \;=\; M(I_2,H) \;-\; M(I_1,H)
\end{equation}
olarak tanımlanır. $I$ arttıkça $\Delta_I M$ büyüklüğünün azalması, iyileştirme sürecinde yakınsama doygunluğuna (saturation) işaret eder.

\paragraph*{Geçmiş Artışı (Sabit $I$)}
Benzer şekilde, sabit iterasyon sayısı $I$ için LU geçmiş boyutunun $H_1$'den $H_2$'ye arttırılmasının artımsal etkisi
\begin{equation}
\Delta_H M(H_1\!\to\! H_2 \,;\, I) \;=\; M(I,H_2) \;-\; M(I,H_1)
\end{equation}
şeklindedir. Bu analiz, LU geçmişinin öncelikle bir optimizasyon itici gücü mü yoksa daha çok bir kararlılık (stabilizasyon) mekanizması mı olduğunu ortaya koyar.

\subsection{Normalize Edilmiş İyileştirme Oranları}

İyileştirmeleri ölçekten bağımsız bir biçimde ifade etmek için, temel konfigürasyona göre normalize edilmiş kazanımlar tanımlanır:
\begin{equation}
\Delta M_{\text{total}}(I,H) \;=\; M(0,1) \;-\; M(I,H).
\end{equation}
$\Delta M_{\text{total}}(I,H)>0$ olduğunda $(I,H)$ konfigürasyonu temel duruma göre iyileşme sağlamaktadır.

İterasyon derinliğine atfedilen ortalama normalize iyileştirme
\begin{equation}
C_I(i) \;=\; \frac{1}{|\mathcal{H}|}
\sum_{h \in \mathcal{H}}
\frac{M(0,1) - M(i,h)}{M(0,1)}
\end{equation}
ile özetlenir; LU geçmiş boyutuna karşılık gelen özet ise
\begin{equation}
C_H(h) \;=\; \frac{1}{|\mathcal{I}|}
\sum_{i \in \mathcal{I}}
\frac{M(0,1) - M(i,h)}{M(0,1)}
\end{equation}
olarak tanımlanır. Bu nicelikler, $M$'nin mutlak ölçeği metrikler arasında farklılık gösterse bile $I$ ve $H$'nin göreli öneminin doğrudan karşılaştırılmasını mümkün kılar.

\subsection{Etkileşim İçin Süperpozisyon Testi}

$I$ ve $H$ etkilerinin toplamsal olarak birleşip birleşmediğini değerlendirmek amacıyla, dört kanonik konfigürasyon üzerinden bir süperpozisyon hipotezi test edilir:
\begin{itemize}
    \item temel: $(0,1)$,
    \item yalnızca iterasyon: $(I,1)$,
    \item yalnızca geçmiş: $(0,H)$,
    \item birleşik: $(I,H)$.
\end{itemize}

Toplamsal davranış altında beklenen performans
\begin{equation}
M_{\text{exp}}(I,H) \;=\;
M(0,1)
+ \Big[M(I,1) - M(0,1)\Big]
+ \Big[M(0,H) - M(0,1)\Big]
\end{equation}
şeklindedir.

Etkileşim sapması
\begin{equation}
\boxed{
\Gamma(I,H) \;=\; M(I,H) \;-\; M_{\text{exp}}(I,H)
}
\end{equation}
olarak tanımlanır; burada:
\begin{itemize}
    \item $\Gamma(I,H)\approx 0$ toplamsal davranışı,
    \item $\Gamma(I,H)<0$ sinerjik etkileşimi (birleşik etkinin toplamsal beklentiden daha iyi olmasını),
    \item $\Gamma(I,H)>0$ azalan getiriyi (birleşik etkinin toplamsal beklentiden daha kötü olmasını)
\end{itemize}
ifade eder.

\subsection{Metrik Bağımlı Yorum}

Farklı metriklerin $I$ ve $H$'ye karşı farklı duyarlılık profilleri sergilemesi beklenir. APE gibi küresel ölçüler birikimli sürüklenmeyi (drift) yansıtır ve iterasyon derinliği $\alpha(I)$'ye güçlü tepki verebilir; buna karşın yerel tutarlılık ölçüleri (ör. kısa menzilli RPE veya varyans temelli istatistikler) zamansal koşullandırma $\beta(H)$'ye daha duyarlı olabilir. Yönelim odaklı hatalar (ör. yaw) ise çoğu durumda aşağı akıştaki SLAM arka uç kısıtları ve sahne gözlenebilirliği tarafından baskın biçimde belirlenebilir; bu nedenle ön-işleme parametreleriyle doğrusal biçimde ölçeklenmeyebilir.

Bu çerçeve, bir sonraki bölümde raporlanan deneysel ızgaraların yorumlanması için ilkeli bir temel sağlar; böylece PG tabanlı füzyon çıktıları (PG$\to$GT) ile stereo-only temel çıktı (ZED/RTAB-Map$\to$GT) tutarlı değerlendirme koşulları altında sistematik olarak karşılaştırılabilir.

%\section{Mathematical Framework for Experimental Analysis}
\label{sec:math_analysis_framework}

This study evaluates the proposed Papoulis--Gerchberg (PG) based LiDAR--stereo fusion pipeline under a structured sweep of algorithmic configurations. Beyond selecting a single ``best'' parameter set, the aim is to characterize how the method’s internal parameters contribute \emph{independently} and \emph{jointly} to odometric accuracy when trajectories are compared against an RTK-based GNSS reference (ground truth).

\subsection{Problem Definition and Evaluation Setting}

Let the proposed method be governed by two independent control parameters:

\begin{itemize}
    \item $I \in \mathbb{N}$: number of PG refinement iterations,
    \item $H \in \mathbb{N}$: LiDAR upsampling (LU) history size used for temporal conditioning.
\end{itemize}

For each configuration $(I,H)$, an estimated trajectory is produced and evaluated against the RTK reference trajectory using \texttt{evo}. Denote by
\begin{equation}
M(I,H)
\end{equation}
a scalar performance metric computed from this comparison. Depending on the evaluation context, $M$ may represent the APE (e.g., RMSE), the RPE at a specified spatial/temporal scale, a yaw-related error measure, or a derived statistic such as maximum deviation or variance. In all cases, \emph{lower values of $M$ indicate improved performance}.

\paragraph{Trajectory association and alignment policy.}
Because real-world trajectories are sampled asynchronously and may be expressed in different coordinate frames, each comparison is performed after (i) pose association between estimate and reference, and (ii) a fixed alignment policy. Let $\mathcal{A}(\cdot)$ denote the alignment operator applied to the estimated trajectory prior to metric computation, where:
\begin{itemize}
    \item $\mathcal{A}=\mathrm{raw}$: no alignment,
    \item $\mathcal{A}=\mathrm{SE(3)}$: rigid alignment (rotation + translation),
    \item $\mathcal{A}=\mathrm{Sim(3)}$: similarity alignment (rotation + translation + scale).
\end{itemize}
Throughout this analysis, $M(I,H)$ is computed under a \emph{fixed} alignment choice $\mathcal{A}$ (reported explicitly with each metric group), ensuring that all configurations remain directly comparable.

\paragraph{Baseline configuration.}
The baseline configuration is defined as
\begin{equation}
M(0,1),
\end{equation}
corresponding to minimal PG refinement ($I=0$) and minimal temporal conditioning ($H=1$). This baseline serves as a reference point for normalized improvements and interaction analysis.

The primary objective is to quantify how variations in $I$ and $H$ influence $M$, and to identify whether performance gains arise from iteration depth, temporal conditioning, or their interaction.

\subsection{Two-Factor Additive Decomposition Model}

To interpret the performance surface $M(I,H)$, we adopt a two-factor additive decomposition:
\begin{equation}
\boxed{
M(I,H) \;=\; \mu \;+\; \alpha(I) \;+\; \beta(H) \;+\; \gamma(I,H)
}
\label{eq:additive_model}
\end{equation}
where:
\begin{itemize}
    \item $\mu$ is the global mean performance over all tested configurations,
    \item $\alpha(I)$ is the \emph{main effect} attributable to PG iteration count,
    \item $\beta(H)$ is the \emph{main effect} attributable to LU history size,
    \item $\gamma(I,H)$ is the interaction term capturing non-additive behavior.
\end{itemize}

This model separates the average contributions of $I$ and $H$ while explicitly retaining the residual interaction structure. Importantly, this is used as an \emph{effect decomposition} of the measured grid, not as a probabilistic claim.

\subsection{Estimation of Model Components}

Let $\mathcal{I}$ and $\mathcal{H}$ denote the discrete sets of evaluated iteration counts and history sizes, respectively.

\paragraph{Global Mean}
\begin{equation}
\mu \;=\; \frac{1}{|\mathcal{I}||\mathcal{H}|}
\sum_{i \in \mathcal{I}} \sum_{h \in \mathcal{H}} M(i,h).
\end{equation}

\paragraph{Main Effect of PG Iterations}
\begin{equation}
\alpha(i) \;=\; \frac{1}{|\mathcal{H}|}
\sum_{h \in \mathcal{H}} M(i,h) \;-\; \mu.
\end{equation}

\paragraph{Main Effect of LU History}
\begin{equation}
\beta(h) \;=\; \frac{1}{|\mathcal{I}|}
\sum_{i \in \mathcal{I}} M(i,h) \;-\; \mu.
\end{equation}

\paragraph{Interaction Term}
\begin{equation}
\gamma(i,h) \;=\; M(i,h) \;-\; \Big(\mu + \alpha(i) + \beta(h)\Big).
\end{equation}

\paragraph{Identifiability constraints (by construction).}
With the above definitions, the decomposition is uniquely determined and satisfies:
\begin{equation}
\sum_{i\in\mathcal{I}}\alpha(i)=0,
\qquad
\sum_{h\in\mathcal{H}}\beta(h)=0,
\qquad
\sum_{i\in\mathcal{I}}\gamma(i,h)=0,
\qquad
\sum_{h\in\mathcal{H}}\gamma(i,h)=0.
\end{equation}

Large magnitudes of $\gamma(i,h)$ indicate non-additive behavior, meaning the impact of one parameter depends on the level of the other.

\subsection{Marginal (Discrete) Sensitivity Analysis}

To examine incremental changes across the tested grid, marginal finite differences are analyzed.

\paragraph{Iteration Increment (Fixed $H$)}
For a fixed history size $H$, the incremental effect of increasing PG iterations from $I_1$ to $I_2$ is
\begin{equation}
\Delta_I M(I_1\!\to\! I_2 \,;\, H) \;=\; M(I_2,H) \;-\; M(I_1,H).
\end{equation}
A diminishing magnitude of $\Delta_I M$ for increasing $I$ suggests convergence saturation in the refinement process.

\paragraph{History Increment (Fixed $I$)}
Similarly, for fixed iteration count $I$, the incremental effect of increasing LU history from $H_1$ to $H_2$ is
\begin{equation}
\Delta_H M(H_1\!\to\! H_2 \,;\, I) \;=\; M(I,H_2) \;-\; M(I,H_1).
\end{equation}
This reveals whether LU history acts primarily as an optimization driver or as a stabilizing mechanism.

\subsection{Normalized Improvement Ratios}

To express improvements in a scale-independent form, normalized gains are defined relative to the baseline configuration:
\begin{equation}
\Delta M_{\text{total}}(I,H) \;=\; M(0,1) \;-\; M(I,H).
\end{equation}
When $\Delta M_{\text{total}}(I,H)>0$, the configuration $(I,H)$ improves upon the baseline.

The average normalized improvement attributable to iteration depth is summarized as
\begin{equation}
C_I(i) \;=\; \frac{1}{|\mathcal{H}|}
\sum_{h \in \mathcal{H}}
\frac{M(0,1) - M(i,h)}{M(0,1)},
\end{equation}
and the corresponding summary for LU history size is
\begin{equation}
C_H(h) \;=\; \frac{1}{|\mathcal{I}|}
\sum_{i \in \mathcal{I}}
\frac{M(0,1) - M(i,h)}{M(0,1)}.
\end{equation}
These quantities enable direct comparison of the relative importance of $I$ and $H$ even when the absolute scale of $M$ differs across metrics.

\subsection{Superposition Test for Interaction}

To evaluate whether the effects of $I$ and $H$ combine additively, a superposition hypothesis is tested using four canonical configurations:
\begin{itemize}
    \item baseline: $(0,1)$,
    \item iteration-only: $(I,1)$,
    \item history-only: $(0,H)$,
    \item combined: $(I,H)$.
\end{itemize}

Under additive behavior, the expected performance is
\begin{equation}
M_{\text{exp}}(I,H) \;=\;
M(0,1)
+ \Big[M(I,1) - M(0,1)\Big]
+ \Big[M(0,H) - M(0,1)\Big].
\end{equation}

The interaction deviation is defined as
\begin{equation}
\boxed{
\Gamma(I,H) \;=\; M(I,H) \;-\; M_{\text{exp}}(I,H)
}
\end{equation}
where:
\begin{itemize}
    \item $\Gamma(I,H)\approx 0$ indicates additive behavior,
    \item $\Gamma(I,H)<0$ indicates synergistic interaction (combined effect better than additive expectation),
    \item $\Gamma(I,H)>0$ indicates diminishing returns (combined effect worse than additive expectation).
\end{itemize}

\subsection{Metric-Dependent Interpretation}

Different metrics are expected to exhibit different sensitivity profiles with respect to $I$ and $H$. Global measures such as APE often reflect accumulated drift and may respond strongly to refinement depth $\alpha(I)$, whereas local consistency measures (e.g., short-range RPE or variance-based statistics) may respond more strongly to temporal conditioning $\beta(H)$. Orientation-specific errors (e.g., yaw) can also be dominated by downstream SLAM back-end constraints and scene observability, and therefore may not scale linearly with preprocessing parameters.

This framework provides a principled basis for interpreting the experimental grids reported in the next section, enabling systematic comparison of PG-based fusion (PG$\to$GT) and stereo-only baselines (ZED/RTAB-Map$\to$GT) under consistent evaluation conditions.


\subsection{Trajektori Hizalama ve Kabsch Algoritması}
\label{sec:kabsch_alignment}

Bu çalışmada, farklı sensör kaynaklarından elde edilen trajektori tahminlerinin karşılaştırılabilmesi amacıyla, ölçüm sonuçları öncelikle ortak bir referans çerçevesine hizalanmıştır. Bu hizalama işlemi, \textit{evo}\cite{grupp2017evo} değerlendirme aracı tarafından kullanılan ve literatürde yaygın olarak kabul gören \textbf{Kabsch algoritması}\cite{Kabsch:a12999} temel alınarak gerçekleştirilmiştir.

Kabsch algoritması, iki eşleşmiş nokta kümesi arasındaki \textit{ortalama kare hatayı} (Root Mean Squared Deviation -- RMSD) minimize eden en uygun rijit dönüşümü (rotasyon ve öteleme) hesaplamayı amaçlar. Bu bağlamda, referans trajektori
$P = \{\mathbf{p}_i\}_{i=1}^N$
ve tahmin edilen trajektori
$Q = \{\mathbf{q}_i\}_{i=1}^N$
olmak üzere, her iki kümede yer alan noktaların zamansal olarak eşleştiği varsayılır.

Algoritmanın ilk adımında, her iki nokta kümesinin ağırlık merkezleri hesaplanarak merkezlenmiş noktalar elde edilir:
\[
\mathbf{p}'_i = \mathbf{p}_i - \bar{\mathbf{p}}, \quad
\mathbf{q}'_i = \mathbf{q}_i - \bar{\mathbf{q}}.
\]

Ardından, merkezlenmiş noktalar kullanılarak kovaryans matrisi tanımlanır:
\[
\mathbf{H} = \sum_{i=1}^{N} \mathbf{p}'_i \mathbf{q}'_i{}^{\top}.
\]

Bu matrisin tekil değer ayrışımı (Singular Value Decomposition -- SVD) şu şekilde ifade edilir:
\[
\mathbf{H} = \mathbf{U}\,\boldsymbol{\Sigma}\,\mathbf{V}^{\top}.
\]

Optimal rotasyon matrisi, yansıma durumlarını önleyecek biçimde
\[
\mathbf{R} =
\mathbf{V}
\begin{pmatrix}
1 & 0 & 0 \\
0 & 1 & 0 \\
0 & 0 & \det(\mathbf{V}\mathbf{U}^{\top})
\end{pmatrix}
\mathbf{U}^{\top}
\]
olarak hesaplanır. Öteleme vektörü ise
\[
\mathbf{t} = \bar{\mathbf{p}} - \mathbf{R}\,\bar{\mathbf{q}}
\]
şeklinde elde edilir.

Bu dönüşüm sayesinde, tahmin edilen trajektori referans trajektoriye en iyi şekilde hizalanır ve iki yol arasındaki farklar, global konum ve yönelim farklarından arındırılmış olarak değerlendirilir. Bu çalışmada sunulan hizalanmış trajektori hata metrikleri (örneğin Absolute Pose Error -- APE ve Relative Pose Error -- RPE), hizalanmış trajektoriler üzerinden \textit{evo} aracı kullanılarak hesaplanmıştır. Böylece elde edilen sonuçlar, sistemin gerçek izleme doğruluğunu yansıtan anlamlı karşılaştırmalar sunmaktadır.

\section{CitrusFarm Veriseti}

Bu başlık altında Deney senaryolarında bahsi geçen deneyler CitrusFarm veriseti ile gerçeklenmiştir. Deney sonuçları yine deneyler başlığında bahsedildiği şeklinde sergilenmiştir.

Deney sonuçlarındaki bir yöne doğru yünelimin esas sebebi tekerlek odometrisi versindeki biased durumdur. Bunun grafiği eklenmeli. Sürekli bir sola dönüş gösterilmiş. Daha sonra da yaw düzeltmesi ele alınabilir.

\subsection{Deneysel Sonuçlar ve Matematiksel Modelin Doğrulanması}
\label{sec:deneysel_sonuclar}

Bu bölümde, Bölüm~\ref{sec:matematiksel_analiz_cercevesi}’de tanımlanan iki faktörlü toplamsal ayrıştırma modeli kullanılarak elde edilen deneysel sonuçlar sunulmakta ve önerilen Papoulis--Gerchberg (PG) tabanlı LiDAR--stereo füzyon yönteminin davranışı nicel olarak değerlendirilmektedir. Analizler, PG iterasyon sayısı ($I$) ve LU geçmiş boyutu ($H$) parametrelerinin performans metrikleri üzerindeki bireysel ve ortak etkilerini ortaya koymayı amaçlamaktadır.

Tüm sonuçlar aynı veri setleri üzerinde, aynı değerlendirme kriterleri kullanılarak otomatik olarak üretilmiş olup, ilgili görsel çıktılar Ekler bölümünde sunulmaktadır.

\subsubsection{Performans Yüzeyi $M(I,H)$ ve Genel Eğilimler}

Ekler bölümünde sunulan ısı haritaları ve kesitsel grafikler incelendiğinde, performans metriği $M(I,H)$’nin hem PG iterasyon sayısına hem de LU geçmiş boyutuna bağlı olarak sistematik ve düzgün bir değişim sergilediği görülmektedir. Referans yapılandırma $(0,1)$ ile karşılaştırıldığında, orta seviye $(I,H)$ yapılandırmalarında mutlak poz hatasında (APE) yaklaşık \%25--\%40 aralığında bir azalma elde edilmiştir.

Benzer şekilde göreli poz hatası (RPE) metriklerinde, özellikle kısa ve orta ölçeklerde, hata değerlerinin \%20 civarında azaldığı gözlemlenmiştir. Bu durum, önerilen yöntemin yalnızca küresel doğruluğu değil, aynı zamanda yerel tutarlılığı da iyileştirdiğini göstermektedir.

\subsubsection{PG İterasyonlarının Ana Etkisi ($\alpha(I)$)}

PG iterasyonlarının ana etkisi deneysel veriler üzerinden açıkça gözlemlenmektedir. LU geçmiş boyutu sabit tutulduğunda, iterasyon sayısının artırılmasıyla birlikte APE değerlerinde belirgin bir düşüş meydana gelmiştir. Düşük iterasyon seviyelerinde ($I=0 \rightarrow I \approx 10$) hata azalımı yaklaşık \%15--\%20 seviyesindeyken, daha yüksek iterasyonlarda ($I \geq 30$) ek kazanımın \%5’in altına düştüğü görülmektedir.

Bu sonuç, Bölüm~\ref{sec:matematiksel_analiz_cercevesi}’de tanımlanan marjinal iterasyon duyarlılığı
\[
\Delta_I M \approx M(I_2,H) - M(I_1,H)
\]
ifadesiyle uyumlu olup, PG algoritmasının belirli bir iterasyon sayısından sonra yakınsama (saturation) davranışı sergilediğini göstermektedir. Dolayısıyla deneysel veriler, $\alpha(I)$ teriminin düşük ve orta iterasyon seviyelerinde baskın olduğunu, ancak yüksek iterasyonlarda azalan getiri etkisinin devreye girdiğini doğrulamaktadır.

\subsubsection{LU Geçmiş Boyutunun Ana Etkisi ($\beta(H)$)}

LU geçmiş boyutunun etkisi özellikle varyans, maksimum hata ve trajektori pürüzsüzlüğü gibi metriklerde belirgin hale gelmektedir. Küçük geçmiş boyutlarından ($H=1 \rightarrow H \approx 5$) orta seviye geçmiş boyutlarına geçişte, hata varyansında yaklaşık \%20--\%30 oranında bir azalma gözlemlenmiştir.

Buna karşılık, daha büyük geçmiş boyutlarında ek iyileşmenin sınırlı kaldığı ve bazı metriklerde neredeyse plato yaptığı görülmektedir. Bu durum,
\[
\Delta_H M \approx M(I,H_2) - M(I,H_1)
\]
ifadesiyle tanımlanan marjinal geçmiş duyarlılığının azalan bir yapıya sahip olduğunu göstermektedir. Deneysel bulgular, $\beta(H)$ teriminin esas olarak zamansal kararlılığı artıran bir rol üstlendiğini ortaya koymaktadır.

\subsubsection{Etkileşim Terimi ve Süperpozisyon Varsayımı}

PG iterasyonları ve LU geçmiş boyutunun birlikte kullanıldığı yapılandırmalar için süperpozisyon varsayımı deneysel olarak test edilmiştir. Beklenen toplamsal performans ile gerçek ölçümler arasındaki fark
\[
\Gamma(I,H)
\]
incelendiğinde, çoğu $(I,H)$ kombinasyonu için bu farkın sıfıra yakın olduğu görülmektedir. Bu durum, parametrelerin etkilerinin büyük ölçüde toplamsal olduğunu göstermektedir.

Bununla birlikte, orta seviye iterasyon ve geçmiş boyutu kombinasyonlarında $\Gamma(I,H) < 0$ durumları gözlemlenmiş; bu yapılandırmalarda toplam hata azalımının beklenen değerin yaklaşık \%5--\%10 altında gerçekleştiği belirlenmiştir. Bu sonuç, PG iteratif iyileştirme ile zamansal koşullama arasında sınırlı ancak ölçülebilir bir sinerji bulunduğunu göstermektedir.

\subsubsection{Citrus Farm Veri Setinde Yönelim Bias’ının Düzeltilmesi}

Citrus Farm veri setinde kullanılan tekerlek odometrisinin, özellikle uzun sekanslarda sola doğru (counter-clockwise, CCW) yönelim bias’ı içerdiği bilinmektedir. Bu bias, SLAM arka ucunda kullanılan odometrik önbilgiye doğrudan yansımakta ve referans trajektorilere kıyasla belirgin bir yaw sapmasına yol açmaktadır.

Ekler bölümünde sunulan trajektori grafikleri incelendiğinde, referans yapılandırma $(0,1)$ için bu CCW bias’ının açıkça gözlemlendiği; trajektorinin zamanla referanstan sistematik olarak saptığı görülmektedir. Önerilen PG tabanlı füzyon yöntemi uygulandığında ise, bu yönelim hatasının belirgin biçimde bastırıldığı gözlemlenmiştir.

Nicel olarak, yaw hatasında yaklaşık \%30--\%45 aralığında bir azalma elde edilmiş; bu iyileşme, yalnızca lokal düzeltmelerle değil, trajektorinin genel yöneliminin düzeltilmesiyle sağlanmıştır. Bu bulgu, önerilen yöntemin yalnızca ölçüm gürültüsünü azaltmakla kalmadığını, aynı zamanda hatalı odometrik bias’ları da telafi edebildiğini göstermektedir.

\subsubsection{Teori ve Deneysel Bulguların Örtüşmesi}

Genel değerlendirme sonucunda, deneysel bulguların Bölüm~\ref{sec:matematiksel_analiz_cercevesi}’de tanımlanan matematiksel model ile yüksek derecede örtüştüğü görülmektedir. Ana etki terimleri $\alpha(I)$ ve $\beta(H)$, marjinal duyarlılık analizleri ve etkileşim terimleri deneysel veriler üzerinden açıkça doğrulanmıştır.

Bu sonuçlar, önerilen matematiksel çerçevenin, PG tabanlı LiDAR--stereo füzyon sisteminin gerçek veri üzerindeki davranışını açıklamada yeterli ve tutarlı bir analiz aracı sunduğunu göstermektedir.

\section{Experimental Results}
\label{sec:results}

This section presents the experimental evaluation of the proposed Papoulis--Gerchberg (PG) based LiDAR--stereo fusion pipeline under a two-parameter sweep:
the PG iteration count \(I\) and the temporal history size \(H\).
All trajectories are evaluated against RTK-GNSS odometry as ground truth using EVO, after alignment.
The stereo-only RTAB-Map output (ZED) is used as the baseline, while the PG pipeline produces a family of outputs indexed by \((I,H)\).

Because ZED does not have the parameters \((I,H)\), the comparison is formulated as a baseline-referenced improvement field over the sweep:
\[
    \Delta(I,H)=M_{\text{ZED}}-M_{\text{PG}}(I,H),
    \qquad
    R(I,H)=\frac{M_{\text{PG}}(I,H)}{M_{\text{ZED}}}.
\]
In this formulation, \(\Delta>0\) indicates PG improvement over the ZED baseline, and \(R<1\) indicates a relative reduction in error.
Non-additive behavior between \(I\) and \(H\) is quantified with the superposition deviation \(\Gamma(I,H)\), as defined in
Section~\ref{sec:math_analysis_framework}.

\paragraph*{How to read the baseline-referenced maps (\(\Delta\) and relative gain).}
Because the ZED (stereo-only) pipeline has no sweep parameters, all comparisons are expressed relative to a fixed baseline metric value \(M_{\mathrm{ZED}}\).
For each configuration \((I,H)\), we compute the PG metric \(M_{\mathrm{PG}}(I,H)\) and visualize the improvement in two complementary forms:

\begin{align}
\Delta(I,H) &= M_{\mathrm{ZED}} - M_{\mathrm{PG}}(I,H), \\
G(I,H) &= 100 \cdot \frac{M_{\mathrm{ZED}} - M_{\mathrm{PG}}(I,H)}{M_{\mathrm{ZED}}}
      = 100 \cdot \frac{\Delta(I,H)}{M_{\mathrm{ZED}}}.
\end{align}

\noindent Here, \(\Delta(I,H)\) is reported in the \emph{native units of the metric} (e.g., meters for APE/RPE, degrees for yaw),
while \(G(I,H)\) is a \emph{unitless normalized percentage} that enables direct comparison across different metrics and scales.

\begin{itemize}
    \item \textbf{\(\Delta>0\)} indicates that PG improves over the baseline (error reduction), since \(M_{\mathrm{PG}}(I,H) < M_{\mathrm{ZED}}\).
    \item \textbf{\(\Delta=0\)} indicates no change relative to baseline.
    \item \textbf{\(\Delta<0\)} indicates degradation relative to baseline (error increase), since \(M_{\mathrm{PG}}(I,H) > M_{\mathrm{ZED}}\).
\end{itemize}

\noindent The \textbf{Relative gain (\%)} map \(G(I,H)\) is especially useful when different metrics have different magnitude ranges.
For example, a 1\,m absolute improvement may be substantial for a short-scale RPE but modest for a large-scale RPE.
By normalizing with \(M_{\mathrm{ZED}}\), the gain map highlights where improvement is \emph{proportionally} most significant.

\paragraph*{Practical interpretation.}
In the heatmaps, each cell corresponds to one configuration \((I,H)\).
Broad contiguous regions of \(\Delta>0\) (or \(G>0\)) indicate that improvements are \emph{systematic and robust} across a range of settings,
rather than being tied to a single tuned parameter choice.
Conversely, isolated positive cells surrounded by negative regions would indicate fragile tuning.

\paragraph*{Notes on scale and stability.}
While \(G(I,H)\) provides convenient normalization, it may visually amplify changes when \(M_{\mathrm{ZED}}\) is very small for a given metric.
Therefore, the interpretation of gain maps should be paired with \(\Delta(I,H)\), which preserves absolute magnitude in the metric’s units.
In this study, we report both maps jointly to avoid misleading conclusions from normalization alone.

\subsection{Global Trajectory Accuracy (APE): Strong and Consistent Improvement}
\label{subsec:results_ape}

Absolute Pose Error (APE) reflects global trajectory consistency with respect to RTK-GNSS and is therefore the primary indicator of long-horizon drift.
Figures~\ref{fig:results-ape-rmse-1}, ~\ref{fig:results-ape-rmse-2}, ~\ref{fig:results-ape-mean-1}, and ~\ref{fig:results-ape-mean-2} show the APE improvement field across the full parameter sweep.
Two complementary observations follow.

First, the improvement is not confined to a single isolated configuration: the \(\Delta\) and ratio maps reveal broad regions where PG outperforms the ZED baseline,
indicating that the proposed preprocessing/refinement is systematically beneficial rather than fragile.
Second, the improvement structure is strongly organized along the temporal history axis \(H\), supporting the interpretation that temporal aggregation stabilizes the LiDAR guidance signal.

% ----------------- FIGURE: APE RMSE -----------------
\begin{figure}[H]
    \centering
    \begin{subfigure}[t]{\linewidth}
        \centering
        \includegraphics[width=\linewidth]{./Experiments/out/metrics_aligned_se3/pg_ape_se3/rmse/slices_pg_ape_se3_rmse.png}
        \caption{Slices}
    \end{subfigure}
    \vspace{0.6em}
    \begin{subfigure}[t]{\linewidth}
        \centering
        \includegraphics[width=0.60\linewidth]{./Experiments/out/metrics_aligned_se3/pg_ape_se3/rmse/surface_delta_vs_zed_ape_se3_pg_ape_se3_rmse.png}
        \caption{3D surface}
    \end{subfigure}
    \caption{APE RMSE improvement field over \((I,H)\) for PG relative to the ZED baseline (aligned SE(3)) (Slices and Surface).}
    \label{fig:results-ape-rmse-1}
\end{figure}

\begin{figure}[H]
    \centering
    \begin{subfigure}[t]{0.49\linewidth}
        \centering
        \vspace{0pt}
        \includegraphics[width=\linewidth]{./Experiments/out/metrics_aligned_se3/pg_ape_se3/rmse/heatmap_delta_vs_zed_ape_se3_pg_ape_se3_rmse.png}
        \caption{$\Delta$ vs baseline}
    \end{subfigure}
    \hfill
    \begin{subfigure}[t]{0.49\linewidth}
        \centering
        \vspace{0pt}
        \includegraphics[width=\linewidth]{./Experiments/out/metrics_aligned_se3/pg_ape_se3/rmse/heatmap_ratio_vs_zed_ape_se3_pg_ape_se3_rmse.png}
        \caption{Relative gain (\%)}
    \end{subfigure}
    \vspace{0.8em}
    \begin{subfigure}[t]{0.49\linewidth}
        \centering
        \vspace{0pt}
        \includegraphics[width=\linewidth]{./Experiments/out/metrics_aligned_se3/pg_ape_se3/rmse/heatmap_gamma_pg_ape_se3_rmse.png}
        \caption{Interaction heatmap ($\Gamma$)}
    \end{subfigure}
    \hfill
    \begin{subfigure}[t]{0.49\linewidth}
        \centering
        \vspace{2.0em}
        \begin{tabular}{rrrrr}
            \toprule
            I \textbackslash H & 1     & 2      & 5      & 10     \\
            \midrule
            0                  & 0.000 & 0.000  & 0.000  & 0.000  \\
            10                 & 0.000 & 0.563  & -2.252 & 1.260  \\
            33                 & 0.000 & 1.541  & -1.790 & 4.691  \\
            100                & 0.000 & -1.416 & -4.340 & -2.020 \\
            \bottomrule
        \end{tabular}
        \vspace{1.5em}
        \caption{Interaction matrix ($\Gamma$)}
    \end{subfigure}
    \caption{APE RMSE improvement field over \((I,H)\) for PG relative to the ZED baseline (aligned SE(3)) (Metric comparisons).}
    \emph{Positive \(\Delta\) and positive gain (\%) indicate improved performance over the ZED baseline (lower error).}
    \label{fig:results-ape-rmse-2}
\end{figure}

% ----------------- FIGURE: APE MEAN -----------------
\begin{figure}[H]
    \centering
    \begin{subfigure}[t]{\linewidth}
        \centering
        \includegraphics[width=\linewidth]{./Experiments/out/metrics_aligned_se3/pg_ape_se3/mean/slices_pg_ape_se3_mean.png}
        \caption{Slices}
    \end{subfigure}
    \vspace{0.6em}
    \begin{subfigure}[t]{\linewidth}
        \centering
        \includegraphics[width=0.60\linewidth]{./Experiments/out/metrics_aligned_se3/pg_ape_se3/mean/surface_delta_vs_zed_ape_se3_pg_ape_se3_mean.png}
        \caption{3D surface}
    \end{subfigure}
    \caption{APE Mean improvement field over \((I,H)\) for PG relative to the ZED baseline (aligned SE(3)) (Slices and Surface).}
    \label{fig:results-ape-mean-1}
\end{figure}

\begin{figure}[H]
    \begin{subfigure}[t]{0.49\linewidth}
        \centering
        \vspace{0pt}
        \includegraphics[width=\linewidth]{./Experiments/out/metrics_aligned_se3/pg_ape_se3/mean/heatmap_delta_vs_zed_ape_se3_pg_ape_se3_mean.png}
        \caption{$\Delta$ vs baseline}
    \end{subfigure}
    \hfill
    \begin{subfigure}[t]{0.49\linewidth}
        \centering
        \vspace{0pt}
        \includegraphics[width=\linewidth]{./Experiments/out/metrics_aligned_se3/pg_ape_se3/mean/heatmap_ratio_vs_zed_ape_se3_pg_ape_se3_mean.png}
        \caption{Relative gain (\%)}
    \end{subfigure}
    \vspace{0.8em}
    \begin{subfigure}[t]{0.49\linewidth}
        \centering
        \vspace{0pt}
        \includegraphics[width=\linewidth]{./Experiments/out/metrics_aligned_se3/pg_ape_se3/mean/heatmap_gamma_pg_ape_se3_mean.png}
        \caption{Interaction heatmap ($\Gamma$)}
    \end{subfigure}
    \hfill
    \begin{subfigure}[t]{0.49\linewidth}
        \centering
        \vspace{2.0em}
        \begin{tabular}{rrrrr}
            \toprule
            I \textbackslash H & 1     & 2      & 5      & 10     \\
            \midrule
            0                  & 0.000 & 0.000  & 0.000  & 0.000  \\
            10                 & 0.000 & 0.165  & -2.443 & 0.974  \\
            33                 & 0.000 & 1.177  & -1.971 & 3.958  \\
            100                & 0.000 & -1.591 & -4.340 & -1.970 \\
            \bottomrule
        \end{tabular}
        \vspace{1.5em}
        \caption{Interaction matrix ($\Gamma$)}
    \end{subfigure}
    \caption{APE Mean improvement field over \((I,H)\) for PG relative to the ZED baseline (aligned SE(3)) (Metric comparisons).}
    \emph{Positive \(\Delta\) and positive gain (\%) indicate improved performance over the ZED baseline (lower error).}
    \label{fig:results-ape-mean-2}
\end{figure}

\subsection{Local Consistency (RPE 1m): Strong Improvement at Short Scale}
\label{subsec:results_rpe_1m}

Relative Pose Error (RPE) at 1\,m emphasizes short-window motion consistency.
Figures~\ref{fig:results-rpe-1m-rmse-1} and ~\ref{fig:results-rpe-1m-rmse-2} shows that PG yields consistent improvements across a wide region of the sweep,
which supports the claim that denser and more stable depth guidance improves local motion estimation robustness in the SLAM backend.

% ----------------- FIGURE: RPE 1m RMSE -----------------
\begin{figure}[H]
    \centering
    \begin{subfigure}[t]{\linewidth}
        \centering
        \includegraphics[width=\linewidth]{./Experiments/out/metrics_aligned_se3/pg_rpe_1m_se3/rmse/slices_pg_rpe_1m_se3_rmse.png}
        \caption{Slices}
    \end{subfigure}
    \vspace{0.6em}
    \begin{subfigure}[t]{\linewidth}
        \centering
        \includegraphics[width=0.80\linewidth]{./Experiments/out/metrics_aligned_se3/pg_rpe_1m_se3/rmse/surface_delta_vs_zed_rpe_1m_se3_pg_rpe_1m_se3_rmse.png}
        \caption{3D surface}
    \end{subfigure}
    \caption{RPE 1m RMSE improvement field over \((I,H)\) for PG relative to the ZED baseline (aligned SE(3)) (Slices and Surfaces).}
    \label{fig:results-rpe-1m-rmse-1}
\end{figure}

\begin{figure}[H]
    \begin{subfigure}[t]{0.49\linewidth}
        \centering
        \vspace{0pt}
        \includegraphics[width=\linewidth]{./Experiments/out/metrics_aligned_se3/pg_rpe_1m_se3/rmse/heatmap_delta_vs_zed_rpe_1m_se3_pg_rpe_1m_se3_rmse.png}
        \caption{$\Delta$ vs baseline}
    \end{subfigure}
    \hfill
    \begin{subfigure}[t]{0.49\linewidth}
        \centering
        \vspace{0pt}
        \includegraphics[width=\linewidth]{./Experiments/out/metrics_aligned_se3/pg_rpe_1m_se3/rmse/heatmap_ratio_vs_zed_rpe_1m_se3_pg_rpe_1m_se3_rmse.png}
        \caption{Relative gain (\%)}
    \end{subfigure}
    \vspace{0.8em}
    \begin{subfigure}[t]{0.49\linewidth}
        \centering
        \vspace{0pt}
        \includegraphics[width=\linewidth]{./Experiments/out/metrics_aligned_se3/pg_rpe_1m_se3/rmse/heatmap_gamma_pg_rpe_1m_se3_rmse.png}
        \caption{Interaction heatmap ($\Gamma$)}
    \end{subfigure}
    \hfill
    \begin{subfigure}[t]{0.49\linewidth}
        \centering
        \vspace{2.0em}
        \begin{tabular}{rrrrr}
            \toprule
            I \textbackslash H & 1     & 2     & 5     & 10    \\
            \midrule
            0                  & 0.000 & 0.000 & 0.000 & 0.000 \\
            10                 & 0.000 & 0.004 & 0.014 & 0.008 \\
            33                 & 0.000 & 0.007 & 0.013 & 0.010 \\
            100                & 0.000 & 0.004 & 0.009 & 0.008 \\
            \bottomrule
        \end{tabular}
        \vspace{1.5em}
        \caption{Interaction matrix ($\Gamma$)}
    \end{subfigure}
    \caption{RPE 1m RMSE improvement field over \((I,H)\) for PG relative to the ZED baseline (aligned SE(3)) (Metric comparisons) (Metric comparisons).}
    \emph{Positive \(\Delta\) and positive gain (\%) indicate improved performance over the ZED baseline (lower error).}
    \label{fig:results-rpe-1m-rmse-2}
\end{figure}

\subsection{Mid-Scale Consistency (RPE 100m): Improvement Under SLAM Graph Constraints}
\label{subsec:results_rpe_100m}

RPE at 100\,m captures mid-scale drift behavior and is more reflective of accumulated backend constraints than short-scale RPE.
Figures~\ref{fig:results-rpe-100m-rmse-1} and ~\ref{fig:results-rpe-100m-rmse-2} demonstrates that improvements remain achievable at this scale,
suggesting that PG contributes not only to local tracking quality but also to more stable mid-horizon trajectory consistency.

% ----------------- FIGURE: RPE 100m RMSE -----------------
\begin{figure}[H]
    \centering
    \begin{subfigure}[t]{\linewidth}
        \centering
        \includegraphics[width=\linewidth]{./Experiments/out/metrics_aligned_se3/pg_rpe_100m_se3/rmse/slices_pg_rpe_100m_se3_rmse.png}
        \caption{Slices}
    \end{subfigure}
    \vspace{0.6em}
    \begin{subfigure}[t]{\linewidth}
        \centering
        \includegraphics[width=0.60\linewidth]{./Experiments/out/metrics_aligned_se3/pg_rpe_100m_se3/rmse/surface_delta_vs_zed_rpe_100m_se3_pg_rpe_100m_se3_rmse.png}
        \caption{3D surface}
    \end{subfigure}
    \caption{RPE 100m RMSE improvement field over \((I,H)\) for PG relative to the ZED baseline (aligned SE(3)) (Slices and Surfaces).}
    \label{fig:results-rpe-100m-rmse-1}
\end{figure}
\begin{figure}[H]
    \centering
    \begin{subfigure}[t]{0.49\linewidth}
        \centering
        \vspace{0pt}
        \includegraphics[width=\linewidth]{./Experiments/out/metrics_aligned_se3/pg_rpe_100m_se3/rmse/heatmap_delta_vs_zed_rpe_100m_se3_pg_rpe_100m_se3_rmse.png}
        \caption{$\Delta$ vs baseline}
    \end{subfigure}
    \hfill
    \begin{subfigure}[t]{0.49\linewidth}
        \centering
        \vspace{0pt}
        \includegraphics[width=\linewidth]{./Experiments/out/metrics_aligned_se3/pg_rpe_100m_se3/rmse/heatmap_ratio_vs_zed_rpe_100m_se3_pg_rpe_100m_se3_rmse.png}
        \caption{Relative gain (\%)}
    \end{subfigure}
    \vspace{0.8em}
    \begin{subfigure}[t]{0.49\linewidth}
        \centering
        \vspace{0pt}
        \includegraphics[width=\linewidth]{./Experiments/out/metrics_aligned_se3/pg_rpe_100m_se3/rmse/heatmap_gamma_pg_rpe_100m_se3_rmse.png}
        \caption{Interaction heatmap ($\Gamma$)}
    \end{subfigure}
    \hfill
    \begin{subfigure}[t]{0.49\linewidth}
        \centering
        \vspace{2.0em}
        \begin{tabular}{rrrrr}
            \toprule
            I \textbackslash H & 1     & 2     & 5     & 10    \\
            \midrule
            0                  & 0.000 & 0.000 & 0.000 & 0.000 \\
            10                 & 0.000 & 0.358 & 0.611 & 0.134 \\
            33                 & 0.000 & 0.073 & 0.011 & 0.024 \\
            100                & 0.000 & 0.458 & 0.612 & 0.311 \\
            \bottomrule
        \end{tabular}
        \vspace{1.5em}
        \caption{Interaction matrix ($\Gamma$)}
    \end{subfigure}
    \caption{RPE 100m RMSE improvement field over \((I,H)\) for PG relative to the ZED baseline (aligned SE(3)) (Metric comparisons).}
    \emph{Positive \(\Delta\) and positive gain (\%) indicate improved performance over the ZED baseline (lower error).}
    \label{fig:results-rpe-100m-rmse-2}
\end{figure}

\subsection{Interaction Analysis: Superposition Deviation \(\Gamma(I,H)\)}
\label{subsec:results_interaction}

The \(\Gamma\) heatmaps and tables embedded in Figures~\ref{fig:results-ape-rmse-2}, ~\ref{fig:results-rpe-1m-rmse-2}, and ~\ref{fig:results-rpe-100m-rmse-2}
allow testing whether the effects of \(I\) and \(H\) combine additively.
Overall, \(\Gamma(I,H)\) remains moderate across large regions, implying that iteration and history effects often behave near-additively.
Nevertheless, localized departures from \(\Gamma \approx 0\) indicate metric-dependent interactions:
negative \(\Gamma\) corresponds to synergistic improvement beyond additive expectation, whereas positive \(\Gamma\) corresponds to diminishing returns.

\subsection{Summary of Findings}
\label{subsec:results_summary}

The selected headline metrics (APE and short/mid-scale RPE) demonstrate that the PG-based fusion pipeline can consistently outperform the stereo-only baseline.
Improvements are structured over the sweep space, indicating systematic benefit rather than isolated tuning effects.
The observed patterns also support the interpretation that temporal history \(H\) is a primary stabilizing mechanism,
while iteration count \(I\) provides secondary refinements with interaction behavior that depends on the metric scale.

\paragraph*{Inference from the fixed-iteration sweep (\(I=100\)).}
With the PG iteration count fixed at \(I=100\), Fig.~\ref{fig:pg_iter_100_xy_raw} isolates the effect of sweeping the temporal history size \(H\) on the resulting trajectory estimate. As \(H\) increases, the fusion stage benefits from a denser cumulative LiDAR point support, which strengthens the geometric prior used during the PG-guided refinement. This increased geometric conditioning is reflected in the trajectory overlays: configurations with larger \(H\) tend to track the RTK-GNSS reference more consistently and exhibit reduced drift compared to lower-history settings. A particularly indicative qualitative cue is the apparent scale contraction observed in the stereo-only (ZED) SLAM trajectory, which appears shorter relative to the high-history PG setting (e.g., \(H=10\)); this suggests that the LiDAR-supported refinement mitigates scale inconsistency and stabilizes the accumulated motion estimate when sufficient temporal context is available.

\begin{figure}[H]
    \centering
    \begin{subfigure}[t]{0.49\linewidth}
        \centering
        \includegraphics[width=\textwidth]{./Experiments/traj_plots/by_iter/iter_100_histories_vs_gt_xy_raw_rpy.png}
        \caption{PG trajectories for iteration 100 over all histories (mode: xy, alignment: raw, view: rpy).}
        \label{fig:pg_iter_100_xy_raw_rpy}
    \end{subfigure}
    \begin{subfigure}[t]{0.49\linewidth}
        \centering
        \includegraphics[width=\textwidth]{./Experiments/traj_plots/by_iter/iter_100_histories_vs_gt_xy_raw_xyz.png}
        \caption{PG trajectories for iteration 100 over all histories (mode: xy, alignment: raw, view: xyz).}
        \label{fig:pg_iter_100_xy_raw_xyz}
    \end{subfigure}
    \begin{subfigure}[t]{0.49\linewidth}
        \centering
        \includegraphics[width=\textwidth]{./Experiments/traj_plots/by_iter/iter_100_histories_vs_gt_xy_raw_trajectories.png}
        \caption{PG trajectories for iteration 100 over all histories (mode: xy, alignment: raw, view: trajectories).}
        \label{fig:pg_iter_100_xy_raw_trajectories}
    \end{subfigure}

    \caption{PG iteration 100 trajectory comparison over all histories (mode: xy, alignment: raw).}
    \emph{A consistent counterclockwise (CCW) drift observed in the SLAM odometry is attributed to a CCW-biased wheel odometry input, which is used both by the RTAB-Map backend and during LiDAR upsampling.}
    \label{fig:pg_iter_100_xy_raw}
\end{figure}

\paragraph*{Interpretation of \(z\)-axis and orientation views (why subfigures~\ref{fig:pg_iter_100_xy_raw_rpy} and~\ref{fig:pg_iter_100_xy_raw_xyz} remain qualitative).}
Subfigures~\ref{fig:pg_iter_100_xy_raw_rpy} and~\ref{fig:pg_iter_100_xy_raw_xyz} additionally visualize the trajectory in 3D and in an RPY-oriented view; however, these plots should be interpreted as \emph{qualitative diagnostics} rather than strict accuracy comparisons to ground truth. The RTK-GNSS reference provides reliable global position, but does not directly provide full 6-DoF attitude ground truth at comparable fidelity (in particular roll and pitch, and often yaw depending on the GNSS setup). Consequently, deviations in \(z\) and roll/pitch cannot be scored against the same reference in the same way as planar position. Nevertheless, these views are intentionally retained because they help reveal configuration-dependent failure modes (e.g., vertical oscillations, orientation instability, or inconsistent 3D motion) that may correlate with translational drift. For quantitative evaluation, the study therefore relies on position-based trajectory error metrics (APE/RPE) computed from translational components, where the RTK-GNSS reference is directly comparable.



\section{SensorSuiteV2 Veriseti}

Bu başlık altında Deney senaryolarında bahsi geçen deneyler SensorSuite veriseti ile gerçeklenmiştir. Deney sonuçları yine deneyler başlığında bahsedildiği şeklinde sergilenmiştir.

\section{Paralel İşleme Performansı}

GPU ile sinyal işlemeye dayalı yöntemler geliştirildiği için çeşitli GPU modelleri ile deneyler tekrarlandı. PG algoritması başarımının iterasyon sayısı ile doğru orantılı olduğu ve işlem gücü dar boğazının asıl PG iterasyon sayılarından meydana geldiği göz önünde bulundurularak bağımsız değişkenin GPU çeşiti ve bağımlı değişkenin 100 mili saniyedeki maksimum PG iterasyon sayısı oldu bir deney yapıldı. Deney sonucunda NVidia GPU'lardaki Cuda çekirdek sayıları ile maksimum PG iterasyonu sayısı arasında Şekil~\ref{fig:cuda_core_vs_pg_iter}'de verilen sayısal sonuçlara ulaşıldı.

\begin{figure}[H]
    \begin{subfigure}[t]{0.48\textwidth}
        \centering
        \includegraphics[height=7cm]{./Figures/pg_filter_selection/dama_brick-wall_0.64_2000.png}
        \caption{Karşılaştırma eğrisi}
        \label{fig:cuda_core_vs_pg_iter_chart}
    \end{subfigure}

    \vspace{0.8em}

    % ---------- Row 3 ----------
    \begin{subfigure}[t]{0.48\textwidth}
        \centering

        \renewcommand{\arraystretch}{1.15}
        \begin{tabular}{lcc}
            \toprule
            \textbf{GPU}           & \textbf{Cuda Çekirdeği Sayısı} & \textbf{Max PG İterasyon Sayısı} \\
            \midrule
            Nvidia Jatson Orin NX? & ?                              & 10?                              \\
            RTX2060                & ?                              & 33                               \\
            T4000?                 & ?                              & 50?                              \\
            A4000?                 & ?                              & 100                              \\
            \bottomrule
        \end{tabular}

        \caption{Nümerik karşılaştırma}
        \label{fig:cuda_core_vs_pg_iter_table}
    \end{subfigure}

    \caption{NVidia GPU ve PG iterasyon sayısı karşılaştırması}
    \label{fig:cuda_core_vs_pg_iter}
\end{figure}

Algoritmanın emekleme ve ilk adım safhaları değerlendirilmdiğinde bu değerler oldukça yüksek başarıyı ifade etmektedir. 3B point cloud verisinde yaptığımız ilk PG iterasyonu (tek iterasyon) CPU üzerinde paralel işleme olmaksızın 50 saniyede (50 s) tamamlanırken gelinen son noktada bir iterasyon tüm GPU çetileri göz çnüne alındığında milisaniyenin onda biri (0.1 ms) mertebelerine düşürülmüştür.

Önerdiğimiz yöntemin gerçek zamanlı çalıştırılabilmesi için ekstra efor harcanmış ve başarılı olunmuştur.
