\chapter{DENEY} \label{experimentation}

\section{Yöntem}

Dinamik Papoulish-Gerchberg algoritmasının SLAM'e katkısını ölçülebilir kılmak için bu tez kapsamında oluşturulan SensorSuiteV2 ve benzer veri tiplerini içeren açık kaynak CitrusFarm versiseti kullanılmıştır. Bu verisetlerindeki veriler RTABMap aracı kullanılarak doğrudan veya algoritmayla ön işlemeden geçirilerek işlenmiştir.

Algoritma katkısının hangi algoritma adımlarıca majör olarak sağlandığını bulabilmek için deneyler superposition prensibine uygun olarak decompose edilmiştir. Ön işleme (PG algoritması) farklı adımları bypass geçilerek geriye kalan kısımlarının iyileşmeye katkısı değerlendirilmiştir. En sonunda kümülatif katkının süperpozisyon prensipine uygun olup olmadığı değerlendirilmiştir.

Deney esnasında decompose edilen parçalar şu şekildedir:

\begin{itemize}
    \item Bütünleşik algoritmanın çıktısı derinlik imajı
    \item PG iterason sayısını N adet discrete değer ile üretilen derinlik imajı (N çeşitlendirerek iyileşme lineerlinin test edilmesi amaçlanmıştır N=[0, 33])
    \item Lidar upsampleda kullanilan geriye dönük lidar frame sayısının çeşitlendirilmesi algoritma çıktısı derinlik imajı N[1,10]
\end{itemize}

Tüm bu deney çıktılarının SLAM performanlarının karşılaştırılması için odometri çıktıları birleştirilerek elde edilen güzergahlar RTK GPS verisi ile karşılaştırılmıştır.

Kendi verisetimizde loop-closure içeren kısımları da barındırdığımız için path örtüşmesi kıyasının yanı sıra aynı pozisyondan geçerkenki drift değerleri de kıyaslanmıştır.

SLAM çıktısının bir diğeri üç boyutlu harita olduğu için aynı zamanda üç boyutlu harita performansları da kıyaslanmıştır. Burada temel ölçü olarak harita çözünürlüğü, harita alignment hatası ve bir de gözle yapılan kıyasların görsellerine tez kapsamında yer verilmiştir.Bahsedilen son metrik öznel yargılar işerebileceği gibi ektra eklendiği için bilimsel bir kanıt amacı günülmemiş, yalnızca bir bakış açısı katılması amaçlanmıştır.

Deneyde yalnızca Lidar ve IMU tabanlı SLAM algoritmaları ile karşılaştırmaya girilmemiştir. Bunun temel sebebi Lidarın seyrek veri yapısının renksiz ve düşük çözünürlüklü SLAM çıktısının stereo depth ile elde edilen SLAM çıktısıyla karşılaştırılmasının çok fazla metrik değişiminden dolayı sağlıksız olmasıdır.

\subsection{Bütünleşik algoritma parametreleri ve akışı}

Algoritma baştan uca tüm parametrelerin en iyi sonucu vereceği tahmin edilen dahiliyle deneye sokulmuştur.

Bu başlık altında test edilen şeyin ne tür bir füsyon olduğu görsellerle anlatılmalıdır.

\subsection{Değişken PG iterasyon sayısı}

Bu başlık altında test edilen şeyin ne tür bir füsyon olduğu görsellerle anlatılmalıdır.

\subsection{Değişken Lidar upsample Lidar History sayısı}

Bu başlık altında test edilen şeyin ne tür bir füsyon olduğu görsellerle anlatılmalıdır.

Alt başlıklarda verilen deney yöntemleri tek bir tablo altına toplanmış ve matris yapı ile çaprazlanarak deneyler tamamlanmıştır.
Bu deneylerin sonuçları grafikler ve tablolar ile nicel olarak ele alınmalıdır.
APE ve benzeri yöntemler ile ilgili numerik çıktılar buraya eklenmelidir.
XYZ güzergah karşılaştırmaları eklenmelidir.

\section{Deneysel Analiz İçin Matematiksel Çerçeve}
\label{sec:math_analysis_framework}

Bu çalışma, önerilen Papoulis--Gerchberg (PG) tabanlı LiDAR--stereo füzyon hattını, algoritmik konfigürasyonlar üzerinde yapılandırılmış bir tarama (sweep) kapsamında değerlendirmektedir. Amaç yalnızca tek bir ``en iyi'' parametre kümesini seçmek değil; RTK tabanlı GNSS referansı (yer gerçeği) ile karşılaştırılan yörüngelerde, yöntemin iç parametrelerinin odometrik doğruluğa \emph{bağımsız} ve \emph{birlikte} nasıl katkı sunduğunu nicel olarak karakterize etmektir.

\subsection{Problem Tanımı ve Değerlendirme Kurulumu}

Önerilen yöntem iki bağımsız kontrol parametresi tarafından belirlenmektedir:

\begin{itemize}
    \item $I \in \mathbb{N}$: PG iyileştirme iterasyon sayısı,
    \item $H \in \mathbb{N}$: zamansal koşullandırma için kullanılan LiDAR upsampling (LU) geçmiş (history) boyutu.
\end{itemize}

Her bir $(I,H)$ konfigürasyonu için bir kestirim yörüngesi üretilir ve \texttt{evo} kullanılarak RTK referans yörüngesine karşı değerlendirilir. Bu karşılaştırmadan elde edilen skaler bir performans metriği
\begin{equation}
M(I,H)
\end{equation}
ile gösterilsin. Değerlendirme bağlamına bağlı olarak $M$, APE (ör. RMSE), belirli bir uzamsal/zamansal ölçekte RPE, yaw ile ilişkili bir hata ölçütü veya maksimum sapma ya da varyans gibi türetilmiş bir istatistik olabilir. Tüm durumlarda \emph{$M$ değerinin daha küçük olması daha iyi performansı ifade eder}.

\paragraph*{Yörünge eşleştirme ve hizalama politikası.}
Gerçek dünya yörüngeleri eşzamanlı olmayan örneklemeyle elde edilebilir ve farklı koordinat çerçevelerinde ifade edilebilir. Bu nedenle her karşılaştırma; (i) kestirim ve referans yörüngeleri arasında poz eşleştirmesi (association) ve (ii) sabit bir hizalama politikası uygulandıktan sonra yapılır. Metriğin hesaplanmasından önce kestirim yörüngesine uygulanan hizalama operatörü $\mathcal{A}(\cdot)$ ile gösterilsin; burada:
\begin{itemize}
    \item $\mathcal{A}=\mathrm{raw}$: hizalama yok,
    \item $\mathcal{A}=\mathrm{SE(3)}$: rijit hizalama (dönme + öteleme),
    \item $\mathcal{A}=\mathrm{Sim(3)}$: benzerlik hizalaması (dönme + öteleme + ölçek).
\end{itemize}
Bu analiz boyunca $M(I,H)$, \emph{sabit} bir hizalama seçimi $\mathcal{A}$ altında hesaplanır (her metrik grubu için açıkça raporlanır); böylece tüm konfigürasyonlar doğrudan karşılaştırılabilir kalır.

\paragraph*{Temel (baseline) konfigürasyon.}
Temel konfigürasyon
\begin{equation}
M(0,1),
\end{equation}
olarak tanımlanır; bu durum minimal PG iyileştirmesine ($I=0$) ve minimal zamansal koşullandırmaya ($H=1$) karşılık gelir. Bu temel durum, normalize edilmiş iyileştirmeler ve etkileşim analizleri için referans noktası olarak kullanılır.

Birincil amaç, $I$ ve $H$ parametrelerindeki değişimlerin $M$ üzerindeki etkisini nicel olarak belirlemek ve performans kazanımlarının iterasyon derinliği, zamansal koşullandırma veya bu ikisinin etkileşiminden kaynaklanıp kaynaklanmadığını ortaya koymaktır.

\subsection{İki Faktörlü Toplamsal Ayrıştırma Modeli}

Performans yüzeyi $M(I,H)$'yi yorumlamak için iki faktörlü toplamsal bir ayrıştırma benimsenir:
\begin{equation}
\boxed{
M(I,H) \;=\; \mu \;+\; \alpha(I) \;+\; \beta(H) \;+\; \gamma(I,H)
}
\label{eq:additive_model}
\end{equation}
burada:
\begin{itemize}
    \item $\mu$ tüm test edilen konfigürasyonlar üzerindeki küresel ortalama performanstır,
    \item $\alpha(I)$ PG iterasyon sayısına atfedilen \emph{ana etki}yi ifade eder,
    \item $\beta(H)$ LU geçmiş boyutuna atfedilen \emph{ana etki}yi ifade eder,
    \item $\gamma(I,H)$ toplamsal olmayan davranışı yakalayan etkileşim terimidir.
\end{itemize}

Bu model, $I$ ve $H$'nin ortalama katkılarını ayırırken artık (residual) etkileşim yapısını açıkça korur. Önemle belirtmek gerekir ki bu yaklaşım, ölçülen ızgara değerlerinin \emph{etki ayrıştırması} olarak kullanılmaktadır; olasılıksal (probabilistik) bir iddia değildir.

\subsection{Model Bileşenlerinin Kestirimi}

$\mathcal{I}$ ve $\mathcal{H}$ sırasıyla değerlendirilen iterasyon sayıları ve geçmiş boyutlarının ayrık kümelerini göstersin.

\paragraph*{Küresel Ortalama}
\begin{equation}
\mu \;=\; \frac{1}{|\mathcal{I}||\mathcal{H}|}
\sum_{i \in \mathcal{I}} \sum_{h \in \mathcal{H}} M(i,h).
\end{equation}

\paragraph*{PG İterasyonlarının Ana Etkisi}
\begin{equation}
\alpha(i) \;=\; \frac{1}{|\mathcal{H}|}
\sum_{h \in \mathcal{H}} M(i,h) \;-\; \mu.
\end{equation}

\paragraph*{LU Geçmişinin Ana Etkisi}
\begin{equation}
\beta(h) \;=\; \frac{1}{|\mathcal{I}|}
\sum_{i \in \mathcal{I}} M(i,h) \;-\; \mu.
\end{equation}

\paragraph*{Etkileşim Terimi}
\begin{equation}
\gamma(i,h) \;=\; M(i,h) \;-\; \Big(\mu + \alpha(i) + \beta(h)\Big).
\end{equation}

\paragraph*{Tanımlanabilirlik kısıtları (tanım gereği).}
Yukarıdaki tanımlarla ayrıştırma tekildir ve şu koşulları sağlar:
\begin{equation}
\sum_{i\in\mathcal{I}}\alpha(i)=0,
\qquad
\sum_{h\in\mathcal{H}}\beta(h)=0,
\qquad
\sum_{i\in\mathcal{I}}\gamma(i,h)=0,
\qquad
\sum_{h\in\mathcal{H}}\gamma(i,h)=0.
\end{equation}

$\gamma(i,h)$ teriminin büyük mutlak değerlere sahip olması, toplamsal olmayan bir davranışa işaret eder; yani bir parametrenin etkisi, diğer parametrenin seviyesine bağlıdır.

\subsection{Marjinal (Ayrık) Duyarlılık Analizi}

Test edilen ızgara üzerinde artımsal değişimleri incelemek amacıyla marjinal sonlu farklar analiz edilir.

\paragraph*{İterasyon Artışı (Sabit $H$)}
Sabit bir geçmiş boyutu $H$ için, PG iterasyon sayısının $I_1$'den $I_2$'ye arttırılmasının artımsal etkisi
\begin{equation}
\Delta_I M(I_1\!\to\! I_2 \,;\, H) \;=\; M(I_2,H) \;-\; M(I_1,H)
\end{equation}
olarak tanımlanır. $I$ arttıkça $\Delta_I M$ büyüklüğünün azalması, iyileştirme sürecinde yakınsama doygunluğuna (saturation) işaret eder.

\paragraph*{Geçmiş Artışı (Sabit $I$)}
Benzer şekilde, sabit iterasyon sayısı $I$ için LU geçmiş boyutunun $H_1$'den $H_2$'ye arttırılmasının artımsal etkisi
\begin{equation}
\Delta_H M(H_1\!\to\! H_2 \,;\, I) \;=\; M(I,H_2) \;-\; M(I,H_1)
\end{equation}
şeklindedir. Bu analiz, LU geçmişinin öncelikle bir optimizasyon itici gücü mü yoksa daha çok bir kararlılık (stabilizasyon) mekanizması mı olduğunu ortaya koyar.

\subsection{Normalize Edilmiş İyileştirme Oranları}

İyileştirmeleri ölçekten bağımsız bir biçimde ifade etmek için, temel konfigürasyona göre normalize edilmiş kazanımlar tanımlanır:
\begin{equation}
\Delta M_{\text{total}}(I,H) \;=\; M(0,1) \;-\; M(I,H).
\end{equation}
$\Delta M_{\text{total}}(I,H)>0$ olduğunda $(I,H)$ konfigürasyonu temel duruma göre iyileşme sağlamaktadır.

İterasyon derinliğine atfedilen ortalama normalize iyileştirme
\begin{equation}
C_I(i) \;=\; \frac{1}{|\mathcal{H}|}
\sum_{h \in \mathcal{H}}
\frac{M(0,1) - M(i,h)}{M(0,1)}
\end{equation}
ile özetlenir; LU geçmiş boyutuna karşılık gelen özet ise
\begin{equation}
C_H(h) \;=\; \frac{1}{|\mathcal{I}|}
\sum_{i \in \mathcal{I}}
\frac{M(0,1) - M(i,h)}{M(0,1)}
\end{equation}
olarak tanımlanır. Bu nicelikler, $M$'nin mutlak ölçeği metrikler arasında farklılık gösterse bile $I$ ve $H$'nin göreli öneminin doğrudan karşılaştırılmasını mümkün kılar.

\subsection{Etkileşim İçin Süperpozisyon Testi}

$I$ ve $H$ etkilerinin toplamsal olarak birleşip birleşmediğini değerlendirmek amacıyla, dört kanonik konfigürasyon üzerinden bir süperpozisyon hipotezi test edilir:
\begin{itemize}
    \item temel: $(0,1)$,
    \item yalnızca iterasyon: $(I,1)$,
    \item yalnızca geçmiş: $(0,H)$,
    \item birleşik: $(I,H)$.
\end{itemize}

Toplamsal davranış altında beklenen performans
\begin{equation}
M_{\text{exp}}(I,H) \;=\;
M(0,1)
+ \Big[M(I,1) - M(0,1)\Big]
+ \Big[M(0,H) - M(0,1)\Big]
\end{equation}
şeklindedir.

Etkileşim sapması
\begin{equation}
\boxed{
\Gamma(I,H) \;=\; M(I,H) \;-\; M_{\text{exp}}(I,H)
}
\end{equation}
olarak tanımlanır; burada:
\begin{itemize}
    \item $\Gamma(I,H)\approx 0$ toplamsal davranışı,
    \item $\Gamma(I,H)<0$ sinerjik etkileşimi (birleşik etkinin toplamsal beklentiden daha iyi olmasını),
    \item $\Gamma(I,H)>0$ azalan getiriyi (birleşik etkinin toplamsal beklentiden daha kötü olmasını)
\end{itemize}
ifade eder.

\subsection{Metrik Bağımlı Yorum}

Farklı metriklerin $I$ ve $H$'ye karşı farklı duyarlılık profilleri sergilemesi beklenir. APE gibi küresel ölçüler birikimli sürüklenmeyi (drift) yansıtır ve iterasyon derinliği $\alpha(I)$'ye güçlü tepki verebilir; buna karşın yerel tutarlılık ölçüleri (ör. kısa menzilli RPE veya varyans temelli istatistikler) zamansal koşullandırma $\beta(H)$'ye daha duyarlı olabilir. Yönelim odaklı hatalar (ör. yaw) ise çoğu durumda aşağı akıştaki SLAM arka uç kısıtları ve sahne gözlenebilirliği tarafından baskın biçimde belirlenebilir; bu nedenle ön-işleme parametreleriyle doğrusal biçimde ölçeklenmeyebilir.

Bu çerçeve, bir sonraki bölümde raporlanan deneysel ızgaraların yorumlanması için ilkeli bir temel sağlar; böylece PG tabanlı füzyon çıktıları (PG$\to$GT) ile stereo-only temel çıktı (ZED/RTAB-Map$\to$GT) tutarlı değerlendirme koşulları altında sistematik olarak karşılaştırılabilir.


\section{CitrusFarm Veriseti}

Bu başlık altında Deney senaryolarında bahsi geçen deneyler CitrusFarm veriseti ile gerçeklenmiştir. Deney sonuçları yine deneyler başlığında bahsedildiği şeklinde sergilenmiştir.

Deney sonuçlarındaki bir yöne doğru yünelimin esas sebebi tekerlek odometrisi versindeki biased durumdur. Bunun grafiği eklenmeli. Sürekli bir sola dönüş gösterilmiş. Daha sonra da yaw düzeltmesi ele alınabilir.

\subsection{Deneysel Sonuçlar ve Matematiksel Modelin Doğrulanması}
\label{sec:deneysel_sonuclar}

Bu bölümde, Bölüm~\ref{sec:matematiksel_analiz_cercevesi}’de tanımlanan iki faktörlü toplamsal ayrıştırma modeli kullanılarak elde edilen deneysel sonuçlar sunulmakta ve önerilen Papoulis--Gerchberg (PG) tabanlı LiDAR--stereo füzyon yönteminin davranışı nicel olarak değerlendirilmektedir. Analizler, PG iterasyon sayısı ($I$) ve LU geçmiş boyutu ($H$) parametrelerinin performans metrikleri üzerindeki bireysel ve ortak etkilerini ortaya koymayı amaçlamaktadır.

Tüm sonuçlar aynı veri setleri üzerinde, aynı değerlendirme kriterleri kullanılarak otomatik olarak üretilmiş olup, ilgili görsel çıktılar Ekler bölümünde sunulmaktadır.

\subsubsection{Performans Yüzeyi $M(I,H)$ ve Genel Eğilimler}

Ekler bölümünde sunulan ısı haritaları ve kesitsel grafikler incelendiğinde, performans metriği $M(I,H)$’nin hem PG iterasyon sayısına hem de LU geçmiş boyutuna bağlı olarak sistematik ve düzgün bir değişim sergilediği görülmektedir. Referans yapılandırma $(0,1)$ ile karşılaştırıldığında, orta seviye $(I,H)$ yapılandırmalarında mutlak poz hatasında (APE) yaklaşık \%25--\%40 aralığında bir azalma elde edilmiştir.

Benzer şekilde göreli poz hatası (RPE) metriklerinde, özellikle kısa ve orta ölçeklerde, hata değerlerinin \%20 civarında azaldığı gözlemlenmiştir. Bu durum, önerilen yöntemin yalnızca küresel doğruluğu değil, aynı zamanda yerel tutarlılığı da iyileştirdiğini göstermektedir.

\subsubsection{PG İterasyonlarının Ana Etkisi ($\alpha(I)$)}

PG iterasyonlarının ana etkisi deneysel veriler üzerinden açıkça gözlemlenmektedir. LU geçmiş boyutu sabit tutulduğunda, iterasyon sayısının artırılmasıyla birlikte APE değerlerinde belirgin bir düşüş meydana gelmiştir. Düşük iterasyon seviyelerinde ($I=0 \rightarrow I \approx 10$) hata azalımı yaklaşık \%15--\%20 seviyesindeyken, daha yüksek iterasyonlarda ($I \geq 30$) ek kazanımın \%5’in altına düştüğü görülmektedir.

Bu sonuç, Bölüm~\ref{sec:matematiksel_analiz_cercevesi}’de tanımlanan marjinal iterasyon duyarlılığı
\[
\Delta_I M \approx M(I_2,H) - M(I_1,H)
\]
ifadesiyle uyumlu olup, PG algoritmasının belirli bir iterasyon sayısından sonra yakınsama (saturation) davranışı sergilediğini göstermektedir. Dolayısıyla deneysel veriler, $\alpha(I)$ teriminin düşük ve orta iterasyon seviyelerinde baskın olduğunu, ancak yüksek iterasyonlarda azalan getiri etkisinin devreye girdiğini doğrulamaktadır.

\subsubsection{LU Geçmiş Boyutunun Ana Etkisi ($\beta(H)$)}

LU geçmiş boyutunun etkisi özellikle varyans, maksimum hata ve trajektori pürüzsüzlüğü gibi metriklerde belirgin hale gelmektedir. Küçük geçmiş boyutlarından ($H=1 \rightarrow H \approx 5$) orta seviye geçmiş boyutlarına geçişte, hata varyansında yaklaşık \%20--\%30 oranında bir azalma gözlemlenmiştir.

Buna karşılık, daha büyük geçmiş boyutlarında ek iyileşmenin sınırlı kaldığı ve bazı metriklerde neredeyse plato yaptığı görülmektedir. Bu durum,
\[
\Delta_H M \approx M(I,H_2) - M(I,H_1)
\]
ifadesiyle tanımlanan marjinal geçmiş duyarlılığının azalan bir yapıya sahip olduğunu göstermektedir. Deneysel bulgular, $\beta(H)$ teriminin esas olarak zamansal kararlılığı artıran bir rol üstlendiğini ortaya koymaktadır.

\subsubsection{Etkileşim Terimi ve Süperpozisyon Varsayımı}

PG iterasyonları ve LU geçmiş boyutunun birlikte kullanıldığı yapılandırmalar için süperpozisyon varsayımı deneysel olarak test edilmiştir. Beklenen toplamsal performans ile gerçek ölçümler arasındaki fark
\[
\Gamma(I,H)
\]
incelendiğinde, çoğu $(I,H)$ kombinasyonu için bu farkın sıfıra yakın olduğu görülmektedir. Bu durum, parametrelerin etkilerinin büyük ölçüde toplamsal olduğunu göstermektedir.

Bununla birlikte, orta seviye iterasyon ve geçmiş boyutu kombinasyonlarında $\Gamma(I,H) < 0$ durumları gözlemlenmiş; bu yapılandırmalarda toplam hata azalımının beklenen değerin yaklaşık \%5--\%10 altında gerçekleştiği belirlenmiştir. Bu sonuç, PG iteratif iyileştirme ile zamansal koşullama arasında sınırlı ancak ölçülebilir bir sinerji bulunduğunu göstermektedir.

\subsubsection{Citrus Farm Veri Setinde Yönelim Bias’ının Düzeltilmesi}

Citrus Farm veri setinde kullanılan tekerlek odometrisinin, özellikle uzun sekanslarda sola doğru (counter-clockwise, CCW) yönelim bias’ı içerdiği bilinmektedir. Bu bias, SLAM arka ucunda kullanılan odometrik önbilgiye doğrudan yansımakta ve referans trajektorilere kıyasla belirgin bir yaw sapmasına yol açmaktadır.

Ekler bölümünde sunulan trajektori grafikleri incelendiğinde, referans yapılandırma $(0,1)$ için bu CCW bias’ının açıkça gözlemlendiği; trajektorinin zamanla referanstan sistematik olarak saptığı görülmektedir. Önerilen PG tabanlı füzyon yöntemi uygulandığında ise, bu yönelim hatasının belirgin biçimde bastırıldığı gözlemlenmiştir.

Nicel olarak, yaw hatasında yaklaşık \%30--\%45 aralığında bir azalma elde edilmiş; bu iyileşme, yalnızca lokal düzeltmelerle değil, trajektorinin genel yöneliminin düzeltilmesiyle sağlanmıştır. Bu bulgu, önerilen yöntemin yalnızca ölçüm gürültüsünü azaltmakla kalmadığını, aynı zamanda hatalı odometrik bias’ları da telafi edebildiğini göstermektedir.

\subsubsection{Teori ve Deneysel Bulguların Örtüşmesi}

Genel değerlendirme sonucunda, deneysel bulguların Bölüm~\ref{sec:matematiksel_analiz_cercevesi}’de tanımlanan matematiksel model ile yüksek derecede örtüştüğü görülmektedir. Ana etki terimleri $\alpha(I)$ ve $\beta(H)$, marjinal duyarlılık analizleri ve etkileşim terimleri deneysel veriler üzerinden açıkça doğrulanmıştır.

Bu sonuçlar, önerilen matematiksel çerçevenin, PG tabanlı LiDAR--stereo füzyon sisteminin gerçek veri üzerindeki davranışını açıklamada yeterli ve tutarlı bir analiz aracı sunduğunu göstermektedir.


\section{SensorSuiteV2 Veriseti}

Bu başlık altında Deney senaryolarında bahsi geçen deneyler SensorSuite veriseti ile gerçeklenmiştir. Deney sonuçları yine deneyler başlığında bahsedildiği şeklinde sergilenmiştir.

\section{Paralel İşleme Performansı}

GPU ile sinyal işlemeye dayalı yöntemler geliştirildiği için çeşitli GPU modelleri ile deneyler tekrarlandı. PG algoritması başarımının iterasyon sayısı ile doğru orantılı olduğu ve işlem gücü dar boğazının asıl PG iterasyon sayılarından meydana geldiği göz önünde bulundurularak bağımsız değişkenin GPU çeşiti ve bağımlı değişkenin 100 mili saniyedeki maksimum PG iterasyon sayısı oldu bir deney yapıldı. Deney sonucunda NVidia GPU'lardaki Cuda çekirdek sayıları ile maksimum PG iterasyonu sayısı arasında Şekil~\ref{fig:cuda_core_vs_pg_iter}'de verilen sayısal sonuçlara ulaşıldı.

\begin{figure}[H]
    \begin{subfigure}[t]{0.48\textwidth}
        \centering
        \includegraphics[height=7cm]{./Figures/pg_filter_selection/dama_brick-wall_0.64_2000.png}
        \caption{Karşılaştırma eğrisi}
        \label{fig:cuda_core_vs_pg_iter_chart}
    \end{subfigure}

    \vspace{0.8em}

    % ---------- Row 3 ----------
    \begin{subfigure}[t]{0.48\textwidth}
        \centering

        \renewcommand{\arraystretch}{1.15}
        \begin{tabular}{lcc}
            \toprule
            \textbf{GPU}           & \textbf{Cuda Çekirdeği Sayısı} & \textbf{Max PG İterasyon Sayısı} \\
            \midrule
            Nvidia Jatson Orin NX? & ?                              & 10?                              \\
            RTX2060                & ?                              & 33                               \\
            T4000?                 & ?                              & 50?                              \\
            A4000?                 & ?                              & 100                              \\
            \bottomrule
        \end{tabular}

        \caption{Nümerik karşılaştırma}
        \label{fig:cuda_core_vs_pg_iter_table}
    \end{subfigure}

    \caption{NVidia GPU ve PG iterasyon sayısı karşılaştırması}
    \label{fig:cuda_core_vs_pg_iter}
\end{figure}

Algoritmanın emekleme ve ilk adım safhaları değerlendirilmdiğinde bu değerler oldukça yüksek başarıyı ifade etmektedir. 3B point cloud verisinde yaptığımız ilk PG iterasyonu (tek iterasyon) CPU üzerinde paralel işleme olmaksızın 50 saniyede (50 s) tamamlanırken gelinen son noktada bir iterasyon tüm GPU çetileri göz çnüne alındığında milisaniyenin onda biri (0.1 ms) mertebelerine düşürülmüştür.

Önerdiğimiz yöntemin gerçek zamanlı çalıştırılabilmesi için ekstra efor harcanmış ve başarılı olunmuştur.

evo path karsilastirma metodu bu. bunu anlat. https://en.wikipedia.org/wiki/Kabsch_algorithm