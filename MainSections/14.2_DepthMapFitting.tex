\subsection{Depth Map Fitting} \label{sec:depth_map_fitting}

Bu adımın amacı, odometri ile yoğunlaştırılmış LiDAR nokta bulutunu kamera düzlemine projekte ederek bir LiDAR derinlik imgesi üretmek ve stereo (ZED) derinlik haritası ile aynı piksel uzayında \textit{örtüşen (overlap)} bölgede tutarlı bir derinlik haritası elde etmektir. Bu işlem, PG yinelemelerinden bağımsız olarak, iki sensörün aynı geometri üzerinde hizalanmış tek bir derinlik alanı üretmesini sağlar.

\subsubsection{Girdiler ve gösterim}

Zaman adımı $t_k$ için:
\begin{itemize}
    \item ZED derinlik imgesi: $D_{\text{ZED}}^{\text{orig}}(v,u)$,
    \item Kamera iç parametreleri (pinhole): $(f_x,f_y,c_x,c_y)$,
    \item Kamera koordinat sisteminde (veya kamera frame'ine dönüştürülmüş) LiDAR nokta kümesi:
    $\mathcal{P}_k=\{\mathbf{p}_i\}_{i=1}^{N_k}$, $\mathbf{p}_i=[X_i,Y_i,Z_i]^\top$.
\end{itemize}

Burada $(u,v)$ piksel koordinatlarını, $(X,Y,Z)$ ise kamera koordinat sistemini temsil eder ve $Z>0$ kamera ekseni boyunca ileri yön derinliktir.

\subsubsection{LiDAR nokta bulutundan derinlik imgesine projeksiyon}

Her bir $\mathbf{p}_i$ noktası pinhole model ile görüntü düzlemine projekte edilir:
\begin{equation}
u_i = \mathrm{round}\!\left(\frac{f_x X_i}{Z_i}+c_x\right),
\qquad
v_i = \mathrm{round}\!\left(\frac{f_y Y_i}{Z_i}+c_y\right).
\end{equation}

Geçerli projeksiyon kümesi şu şekilde tanımlanır:
\begin{equation}
\mathcal{V} =
\left\{
i \,\middle|\,
0\le u_i < W,\;
0\le v_i < H,\;
Z_i>0,\;
Z_i \in \mathbb{R}
\right\}.
\end{equation}

LiDAR derinlik imgesi $D_{\text{L}}(v,u)$, aynı piksele düşen birden fazla nokta olması durumunda en yakın derinliği (z-buffer) seçerek oluşturulur:
\begin{equation}
D_{\text{L}}(v,u) =
\min_{i \in \mathcal{V}:\ (u_i,v_i)=(u,v)} Z_i,
\qquad
\text{aksi halde } D_{\text{L}}(v,u)=\mathrm{NaN}.
\end{equation}

\subsubsection{ZED derinlik ön-işleme ve geçersiz maske}

ZED derinlik haritasındaki geçersiz ölçümler için maske:
\begin{equation}
M(v,u) \triangleq \mathbb{I}\big(D_{\text{ZED}}^{\text{orig}}(v,u)\ \text{sonlu değil}\big)
\end{equation}
tanımlanır. İlerleyen adımlarda işlem kararlılığı için $D_{\text{ZED}}^{\text{orig}}$ üzerinde NaN-farkındalıklı bir doldurma (inpainting) uygulanarak
$D_{\text{ZED}}(v,u)$ elde edilir. Bu aşama yalnızca ara hesaplama içindir; nihai sonuçta $M$ maskesi tekrar uygulanır.

\subsubsection{Örtüşme bölgesi (cropping) ve tutarlılık filtresi}

İşlem yalnızca iki sensörün aynı görüş hacminde güvenilir olduğu bir ROI üzerinde yapılır. Bu kırpılmış bölge:
\begin{equation}
\Omega_c =
\left\{
(v,u)\ \middle|\
t \le v < H-b,\;
\ell \le u < W-r
\right\}
\end{equation}
ile tanımlanır (burada $t,b,\ell,r$ ilgili kırpma parametreleridir). Kırpılmış imgeler:
\begin{equation}
D_{\text{ZED}}^{c} = D_{\text{ZED}}|_{\Omega_c},
\qquad
D_{\text{L}}^{c} = D_{\text{L}}|_{\Omega_c}.
\end{equation}

LiDAR projeksiyonundaki sıçramaları (spike) bastırmak için ZED ile farkı çok büyük olan LiDAR pikselleri elenir. Geçerli LiDAR ankraj maskesi:
\begin{equation}
M_{\text{L}}(v,u) \triangleq
\mathbb{I}\big(D_{\text{L}}^{c}(v,u)\ \text{sonlu}\big)\;
\wedge\;
\mathbb{I}\big(\lvert D_{\text{ZED}}^{c}(v,u) - D_{\text{L}}^{c}(v,u)\rvert \le \tau\big),
\quad (v,u)\in\Omega_c
\end{equation}
olarak tanımlanır. Burada $\tau$ sensörler arası maksimum izin verilen derinlik farkıdır.

\subsubsection{Derinlik haritası oturtma (fitting) ve yeniden birleştirme}

Kırpılmış bölgede \textit{fitted} derinlik haritası, geçerli LiDAR ankrajları varsa LiDAR değerini, yoksa ZED değerini seçen birleştirme kuralı ile tanımlanır:
\begin{equation}
D_{\text{fit}}^{c}(v,u) =
\begin{cases}
D_{\text{L}}^{c}(v,u), & \text{eğer } M_{\text{L}}(v,u)=1,\\
D_{\text{ZED}}^{c}(v,u), & \text{aksi halde.}
\end{cases}
\end{equation}

Ardından $D_{\text{fit}}^{c}$, tam çözünürlüklü görüntüye yerleştirilerek $D_{\text{fit}}(v,u)$ elde edilir (ROI dışı pikseller $D_{\text{ZED}}^{\text{orig}}$'dan korunur). Son olarak, özgün ZED geçersiz maskesi tekrar uygulanır:
\begin{equation}
M(v,u)=1 \ \Rightarrow\ D_{\text{fit}}(v,u)=\mathrm{NaN}.
\end{equation}

Bu çıktı, bir sonraki füzyon adımına (PG veya doğrudan geri-izdüşüm) giriş olarak kullanılır.

\begin{figure}[H]
    \centering
    \begin{adjustbox}{width=\textwidth * 10 /12}
        \includegraphics{mermaid/depth_map_fitting.png}
    \end{adjustbox}
    \caption{DMF akış şeması}
    \label{fig:depth_map_fitting_flow}
\end{figure}