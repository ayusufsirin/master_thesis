\section{Derinlik Haritası Uyumlama} \label{sec:depth_map_fitting}

Bu adımın amacı, odometri ile yoğunlaştırılmış LiDAR nokta bulutunu kamera düzlemine projekte ederek bir LiDAR derinlik imgesi üretmek ve stereo (ZED) derinlik haritası ile aynı piksel uzayında \textit{örtüşen (overlap)} bölgede tutarlı bir derinlik haritası elde etmektir. Bu işlem, PG yinelemelerinden bağımsız olarak, iki sensörün aynı geometri üzerinde hizalanmış tek bir derinlik alanı üretmesini sağlar.

\subsubsection{Gösterim ve Girdiler}

ZED stereo kameradan elde edilen derinlik görüntüsü aşağıdaki gibi tanımlanmaktadır:
\begin{equation}
    D_{\text{ZED}}^{\text{orig}}(v,u) \in \mathbb{R} \cup \{\text{NaN}\},
\end{equation}
burada piksel koordinatları
\begin{equation}
    u \in \{0,\dots,W-1\}, \quad v \in \{0,\dots,H-1\}
\end{equation}
şeklindedir.

LiDAR sensörü tarafından döndürülen üç boyutlu nokta bulutu
\begin{equation}
    \mathcal{P}_{\text{L}}
    =
    \left\{
        \mathbf{p}_k
    =
        \begin{bmatrix}
            x_k \\ y_k \\ z_k
        \end{bmatrix}
        \in \mathbb{R}^3
    \right\}_{k=1}^{N}
\end{equation}
olarak tanımlanmaktadır.

Kamera içsel parametreleri (ROS \texttt{CameraInfo} mesajından elde edilmektedir) şu şekildedir:
\begin{equation}
    f_x, f_y \quad \text{(odak uzaklıkları)}, \qquad
    c_x, c_y \quad \text{(ana nokta koordinatları)}.
\end{equation}

Derinlik işlemlerinde kullanılan kırpma bölgesi
\begin{equation}
    \Omega_c = \left\{ (v,u) \;\middle|\;
                   t \le v < H - b,\
                   \ell \le u < W - r
    \right\}
\end{equation}
olarak tanımlanır. Burada
\begin{equation}
\begin{aligned}
    t &= \texttt{MORTAL\_ROWS\_TOP} \\
    b &= \texttt{MORTAL\_ROWS\_BOTTOM} \\
    \ell &= \texttt{MORTAL\_COLUMNS\_LEFT} \\
    r &= \texttt{MORTAL\_COLUMNS\_RIGHT}
\end{aligned}
\end{equation}
parametreleri görüntüden çıkarılacak bölgeleri ifade etmektedir.

ZED ve LiDAR derinlikleri arasındaki tutarlılık eşiği ise
\begin{equation}
    \tau = \texttt{ZED\_VLP\_DIFF\_MAX}
\end{equation}
olarak tanımlanmaktadır.

\subsubsection{LiDAR Kartezyen Gösteriminden Küresel Gösterime Dönüşüm}

Her bir LiDAR noktası $\mathbf{p}_k = (x_k, y_k, z_k)^\top$ için küresel koordinatlar aşağıdaki şekilde tanımlanmaktadır:
\begin{align}
    r_k     &= \sqrt{x_k^2 + y_k^2 + z_k^2}, \\
    \theta_k &= \arctan\!\left( \frac{z_k}{\sqrt{x_k^2 + y_k^2}} \right), \\
    \phi_k   &= \arctan\!\left( \frac{y_k}{x_k} \right),
\end{align}
ve bu gösterim
\begin{equation}
    \mathbf{s}_k =
    \begin{bmatrix}
        r_k \\ \theta_k \\ \phi_k
    \end{bmatrix}
\end{equation}
şeklinde ifade edilir.

Bu küresel gösterim, uygulama kapsamında açısal filtreleme ve tanılama (diagnostics) amaçlı olarak kullanılmaktadır.

\subsubsection{LiDAR Noktalarının Kamera Düzlemine Yansıtılması}

\paragraph{Eksen Yeniden Eşlemesi.}
LiDAR noktaları öncelikle kamera benzeri bir koordinat sistemine dönüştürülmektedir:
\begin{equation}
    \begin{bmatrix}
        X_k \\ Y_k \\ Z_k
    \end{bmatrix}
    =
    R
    \begin{bmatrix}
        x_k \\ y_k \\ z_k
    \end{bmatrix},
    \qquad
    R =
    \begin{bmatrix}
        0 & -1 & 0 \\
        0 &  0 & -1 \\
        1 &  0 & 0
    \end{bmatrix}.
\end{equation}
Bu dönüşüm sonucunda $Z_k$ ekseni kameraya doğru olan ileri yönü, $X_k$ yatay ekseni ve $Y_k$ ise dikey ekseni temsil etmektedir.

\paragraph{İğne Deliği (Pinhole) Kamera Modeli.}
Kamera izdüşümü aşağıdaki denklemlerle elde edilmektedir:
\begin{align}
    \tilde{u}_k &= \frac{X_k f_x}{Z_k} + c_x, \\
    \tilde{v}_k &= \frac{Y_k f_y}{Z_k} + c_y.
\end{align}
Bu değerler, piksel indislerine yuvarlanarak dönüştürülür:
\begin{equation}
    u_k = \operatorname{round}(\tilde{u}_k), \qquad
    v_k = \operatorname{round}(\tilde{v}_k).
\end{equation}

Yalnızca kamera görüş alanında kalan ve kameranın önünde bulunan noktalar korunur:
\begin{equation}
    \mathcal{K}
    =
    \left\{ k \;\middle|\;
        0 \le u_k < W,\
        0 \le v_k < H,\
        Z_k > 0,\
        Z_k \text{ sonlu}
    \right\}.
\end{equation}

\paragraph{LiDAR Derinlik Görüntüsünün Oluşturulması.}
LiDAR derinlik görüntüsü, aynı piksele düşen noktalar arasından en yakın yüzeyin seçilmesiyle elde edilir:
\begin{equation}
    D_{\text{L}}(v,u) =
    \begin{cases}
        \displaystyle
        \min\limits_{k \in \mathcal{K} : (v_k,u_k) = (v,u)} Z_k,
        & \text{eğer böyle bir } k \text{ varsa}, \\[1.2ex]
        \text{NaN}, & \text{aksi halde}.
    \end{cases}
\end{equation}

Bu işlem, LiDAR nokta bulutundan elde edilen en yakın yüzeye karşılık gelen bir derinlik görüntüsü üretmektedir.

\subsubsection{ZED Derinlik Haritasının Doldurulması (Inpainting)}

Ham ZED derinlik haritası $D_{\text{ZED}}^{\text{orig}}(v,u)$ geçersiz değerler (NaN veya $\pm\infty$) içerebilmektedir. Bu nedenle bir geçersizlik maskesi şu şekilde tanımlanır:
\begin{equation}
    M(v,u) =
    \begin{cases}
        1, & \text{eğer } D_{\text{ZED}}^{\text{orig}}(v,u) \text{ NaN veya } \pm\infty \text{ ise}, \\
        0, & \text{aksi halde}.
    \end{cases}
\end{equation}

Başlangıç derinlik haritası
\begin{equation}
    D^{(0)}(v,u) =
    \begin{cases}
        D_{\text{ZED}}^{\text{orig}}(v,u), & M(v,u) = 0, \\
        0, & M(v,u) = 1
    \end{cases}
\end{equation}
şeklinde tanımlanır.

Geçersiz pikseller için, dört bağlantılı komşuluk
\begin{equation}
    \mathcal{N}(v,u) =
    \{(v-1,u), (v+1,u), (v,u-1), (v,u+1)\}
\end{equation}
kullanılarak yinelemeli bir doldurma işlemi uygulanır.

$t+1$ iterasyonunda güncelleme
\begin{equation}
    D^{(t+1)}(v,u)
    =
    \frac{
        \displaystyle
        \sum\limits_{(p,q) \in \mathcal{N}(v,u)}
        D^{(t)}(p,q)\,
        \mathbf{1}_{\{M(p,q) = 0\}}
    }{
        \displaystyle
        \sum\limits_{(p,q) \in \mathcal{N}(v,u)}
        \mathbf{1}_{\{M(p,q) = 0\}}
        + \varepsilon
    }
\end{equation}
şeklindedir. Burada $\varepsilon \ll 1$ sıfıra bölmeyi önlemek amacıyla eklenmiştir.

$T$ iterasyon sonunda doldurulmuş ZED derinlik haritası
\begin{equation}
    D_{\text{ZED}}(v,u) = D^{(T)}(v,u)
\end{equation}
olarak elde edilir.

\subsubsection{Kırpma ve ZED--LiDAR Tutarlılık Filtresi}

Derinlik haritaları kırpma bölgesi ile sınırlandırılır:
\begin{align}
    D_{\text{ZED}}^{c}(v,u) &= D_{\text{ZED}}(v,u), \\
    D_{\text{L}}^{c}(v,u)   &= D_{\text{L}}(v,u),
\end{align}
\quad $(v,u) \in \Omega_c$.

ZED ve LiDAR arasındaki tutarsız ölçümler şu koşula göre elenir:
\begin{equation}
    D_{\text{L}}^{c}(v,u) \leftarrow
    \begin{cases}
        D_{\text{L}}^{c}(v,u),
        & \left| D_{\text{ZED}}^{c}(v,u) - D_{\text{L}}^{c}(v,u) \right| \le \tau, \\
        \text{NaN},
        & \text{aksi halde}.
    \end{cases}
\end{equation}

Bu çalışmada, kırpma bölgesi içerisindeki DMF derinliği doğrudan filtrelenmiş LiDAR derinliği olarak alınmaktadır:
\begin{equation}
    D_{\text{fit}}^{c}(v,u) = D_{\text{L}}^{c}(v,u).
\end{equation}

Bu çıktı, bir sonraki füzyon adımına (örneğin Bölüm~\ref{sec:dynamic_pg} altında tanımlanan yayınlama/sonlandırma kuralı veya iteratif bir iyileştirme yöntemi) giriş olarak kullanılır.

\begin{figure}[H]
    \centering
    \begin{adjustbox}{width=\textwidth * 10 /12}
        \includegraphics{mermaid/depth_map_fitting.png}
    \end{adjustbox}
    \caption{DMF akış şeması}
    \label{fig:depth_map_fitting_flow}
\end{figure}