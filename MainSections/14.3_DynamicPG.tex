\section{Dinamik Papoulis-Gerchberg}  \label{sec:dynamic_pg}

Bu bölümde, DMF adımı ile aynı piksel uzayında hizalanmış hâle getirilen ZED ve LiDAR derinlik bilgilerinden hareketle, yayınlanan nihai derinlik haritasının $D_{\text{PG}}(v,u)$ nasıl oluşturulduğu tanımlanmaktadır.
Bu alt bölüm, PG'nin iteratif çekirdek güncelleme denklemlerinden ziyade, çalışma kapsamında \textit{PG çıktısı olarak yayınlanan derinlik alanının} birleştirme ve maskeleme kuralını sunmaktadır.

Matematiksel formulasyona geçmeden önce algoritmanın geliştirilmesi ile varılmak istenen noktayı analojik olarak balsa uçak kanadı modeli üzerinden anlatımı

\subsection{Sezgisel Analoji: Balsa Kanat Üzerine Gerilen Film}

Bu çalışmada önerilen Dinamik Papoulis--Gerchberg yaklaşımı, sezgisel olarak, model uçak yapımında balsa ağacından oluşturulan kanat iskeleti üzerine gerilen ince kaplama filmi analojisi ile açıklanabilir (bkz. Şekil~\ref{fig:film_analogy}). Balsa iskelet, kanadın temel geometrisini ve taşıyıcı hatlarını belirlerken; üzerine gerilen film, bu iskelet boyunca gerdirilerek sürekli ve aerodinamik bir yüzey oluşturur. Film gerdirildiğinde, iskeletin belirlediği ana hatları takip ederken yerel ölçekte küçük gerilmeler, hafif deformasyonlar veya yüzey düzgünsüzlükleri oluşabilir. Ancak bu mikroskobik bozulmalar, yüzeyin genel formunu bozmaz; aksine, film sayesinde kanat, çıplak iskelete kıyasla çok daha keskin, sürekli ve tanımlı bir profile kavuşur.

\begin{figure}[H]
    \centering
    \begin{adjustbox}{width=\textwidth * 9 / 12}
        \includegraphics{./Figures/analogy.png}
    \end{adjustbox}
    \caption{Balsa kanat üzerine gerilen kaplama filmi analojisi. Rijit balsa iskelet, kanadın ana geometrik hatlarını belirlerken; üzerine gerilen film, bu kılavuz yapı boyunca sürekli bir yüzey oluşturmaktadır. Film gerilme sürecinde lokal deformasyonlar gösterebilse de, bu durum yüzeyin genel formunu bozmaz ve aksine daha keskin, tanımlı bir yapı elde edilmesini sağlar. Bu davranış, Dinamik Papoulis--Gerchberg yaklaşımında ZED derinlik haritasının LiDAR kılavuz noktalarına oturtulmasına sezgisel bir karşılık sunmaktadır.}
    \label{fig:film_analogy}
\end{figure}

Bu analojide, LiDAR ölçümleri kanat iskeletini oluşturan rijit kılavuz noktalar olarak düşünülebilirken, stereo (ZED) derinlik haritası ise bu iskelet üzerine gerilen sürekli film yüzeyine karşılık gelmektedir. LiDAR noktaları, sahnenin geometrik doğruluğunu ve ana yapısal sınırlarını belirlerken; ZED derinliği bu noktalar arasında sürekli bir yüzey oluşturarak yoğun ve pürüzsüz bir derinlik alanı sağlar.

\subsection{Dinamik PG ile Analoji Arasındaki İlişki}

Dinamik Papoulis--Gerchberg tabanlı derinlik füzyonu da benzer şekilde, ZED derinlik haritasını LiDAR tarafından sağlanan güvenilir kılavuz noktalar üzerine ``gererek'' oturtmaktadır. Bu süreçte, ZED derinliğinin bazı bölgelerde LiDAR ölçümlerine uyum sağlamak adına lokal olarak deforme olması kaçınılmazdır. Ancak bu deformasyonlar, tıpkı kanat üzerindeki filmde olduğu gibi, global geometrik bütünlüğü bozacak nitelikte değildir. Aksine, LiDAR kılavuz noktalarının zorlayıcı etkisi sayesinde, yüzeyin keskin kenarları, yapı sınırları ve derinlik süreksizlikleri daha net hâle gelmektedir.

Bu bağlamda, elde edilen derinlik haritasının pürüzsüzlüğünden ziyade, geometrik tutarlılığı ve yapısal doğruluğu ön plana çıkmaktadır. SLAM ve odometri performansı açısından kritik olan unsur, yüzeyin lokal olarak ne kadar ideal olduğu değil; sahnedeki nesnelerin göreli konumlarının, kenarlarının ve süreksizliklerinin ne kadar doğru temsil edildiğidir. Tıpkı gerçek bir uçakta olduğu gibi, kanat yüzeyindeki mikroskobik gerilme izleri uçuş performansını olumsuz etkilemez; önemli olan, kanadın genel formunun ve taşıyıcı geometrisinin doğru olmasıdır. Benzer şekilde, Dinamik PG yaklaşımında da küçük lokal bozulmalar tolere edilebilir olup, asıl kazanım SLAM algoritmalarının daha kararlı, tutarlı ve tekrar edilebilir sonuçlar üretmesidir.

\subsection{Derinlik Füzyon Hattının Matematiksel Formülasyonu}

Bu bölümde, ROS--CuPy tabanlı ZED--LiDAR füzyon düğümünün altında yatan matematiksel model formel olarak sunulmaktadır. Uygulama, stereo kameradan (ZED) elde edilen senkronize derinlik görüntüleri ile LiDAR sensöründen alınan nokta bulutlarını DMF[~\ref{sec:depth_map_fitting}] önişlemesinden sonra girdi olarak almakta ve gerçek zamanlı olarak füzyonlanmış bir derinlik görüntüsü ile buna karşılık gelen bir nokta bulutu üretmektedir.

\begin{figure}[H]
    \centering
    \begin{adjustbox}{width=\textwidth}
        \includegraphics[width=\textwidth]{mermaid/pg.png}
    \end{adjustbox}
    \caption{Füzyon algoritması akış diyagramı}
    \label{methodology_2}
\end{figure}

Bu bölümde kullanılan kırpma bölgesi $\Omega_c$,
ZED geçersizlik maskesi $M(v,u)$,
ve ZED--LiDAR tutarlılık/eşikleme mekanizması
Depth Map Fitting (DMF) aşamasında tanımlandığı şekliyle
doğrudan kullanılmaktadır (bkz. Denklem~\eqref{eq:dmf_roi}--\eqref{eq:dmf_anchor_mask}).
Dolayısıyla bu bölümde amaç, DMF ile ortak piksel uzayına taşınmış
$D_{\text{ZED}}^{\text{orig}}$ ve $D_{\text{L}}^{c}$ derinliklerinin,
nihai yayınlanan derinlik alanı $D_{\text{PG}}(v,u)$ üzerinde nasıl birleştirildiğini
ve maske koşulları altında nasıl sonlandırıldığını formel olarak ifade etmektir.

\subsubsection{Nihai Füzyonlanmış Derinlik Haritası}

Yayınlanan nihai derinlik haritası $D_{\text{PG}}(v,u)$ aşağıdaki adımlar ile oluşturulmaktadır:

\begin{enumerate}
    \item Başlangıçta ham ZED derinliği atanır.
    \item Kırpma bölgesinde geçerli LiDAR derinliği mevcutsa ZED değeri üzerine yazılır.
    \item Orijinal ZED geçersizlik maskesi tekrar uygulanır.
\end{enumerate}

Bu adımlar birleştirildiğinde:
\begin{equation}
    D_{\text{PG}}(v,u) =
    \begin{cases}
        \text{NaN},
        & M(v,u) = 1, \\
        D_{\text{L}}^{c}(v,u),
        & (v,u) \in \Omega_c,\ D_{\text{L}}^{c}(v,u) \text{ sonlu}, \\
        D_{\text{ZED}}^{\text{orig}}(v,u),
        & \text{aksi halde}
    \end{cases}
\end{equation}
elde edilir.

\subsubsection{3B Geri Yansıtım ve Çıkış Nokta Bulutu (Yardımcı adım)}

Son olarak, füzyonlanmış derinlik haritası kamera içsel parametreleri kullanılarak üç boyutlu nokta bulutuna dönüştürülür. Her bir geçerli piksel için:
\begin{equation}
    \mathcal{P}_{\text{PG}}
    =
    \left\{
        \mathbf{p}'(v,u)
        \;\middle|\;
        D_{\text{PG}}(v,u) \text{ sonlu}
    \right\}
\end{equation}
elde edilir ve bu nokta kümesi \texttt{sensor\_msgs/PointCloud2} mesajı olarak yayınlanır.

\subsection{Sentetik Veri ile Algoritmanın Doğrulanması}
Gerçek sensör verisiyle yapılan iyileştirme denemelerinde gözlemlenen artefaktların (özellikle ``spike'' benzeri geometrilerin ve yakınsama kararsızlıklarının) kaynağını daha net ayırt edebilmek amacıyla, algoritmanın doğrulama aşamasında kontrollü bir sentetik veri seti kullanılmıştır. Bu doğrulama yaklaşımının temel motivasyonu; sensör gürültüsü, görüş alanı (FOV) uyuşmazlığı, eşzamanlılık sapmaları, hareket kaynaklı bozulmalar ve eksik örneklem gibi gerçek dünya etkilerini geçici olarak devre dışı bırakarak, Papoulis--Gerchberg (PG) tabanlı iyileştirme sürecinin ``beklenen'' davranışını gözlemleyebilmektir.

Bu kapsamda dama tahtası (satranç tahtası) formunda üretilmiş sentetik desen üzerinden iki ayrı kaynak veri hazırlanmıştır: (i) ZED stereo kamera temsili nokta/derinlik verisi ve (ii) VLP-16 LiDAR temsili nokta bulutu. Şekil~\ref{fig:synthetic_validation_triplet} içerisinde sırasıyla ZED temsili veri (Şekil~\ref{fig:dama_zed}), VLP temsili veri (Şekil~\ref{fig:dama_vlp}) ve PG çıktısı (Şekil~\ref{fig:dama_pg}) birlikte sunulmuştur. Bu deneyde, desenin yüksek kontrastlı ve düzenli geometrik yapısı sayesinde, PG algoritmasının frekans alanındaki kısıtlama ve uzamsal alandaki yeniden yapılandırma adımlarının ürettiği etki kolayca izlenebilmiştir.

\begin{figure}[H]
    \centering

    \begin{subfigure}[t]{0.6\textwidth}
        \centering
        \includegraphics[width=\linewidth]{./Figures/pg_synthetic_validation/dama_zed.png}
        \caption{ZED stereo kameradan elde edilen sentetik dama tahtası verisi}
        \label{fig:dama_zed}
    \end{subfigure}

    \vspace{0.6em}

    \begin{subfigure}[t]{0.6\textwidth}
        \centering
        \includegraphics[width=\linewidth]{./Figures/pg_synthetic_validation/dama_vlp.png}
        \caption{VLP-16 LiDAR’dan elde edilen sentetik dama tahtası verisi}
        \label{fig:dama_vlp}
    \end{subfigure}

    \vspace{0.6em}

    \begin{subfigure}[t]{0.6\textwidth}
        \centering
        \includegraphics[width=\linewidth]{./Figures/pg_synthetic_validation/dama_pg.png}
        \caption{Papoulis--Gerchberg (PG) algoritması sonrası elde edilen çıktı}
        \label{fig:dama_pg}
    \end{subfigure}

    \caption{Sentetik dama tahtası verisi ile ZED, VLP-16 ve PG çıktılarının karşılaştırılması. Bu deney, gerçek veri kaynaklı belirsizliklerden bağımsız olarak algoritmanın temel yakınsama davranışını doğrulamak amacıyla gerçekleştirilmiştir.}
    \label{fig:synthetic_validation_triplet}
\end{figure}

Görsellerde görüldüğü üzere LiDAR verisine karşılık üretilen sentetik veri satranç tahtası üzerinde ölçüm hassasiyeti çok yüksek ancak seyrek noktalardan oluşmaktadır. Benzer mantıkla stereo sentetik verisi ise yüksek çözünürlüklü ancak gürültülüdür.

Elde edilen sonuçlar, algoritmanın sentetik veri üzerinde kararlı bir yakınsama sergilediğini ve iki veri kaynağı arasında beklenen yönde bir uyum üretebildiğini göstermektedir. Ayrıca danışman değerlendirmesi kapsamında, elde edilen çıktının görsel olarak tutarlı bulunduğu ve algoritmanın temel çalışma prensibinin doğrulandığı not edilmiştir. Bu bulgu, gerçek veri ile benzer bir görsel çıktının elde edilememesi durumunda problemin doğrudan algoritmanın çekirdek mantığından ziyade, gerçek sensör koşullarına (özellikle ZED ve VLP-16 görüş alanlarının PG sürecinde etkin kullanılan ortak FOV ile tam örtüşmemesi, senkronizasyon/örnekleme farklılıkları ve tarihsel LiDAR birikiminin gölgeleme etkileri) bağlı olabileceğine işaret etmektedir.

Dolayısıyla bu doğrulama adımı, çalışmanın devamında gerçekleştirilen filtre tipi seçimi ve parametre ayarlama sürecine metodolojik bir temel sağlamaktadır. Sentetik veri deneyinde algoritmanın ``ideal'' koşullarda çalıştığının gösterilmesi sayesinde, bir sonraki aşamada ele alınan filtre seçimi (Brick-wall, Gaussian, Butterworth) ve cutoff/iterasyon gibi parametrelerin belirlenmesi, gerçek veride ortaya çıkan artefaktları azaltmaya ve yakınsamayı daha kararlı hâle getirmeye yönelik sistematik bir optimizasyon problemi olarak ele alınabilmiştir. Bu nedenle, sentetik doğrulama sonrasında Bölüm~\ref{sec:pg_filter_selection} ve Bölüm~\ref{sec:pg_parameter_tuning} altında sunulan filtre/parametre analizlerine geçilmiştir.

\subsection{3B İçin Filtre Seçimi} \label{sec:pg_filter_selection}

Bu çalışmada, LiDAR (VLP-16) ve stereo kamera (ZED) verilerinin birlikte kullanıldığı üç boyutlu veri füzyonu sürecinde, Papoulis--Gerchberg (PG) tabanlı iyileştirme algoritmasının filtreleme parametrelerine olan duyarlılığı incelenmiştir. PG algoritması, eksik veya örtüşmeyen frekans bileşenlerinin iteratif olarak yeniden yapılandırılmasına dayandığından, özellikle frekans alanında uygulanan filtreleme işlemleri, sonuç geometrisi üzerinde belirleyici bir etkiye sahiptir.

Gerçek veri üzerinde yapılan deneylerde, sensörlerin anlık olarak algılayamadığı ancak zamansal olarak biriktirilmiş (historical) LiDAR noktalarının, PG algoritmasını fiziksel olarak anlamlı olmayan yakınsama davranışlarına zorladığı gözlemlenmiştir. Bu durum, düşük geçiren filtrelerin (Low-Pass Filter, LPF) yanlış yapılandırılması halinde, uzaysal alanda orijine doğru uzanan ve “spike” benzeri yapay geometrilerin ortaya çıkmasına neden olmaktadır. Bu bağlamda, filtreleme süreci yalnızca gürültü bastırma amacıyla değil, aynı zamanda sensörler arası tutarsızlıkların etkisini sınırlamak için ele alınmıştır.

\subsubsection{Cutoff Frekansını Ayarlama}

Filtreleme sürecinde kullanılan cutoff frekansı, Papoulis--Gerchberg algoritmasının yakınsama karakteristiğini doğrudan etkilemektedir. Bu çalışmada, farklı normalize cutoff frekans değerleri için (0.01--0.16 aralığında) sabit iterasyon sayısı altında elde edilen çıktılar karşılaştırılmıştır. Düşük cutoff değerlerinde, yüksek frekans bileşenlerinin büyük ölçüde bastırıldığı ve bunun sonucunda özellikle ince geometrik detayların kaybolduğu gözlemlenmiştir. Buna karşılık, yüksek cutoff değerlerinde ise sensörler arası uyumsuzluklardan kaynaklanan yüksek frekanslı bileşenlerin yeterince bastırılamadığı görülmüştür.

Sentetik satranç tahtası (checkerboard) verisi üzerinde gerçekleştirilen deneyler (Şekil~\ref{fig:pg_cutoff_comparison}), cutoff frekansının uygun seçilmesi halinde PG algoritmasının teorik beklentilerle uyumlu şekilde çalıştığını doğrulamıştır. Ancak gerçek veri üzerinde, ZED kamera ve VLP LiDAR sensörlerinin görüş alanlarının (Field of View, FOV) tam olarak örtüşmemesi nedeniyle, aynı cutoff değerlerinin her durumda benzer sonuçlar üretmediği tespit edilmiştir. Bu durum, cutoff frekansının sabit bir parametre olarak değil, sensör geometrisi ve veri dağılımına bağlı olarak değerlendirilmesi gerektiğini göstermektedir.

\begin{figure}[H]
    \centering

    % -------- Row 1 --------
    \begin{subfigure}[t]{0.48\textwidth}
        \centering
        \includegraphics[height=6cm]{./Figures/pg_filter_selection/dama_0.01_100.png}
        \caption{Cutoff = 0.01, Iter = 100}
        \label{fig:pg_001}
    \end{subfigure}
    \hfill
    \begin{subfigure}[t]{0.48\textwidth}
        \centering
        \includegraphics[height=6cm]{./Figures/pg_filter_selection/dama_0.02_100.png}
        \caption{Cutoff = 0.02, Iter = 100}
        \label{fig:pg_002}
    \end{subfigure}

    \vspace{0.8em}

    % -------- Row 2 --------
    \begin{subfigure}[t]{0.48\textwidth}
        \centering
        \includegraphics[height=6cm]{./Figures/pg_filter_selection/dama_0.04_100.png}
        \caption{Cutoff = 0.04, Iter = 100}
        \label{fig:pg_004}
    \end{subfigure}
    \hfill
    \begin{subfigure}[t]{0.48\textwidth}
        \centering
        \includegraphics[height=6cm]{./Figures/pg_filter_selection/dama_0.08_100.png}
        \caption{Cutoff = 0.08, Iter = 100}
        \label{fig:pg_008}
    \end{subfigure}

    \vspace{0.8em}

    % -------- Row 3 --------
    \begin{subfigure}[t]{\textwidth}
        \centering
        \includegraphics[height=6cm]{./Figures/pg_filter_selection/dama_0.16_100.png}
        \caption{Cutoff = 0.16, Iter = 100}
        \label{fig:pg_016}
    \end{subfigure}

    \caption{Papoulis--Gerchberg algoritmasında farklı normalize cutoff frekansları için elde edilen çıktılar.
    Düşük cutoff değerlerinde yüksek frekans bileşenlerinin baskılandığı, yüksek cutoff değerlerinde ise detayların
    daha belirgin hâle geldiği gözlemlenmektedir.}
    \label{fig:pg_cutoff_comparison}
\end{figure}

Ayrıca, derinlik imajlarının kare olmayan boyutlara sahip olması durumunda, frekans alanında kullanılan dairesel maskelemenin hatalı sonuçlar ürettiği gözlemlenmiştir. Bu problem, eliptik maske kullanımı ile giderilmiş ve dikdörtgen ve kare imajlar arasında tutarlı sonuçlar elde edilmiştir.

\subsubsection{Filtre Tipi Seçimi}

Varsayılan olarak kullanılan brick-wall (ideal) düşük geçiren filtre, frekans alanında keskin bir kesim uygulaması nedeniyle uzaysal alanda salınımlara (ringing artefaktları) yol açmaktadır. Bu durum, özellikle sensörler arası mesafenin fazla olduğu bölgelerde, PG çıktısında yapay dalgalanmalara neden olmaktadır. Bu nedenle, çalışmada brick-wall filtreye alternatif olarak Gaussian ve Butterworth filtreler değerlendirilmiştir.

Farklı filtre tiplerinin aynı parametrelerdeki çıktıları Şekil~\ref{fig:pg_filter_comparison} içinde birlikte sunulmuştur. Gaussian filtre (Şekil~\ref{fig:pg_gaussian}), frekans alanında yumuşak bir geçiş sağladığından, yüksek frekans bileşenlerini kademeli olarak bastırmakta ve uzaysal alanda daha kararlı sonuçlar üretmektedir. Butterworth filtre (Şekil~\ref{fig:pg_butterworth}) ise geçiş bölgesinin eğimini kontrol edilebilir kılması sayesinde, brick-wall filtreden (Şekil~\ref{fig:pg_brickwall}) daha yumuşak, Gaussian filtreden ise daha seçici bir yapı sunmaktadır. Yapılan deneyler, her iki filtrenin de brick-wall filtreye kıyasla, özellikle spike oluşumlarını belirgin şekilde azalttığını göstermiştir.

Bununla birlikte, Gaussian ve Butterworth filtrelerin kullanımı sonucunda, zemin düzlemi gibi LiDAR verisinin yoğun olduğu bölgelerde bozulmalar meydana geldiği gözlemlenmiştir. Bu durum, filtreleme işleminin tüm sahneye homojen biçimde uygulanmasının her zaman ideal olmadığını ve baskın geometrik yapıların ayrı ele alınması gerektiğini ortaya koymaktadır.

\begin{figure}[H]
    \centering

    % ---------- Row 1 ----------
    \begin{subfigure}[t]{0.48\textwidth}
        \centering
        \includegraphics[height=7cm]{./Figures/pg_filter_selection/dama_gaussian_0.64_2000.png}
        \caption{Gaussian LPF çıktısı}
        \label{fig:pg_gaussian}
    \end{subfigure}
    \hfill
    \begin{subfigure}[t]{0.48\textwidth}
        \centering
        \includegraphics[height=7cm]{./Figures/pg_filter_selection/dama_butterworth_0.64_2000.png}
        \caption{Butterworth LPF çıktısı}
        \label{fig:pg_butterworth}
    \end{subfigure}

    \vspace{0.8em}

    % ---------- Row 2 ----------
    \begin{subfigure}[t]{0.48\textwidth}
        \centering
        \includegraphics[height=7cm]{./Figures/pg_filter_selection/dama_brick-wall_0.64_2000.png}
        \caption{Brick-wall LPF çıktısı}
        \label{fig:pg_brickwall}
    \end{subfigure}

    \vspace{0.8em}

    % ---------- Row 3 ----------
    \begin{subfigure}[t]{\textwidth}
        \centering

        \renewcommand{\arraystretch}{1.15}
        \begin{tabular}{lccc}
            \toprule
            \textbf{Filtre Tipi} & \textbf{Cutoff} & \textbf{Iterasyon} & \textbf{Notlar}            \\
            \midrule
            Gaussian             & 0.64            & 2000               & Yumuşak geçiş, düşük spike \\
            Butterworth          & 0.64            & 2000               & Dengeli yapı               \\
            Brick-wall           & 0.64            & 2000               & Keskin geçiş, artefakt     \\
            \bottomrule
        \end{tabular}

        \caption{Filtre parametreleri}
        \label{fig:pg_filter_table}
    \end{subfigure}

    \caption{Papoulis--Gerchberg algoritması için farklı düşük geçiren filtrelerin karşılaştırılması.
    Sarı noktalar ZED, yeşil noktalar VLP, uzaklığa göre geçişli renklerle gösterilen noktalar füzyon çıktısı verileri temsil etmektedir.}
    \label{fig:pg_filter_comparison}
\end{figure}

\subsubsection{Filtrelerin Çapraz Karşılaştırılması} \label{sec:pg_parameter_tuning}

Filtre tiplerinin karşılaştırmalı analizi hem sentetik hem de gerçek veri üzerinde gerçekleştirilmiştir. Sentetik veride, ideal koşullar altında filtrelerin benzer yakınsama davranışı sergilediği görülürken, gerçek veri üzerinde belirgin farklar ortaya çıkmıştır. Brick-wall filtre, frekans alanında keskin bir kesim uyguladığı için PG çıktısında artefaktlara ve yer yer kararsız yakınsamaya neden olmuştur. Bu çalışmada tercih edilen Gaussian filtre ise yumuşak geçiş karakteristiği sayesinde sensörler arası uyumsuzlukları daha iyi tolere etmiş ve görsel olarak daha kararlı, daha tutarlı bir iyileştirme sağlamıştır. Butterworth filtre belirli senaryolarda dengeli bir alternatif sunsa da, yapılan deneylerde en tutarlı sonuçlar Gaussian filtre ile elde edilmiştir (Bakınız Şekil~\ref{fig:pg_gaussian_vs_brickwall_citrus}).

\begin{figure}[H]
    \centering

    \begin{subfigure}[t]{0.48\textwidth}
        \centering
        \includegraphics[height=5.5cm]{./Figures/pg_filter_selection/pg_nan_filled_1.png}
        \caption{Brick-wall LPF ile elde edilen PG çıktısı (Kırmızı noktalar PG, sarı noktalar ZED, yeşil noktalar VLP çıktısı)}
        \label{fig:pg_brickwall_not_variance_filtered}
    \end{subfigure}
    \hfill
    \begin{subfigure}[t]{0.48\textwidth}
        \centering
        \includegraphics[height=5.5cm]{./Figures/pg_filter_selection/variance_filtered.png}
        \caption{Brick-wall LPF ve Varyans Filtresi ile elde edilen PG çıktısı (Kırmızı noktalar PG, sarı noktalar ZED, yeşil noktalar VLP çıktısı)}
        \label{fig:pg_brickwall_variance_filtered}
    \end{subfigure}

    \vspace{0.8em}

    \begin{subfigure}[t]{\textwidth}
        \centering
        \includegraphics[width=0.5\linewidth]{./Figures/pg_filter_selection/citrus_gaussian_0.16_33.png}
        \caption{Gaussian LPF ve Varyans Filtresi ile elde edilen PG çıktısı (Beyaz noktalar ZED, kırmızı noktalar VLP, gökkuşağı noktalar PG çıktısı)}
        \label{fig:pg_gaussian_citrus}
    \end{subfigure}

    \caption{Citrus Farm verisinde benzer frame’ler üzerinde Papoulis--Gerchberg (PG) çıktılarının karşılaştırılması.
    Brick-wall düşük geçiren filtre kullanıldığında düzeltme süreci daha belirgin artefaktlar (örn. küçük spike yapıları)
    üretebilirken, Gaussian filtre kullanımı daha yumuşak geçiş sağlayarak yakınsamanın görsel kararlılığını artırmıştır.
    Bu nedenle, ilgili deney setinde Gaussian filtre, Brick-wall filtreye kıyasla daha tutarlı bir PG iyileştirmesi sunmuştur.}
    \label{fig:pg_gaussian_vs_brickwall_citrus}
\end{figure}

Ayrıca, ZED ve VLP referans çerçeveleri arasında hesaplanan varyans (covariance) temelli maskeleme yaklaşımı, filtreleme süreciyle birlikte kullanıldığında, yüksek varyanslı LiDAR noktalarının PG güncellemesine dahil edilmemesi sayesinde daha kararlı sonuçlar elde edilmesini sağlamıştır. Bu yaklaşım, algoritmanın zorlayıcı ve fiziksel olarak anlamsız düzeltmeler yapması yerine, yalnızca güvenilir bölgelerde iyileştirme gerçekleştirmesine olanak tanımaktadır.

Sonuç olarak, bu çalışmada elde edilen bulgular, üç boyutlu veri füzyonunda filtre seçiminin yalnızca matematiksel değil, aynı zamanda fiziksel sahne yapısı ve sensör karakteristikleriyle birlikte değerlendirilmesi gerektiğini göstermektedir. Filtre tipi, cutoff frekansı ve maskeleme stratejilerinin birlikte ele alındığı bir yapı, Papoulis--Gerchberg tabanlı iyileştirme algoritmasının gerçek veri üzerindeki başarımını anlamlı ölçüde artırmaktadır.
