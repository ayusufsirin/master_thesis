\section{Experimental Results}
\label{sec:results}

This section presents the experimental evaluation of the proposed Papoulis--Gerchberg (PG) based LiDAR--stereo fusion pipeline under a two-parameter sweep:
the PG iteration count \(I\) and the temporal history size \(H\).
All trajectories are evaluated against RTK-GNSS odometry as ground truth using EVO, after alignment.
The stereo-only RTAB-Map output (ZED) is used as the baseline, while the PG pipeline produces a family of outputs indexed by \((I,H)\).

Because ZED does not have the parameters \((I,H)\), the comparison is formulated as a baseline-referenced improvement field over the sweep:
\[
\Delta(I,H)=M_{\text{ZED}}-M_{\text{PG}}(I,H),
\qquad
R(I,H)=\frac{M_{\text{PG}}(I,H)}{M_{\text{ZED}}}.
\]
In this formulation, \(\Delta>0\) indicates PG improvement over the ZED baseline, and \(R<1\) indicates a relative reduction in error.
Non-additive behavior between \(I\) and \(H\) is quantified with the superposition deviation \(\Gamma(I,H)\), as defined in
Section~\ref{sec:math_analysis_framework}.

\subsection{Global Trajectory Accuracy (APE): Strong and Consistent Improvement}
\label{subsec:results_ape}

Absolute Pose Error (APE) reflects global trajectory consistency with respect to RTK-GNSS and is therefore the primary indicator of long-horizon drift.
Figures~\ref{fig:results-ape-rmse} and \ref{fig:results-ape-mean} show the APE improvement field across the full parameter sweep.
Two complementary observations follow.

First, the improvement is not confined to a single isolated configuration: the \(\Delta\) and ratio maps reveal broad regions where PG outperforms the ZED baseline,
indicating that the proposed preprocessing/refinement is systematically beneficial rather than fragile.
Second, the improvement structure is strongly organized along the temporal history axis \(H\), supporting the interpretation that temporal aggregation stabilizes the LiDAR guidance signal.

% ----------------- FIGURE: APE RMSE -----------------
\begin{figure}[H]
\centering
\begin{subfigure}[t]{\linewidth}
\centering
\includegraphics[width=\linewidth]{./Experiments/out/metrics_aligned_se3/pg_ape_se3/rmse/slices_pg_ape_se3_rmse.png}
\caption{Slices}
\end{subfigure}
\vspace{0.6em}
\begin{subfigure}[t]{\linewidth}
\centering
\includegraphics[width=0.80\linewidth]{./Experiments/out/metrics_aligned_se3/pg_ape_se3/rmse/surface_delta_vs_zed_ape_se3_pg_ape_se3_rmse.png}
\caption{3D surface}
\end{subfigure}
\vspace{0.8em}
\begin{subfigure}[t]{0.49\linewidth}
\centering
\vspace{0pt}
\includegraphics[width=\linewidth]{./Experiments/out/metrics_aligned_se3/pg_ape_se3/rmse/heatmap_delta_vs_zed_ape_se3_pg_ape_se3_rmse.png}
\caption{$\Delta$ vs baseline}
\end{subfigure}
\hfill
\begin{subfigure}[t]{0.49\linewidth}
\centering
\vspace{0pt}
\includegraphics[width=\linewidth]{./Experiments/out/metrics_aligned_se3/pg_ape_se3/rmse/heatmap_ratio_vs_zed_ape_se3_pg_ape_se3_rmse.png}
\caption{Relative gain (\%)}
\end{subfigure}
\vspace{0.8em}
\begin{subfigure}[t]{0.49\linewidth}
\centering
\vspace{0pt}
\includegraphics[width=\linewidth]{./Experiments/out/metrics_aligned_se3/pg_ape_se3/rmse/heatmap_gamma_pg_ape_se3_rmse.png}
\caption{Interaction heatmap ($\Gamma$)}
\end{subfigure}
\hfill
\begin{subfigure}[t]{0.49\linewidth}
\centering
\vspace{2.0em}
\begin{tabular}{rrrrr}
\toprule
I \textbackslash H & 1 & 2 & 5 & 10 \\
\midrule
0 & 0.000 & 0.000 & 0.000 & 0.000 \\
10 & 0.000 & 0.563 & -2.252 & 1.260 \\
33 & 0.000 & 1.541 & -1.790 & 4.691 \\
100 & 0.000 & -1.416 & -4.340 & -2.020 \\
\bottomrule
\end{tabular}
\vspace{1.5em}
\caption{Interaction matrix ($\Gamma$)}
\end{subfigure}
\caption{APE RMSE improvement field over \((I,H)\) for PG relative to the ZED baseline (aligned SE(3)).}
\label{fig:results-ape-rmse}
\end{figure}

% ----------------- FIGURE: APE MEAN -----------------
\begin{figure}[H]
\centering
\begin{subfigure}[t]{\linewidth}
\centering
\includegraphics[width=\linewidth]{./Experiments/out/metrics_aligned_se3/pg_ape_se3/mean/slices_pg_ape_se3_mean.png}
\caption{Slices}
\end{subfigure}
\vspace{0.6em}
\begin{subfigure}[t]{\linewidth}
\centering
\includegraphics[width=0.80\linewidth]{./Experiments/out/metrics_aligned_se3/pg_ape_se3/mean/surface_delta_vs_zed_ape_se3_pg_ape_se3_mean.png}
\caption{3D surface}
\end{subfigure}
\vspace{0.8em}
\begin{subfigure}[t]{0.49\linewidth}
\centering
\vspace{0pt}
\includegraphics[width=\linewidth]{./Experiments/out/metrics_aligned_se3/pg_ape_se3/mean/heatmap_delta_vs_zed_ape_se3_pg_ape_se3_mean.png}
\caption{$\Delta$ vs baseline}
\end{subfigure}
\hfill
\begin{subfigure}[t]{0.49\linewidth}
\centering
\vspace{0pt}
\includegraphics[width=\linewidth]{./Experiments/out/metrics_aligned_se3/pg_ape_se3/mean/heatmap_ratio_vs_zed_ape_se3_pg_ape_se3_mean.png}
\caption{Relative gain (\%)}
\end{subfigure}
\vspace{0.8em}
\begin{subfigure}[t]{0.49\linewidth}
\centering
\vspace{0pt}
\includegraphics[width=\linewidth]{./Experiments/out/metrics_aligned_se3/pg_ape_se3/mean/heatmap_gamma_pg_ape_se3_mean.png}
\caption{Interaction heatmap ($\Gamma$)}
\end{subfigure}
\hfill
\begin{subfigure}[t]{0.49\linewidth}
\centering
\vspace{2.0em}
\begin{tabular}{rrrrr}
\toprule
I \textbackslash H & 1 & 2 & 5 & 10 \\
\midrule
0 & 0.000 & 0.000 & 0.000 & 0.000 \\
10 & 0.000 & 0.165 & -2.443 & 0.974 \\
33 & 0.000 & 1.177 & -1.971 & 3.958 \\
100 & 0.000 & -1.591 & -4.340 & -1.970 \\
\bottomrule
\end{tabular}
\vspace{1.5em}
\caption{Interaction matrix ($\Gamma$)}
\end{subfigure}
\caption{APE Mean improvement field over \((I,H)\) for PG relative to the ZED baseline (aligned SE(3)).}
\label{fig:results-ape-mean}
\end{figure}

\subsection{Local Consistency (RPE 1m): Strong Improvement at Short Scale}
\label{subsec:results_rpe_1m}

Relative Pose Error (RPE) at 1\,m emphasizes short-window motion consistency.
Figure~\ref{fig:results-rpe-1m-rmse} shows that PG yields consistent improvements across a wide region of the sweep,
which supports the claim that denser and more stable depth guidance improves local motion estimation robustness in the SLAM backend.

% ----------------- FIGURE: RPE 1m RMSE -----------------
\begin{figure}[H]
\centering
\begin{subfigure}[t]{\linewidth}
\centering
\includegraphics[width=\linewidth]{./Experiments/out/metrics_aligned_se3/pg_rpe_1m_se3/rmse/slices_pg_rpe_1m_se3_rmse.png}
\caption{Slices}
\end{subfigure}
\vspace{0.6em}
\begin{subfigure}[t]{\linewidth}
\centering
\includegraphics[width=0.80\linewidth]{./Experiments/out/metrics_aligned_se3/pg_rpe_1m_se3/rmse/surface_delta_vs_zed_rpe_1m_se3_pg_rpe_1m_se3_rmse.png}
\caption{3D surface}
\end{subfigure}
\vspace{0.8em}
\begin{subfigure}[t]{0.49\linewidth}
\centering
\vspace{0pt}
\includegraphics[width=\linewidth]{./Experiments/out/metrics_aligned_se3/pg_rpe_1m_se3/rmse/heatmap_delta_vs_zed_rpe_1m_se3_pg_rpe_1m_se3_rmse.png}
\caption{$\Delta$ vs baseline}
\end{subfigure}
\hfill
\begin{subfigure}[t]{0.49\linewidth}
\centering
\vspace{0pt}
\includegraphics[width=\linewidth]{./Experiments/out/metrics_aligned_se3/pg_rpe_1m_se3/rmse/heatmap_ratio_vs_zed_rpe_1m_se3_pg_rpe_1m_se3_rmse.png}
\caption{Relative gain (\%)}
\end{subfigure}
\vspace{0.8em}
\begin{subfigure}[t]{0.49\linewidth}
\centering
\vspace{0pt}
\includegraphics[width=\linewidth]{./Experiments/out/metrics_aligned_se3/pg_rpe_1m_se3/rmse/heatmap_gamma_pg_rpe_1m_se3_rmse.png}
\caption{Interaction heatmap ($\Gamma$)}
\end{subfigure}
\hfill
\begin{subfigure}[t]{0.49\linewidth}
\centering
\vspace{2.0em}
\begin{tabular}{rrrrr}
\toprule
I \textbackslash H & 1 & 2 & 5 & 10 \\
\midrule
0 & 0.000 & 0.000 & 0.000 & 0.000 \\
10 & 0.000 & 0.004 & 0.014 & 0.008 \\
33 & 0.000 & 0.007 & 0.013 & 0.010 \\
100 & 0.000 & 0.004 & 0.009 & 0.008 \\
\bottomrule
\end{tabular}
\vspace{1.5em}
\caption{Interaction matrix ($\Gamma$)}
\end{subfigure}
\caption{RPE 1m RMSE improvement field over \((I,H)\) for PG relative to the ZED baseline (aligned SE(3)).}
\label{fig:results-rpe-1m-rmse}
\end{figure}

\subsection{Mid-Scale Consistency (RPE 100m): Improvement Under SLAM Graph Constraints}
\label{subsec:results_rpe_100m}

RPE at 100\,m captures mid-scale drift behavior and is more reflective of accumulated backend constraints than short-scale RPE.
Figure~\ref{fig:results-rpe-100m-rmse} demonstrates that improvements remain achievable at this scale,
suggesting that PG contributes not only to local tracking quality but also to more stable mid-horizon trajectory consistency.

% ----------------- FIGURE: RPE 100m RMSE -----------------
\begin{figure}[H]
\centering
\begin{subfigure}[t]{\linewidth}
\centering
\includegraphics[width=\linewidth]{./Experiments/out/metrics_aligned_se3/pg_rpe_100m_se3/rmse/slices_pg_rpe_100m_se3_rmse.png}
\caption{Slices}
\end{subfigure}
\vspace{0.6em}
\begin{subfigure}[t]{\linewidth}
\centering
\includegraphics[width=0.80\linewidth]{./Experiments/out/metrics_aligned_se3/pg_rpe_100m_se3/rmse/surface_delta_vs_zed_rpe_100m_se3_pg_rpe_100m_se3_rmse.png}
\caption{3D surface}
\end{subfigure}
\vspace{0.8em}
\begin{subfigure}[t]{0.49\linewidth}
\centering
\vspace{0pt}
\includegraphics[width=\linewidth]{./Experiments/out/metrics_aligned_se3/pg_rpe_100m_se3/rmse/heatmap_delta_vs_zed_rpe_100m_se3_pg_rpe_100m_se3_rmse.png}
\caption{$\Delta$ vs baseline}
\end{subfigure}
\hfill
\begin{subfigure}[t]{0.49\linewidth}
\centering
\vspace{0pt}
\includegraphics[width=\linewidth]{./Experiments/out/metrics_aligned_se3/pg_rpe_100m_se3/rmse/heatmap_ratio_vs_zed_rpe_100m_se3_pg_rpe_100m_se3_rmse.png}
\caption{Relative gain (\%)}
\end{subfigure}
\vspace{0.8em}
\begin{subfigure}[t]{0.49\linewidth}
\centering
\vspace{0pt}
\includegraphics[width=\linewidth]{./Experiments/out/metrics_aligned_se3/pg_rpe_100m_se3/rmse/heatmap_gamma_pg_rpe_100m_se3_rmse.png}
\caption{Interaction heatmap ($\Gamma$)}
\end{subfigure}
\hfill
\begin{subfigure}[t]{0.49\linewidth}
\centering
\vspace{2.0em}
\begin{tabular}{rrrrr}
\toprule
I \textbackslash H & 1 & 2 & 5 & 10 \\
\midrule
0 & 0.000 & 0.000 & 0.000 & 0.000 \\
10 & 0.000 & 0.358 & 0.611 & 0.134 \\
33 & 0.000 & 0.073 & 0.011 & 0.024 \\
100 & 0.000 & 0.458 & 0.612 & 0.311 \\
\bottomrule
\end{tabular}
\vspace{1.5em}
\caption{Interaction matrix ($\Gamma$)}
\end{subfigure}
\caption{RPE 100m RMSE improvement field over \((I,H)\) for PG relative to the ZED baseline (aligned SE(3)).}
\label{fig:results-rpe-100m-rmse}
\end{figure}

\subsection{Interaction Analysis: Superposition Deviation \(\Gamma(I,H)\)}
\label{subsec:results_interaction}

The \(\Gamma\) heatmaps and tables embedded in Figures~\ref{fig:results-ape-rmse}--\ref{fig:results-rpe-100m-rmse}
allow testing whether the effects of \(I\) and \(H\) combine additively.
Overall, \(\Gamma(I,H)\) remains moderate across large regions, implying that iteration and history effects often behave near-additively.
Nevertheless, localized departures from \(\Gamma \approx 0\) indicate metric-dependent interactions:
negative \(\Gamma\) corresponds to synergistic improvement beyond additive expectation, whereas positive \(\Gamma\) corresponds to diminishing returns.

\subsection{Summary of Findings}
\label{subsec:results_summary}

The selected headline metrics (APE and short/mid-scale RPE) demonstrate that the PG-based fusion pipeline can consistently outperform the stereo-only baseline.
Improvements are structured over the sweep space, indicating systematic benefit rather than isolated tuning effects.
The observed patterns also support the interpretation that temporal history \(H\) is a primary stabilizing mechanism,
while iteration count \(I\) provides secondary refinements with interaction behavior that depends on the metric scale.
