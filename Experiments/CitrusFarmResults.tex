\section{Experimental Results}
\label{sec:results}

This section reports the experimental evaluation of the proposed Papoulis--Gerchberg (PG) based LiDAR--stereo fusion pipeline under a two-parameter sweep:
the PG iteration count \(I\) and the temporal history size \(H\).
Trajectory accuracy is evaluated against RTK-GNSS odometry as ground truth, using EVO-based trajectory alignment and metric extraction.
The stereo-only RTAB-Map output (ZED) is used as the baseline, while the PG pipeline produces a family of trajectories indexed by \((I,H)\).

To make the ZED-vs-PG comparison explicit and fair (since ZED has no \((I,H)\)), each metric is visualized as an improvement field over the parameter space:
\(\Delta(I,H)=M_{\text{ZED}}-M_{\text{PG}}(I,H)\) and the corresponding ratio \(M_{\text{PG}}(I,H)/M_{\text{ZED}}\).
We additionally quantify non-additive behavior through the superposition deviation \(\Gamma(I,H)\) (interaction term), as defined in
Section~\ref{sec:math_analysis_framework}.

\subsection{First-Order Effects: Dominant Role of History \(H\) and Saturation in Iterations \(I\)}
\label{subsec:first_order_effects}

Across the evaluated metrics, the slice plots and improvement surfaces indicate that increasing the temporal history size \(H\) yields the primary reduction in error,
while increasing the iteration count \(I\) provides secondary gains that tend to saturate beyond a limited range.
This behavior is consistent with the intuition that temporal aggregation (history) stabilizes and densifies the LiDAR guidance signal used by the PG refinement,
whereas additional iterations mainly refine an already-improved depth field with diminishing returns.

\subsection{Absolute Pose Error (APE): Global Consistency vs. Baseline}
\label{subsec:ape_results}

APE summarizes global trajectory deviation with respect to RTK-GNSS and is therefore the primary metric for assessing absolute drift and long-horizon consistency.
The PG parameter sweep produces a structured improvement region in \((I,H)\)-space relative to the ZED baseline.
In the delta and ratio heatmaps, broad areas of the sweep demonstrate consistent improvement, indicating that the proposed preprocessing/refinement pipeline
systematically benefits the SLAM backend rather than providing isolated or fragile gains.

% ----------------- FIGURE: APE RMSE (inline) -----------------
\begin{figure}[H]
\centering
\begin{subfigure}[t]{\linewidth}
\centering
\includegraphics[width=\linewidth]{./Experiments/out/metrics_aligned_se3/pg_ape_se3/rmse/slices_pg_ape_se3_rmse.png}
\caption{Slices}
\end{subfigure}

\vspace{0.8em}

\begin{subfigure}[t]{\linewidth}
\centering
\includegraphics[width=0.8\linewidth]{./Experiments/out/metrics_aligned_se3/pg_ape_se3/rmse/surface_delta_vs_zed_ape_se3_pg_ape_se3_rmse.png}
\caption{3D surface (vs. ZED baseline)}
\end{subfigure}

\vspace{0.8em}

\begin{subfigure}[t]{0.49\linewidth}
\centering
\vspace{0pt}
\includegraphics[width=\linewidth]{./Experiments/out/metrics_aligned_se3/pg_ape_se3/rmse/heatmap_delta_vs_zed_ape_se3_pg_ape_se3_rmse.png}
\caption{$\Delta$ vs baseline}
\end{subfigure}
\hfill
\begin{subfigure}[t]{0.49\linewidth}
\centering
\vspace{0pt}
\includegraphics[width=\linewidth]{./Experiments/out/metrics_aligned_se3/pg_ape_se3/rmse/heatmap_ratio_vs_zed_ape_se3_pg_ape_se3_rmse.png}
\caption{Relative gain (\%)}
\end{subfigure}

\vspace{0.8em}

\begin{subfigure}[t]{0.49\linewidth}
\centering
\vspace{0pt}
\includegraphics[width=\linewidth]{./Experiments/out/metrics_aligned_se3/pg_ape_se3/rmse/heatmap_gamma_pg_ape_se3_rmse.png}
\caption{Interaction heatmap ($\Gamma$)}
\end{subfigure}
\hfill
\begin{subfigure}[t]{0.49\linewidth}
\centering
\vspace{2.0em}
\begin{tabular}{rrrrr}
\toprule
I \textbackslash H & 1 & 2 & 5 & 10 \\
\midrule
0 & 0.000 & 0.000 & 0.000 & 0.000 \\
10 & 0.000 & 0.563 & -0.218 & -0.118 \\
33 & 0.000 & -0.926 & 0.212 & 0.158 \\
100 & 0.000 & -4.967 & 26.640 & 30.184 \\
\bottomrule
\end{tabular}
\vspace{1.5em}
\caption{Interaction matrix ($\Gamma$)}
\end{subfigure}

\caption{APE RMSE improvement field over \((I,H)\) for PG, relative to the ZED baseline (aligned SE(3)).}
\label{fig:results-ape-rmse}
\end{figure}

The APE RMSE field in \autoref{fig:results-ape-rmse} shows that improvements are primarily driven by increased history size \(H\), while large iteration counts may not yield proportional gains, supporting the saturation interpretation.

\subsection{Relative Pose Error (RPE): Local Consistency at Multiple Scales}
\label{subsec:rpe_results}

RPE evaluates local motion consistency and is reported at multiple spatial/temporal scales (e.g., 1\,m and 50\,m).
At short scales (1\,m), the PG sweep exhibits broad improvement across \((I,H)\), suggesting that refined depth guidance improves short-window motion estimation robustness.

% ----------------- FIGURE: RPE 1m RMSE (inline) -----------------
\begin{figure}[H]
\centering
\begin{subfigure}[t]{\linewidth}
\centering
\includegraphics[width=\linewidth]{./Experiments/out/metrics_aligned_se3/pg_rpe_1m_se3/rmse/slices_pg_rpe_1m_se3_rmse.png}
\caption{Slices}
\end{subfigure}

\vspace{0.8em}

\begin{subfigure}[t]{\linewidth}
\centering
\includegraphics[width=0.8\linewidth]{./Experiments/out/metrics_aligned_se3/pg_rpe_1m_se3/rmse/surface_delta_vs_zed_rpe_1m_se3_pg_rpe_1m_se3_rmse.png}
\caption{3D surface (vs. ZED baseline)}
\end{subfigure}

\vspace{0.8em}

\begin{subfigure}[t]{0.49\linewidth}
\centering
\vspace{0pt}
\includegraphics[width=\linewidth]{./Experiments/out/metrics_aligned_se3/pg_rpe_1m_se3/rmse/heatmap_delta_vs_zed_rpe_1m_se3_pg_rpe_1m_se3_rmse.png}
\caption{$\Delta$ vs baseline}
\end{subfigure}
\hfill
\begin{subfigure}[t]{0.49\linewidth}
\centering
\vspace{0pt}
\includegraphics[width=\linewidth]{./Experiments/out/metrics_aligned_se3/pg_rpe_1m_se3/rmse/heatmap_ratio_vs_zed_rpe_1m_se3_pg_rpe_1m_se3_rmse.png}
\caption{Relative gain (\%)}
\end{subfigure}

\vspace{0.8em}

\begin{subfigure}[t]{0.49\linewidth}
\centering
\vspace{0pt}
\includegraphics[width=\linewidth]{./Experiments/out/metrics_aligned_se3/pg_rpe_1m_se3/rmse/heatmap_gamma_pg_rpe_1m_se3_rmse.png}
\caption{Interaction heatmap ($\Gamma$)}
\end{subfigure}
\hfill
\begin{subfigure}[t]{0.49\linewidth}
\centering
\vspace{2.0em}
\begin{tabular}{rrrrr}
\toprule
I \textbackslash H & 1 & 2 & 5 & 10 \\
\midrule
0 & 0.000 & 0.000 & 0.000 & 0.000 \\
10 & 0.000 & -0.001 & -0.000 & 0.000 \\
33 & 0.000 & 0.000 & 0.001 & 0.001 \\
100 & 0.000 & 0.002 & 0.002 & 0.001 \\
\bottomrule
\end{tabular}
\vspace{1.5em}
\caption{Interaction matrix ($\Gamma$)}
\end{subfigure}

\caption{RPE 1m RMSE improvement field over \((I,H)\) for PG, relative to the ZED baseline (aligned SE(3)).}
\label{fig:results-rpe-1m-rmse}
\end{figure}

At larger scales (e.g., 50\,m), improvements remain possible but become more sensitive to backend SLAM graph behavior and route structure, which can reduce uniformity across the sweep.

% ----------------- FIGURE: RPE 50m RMSE (inline) -----------------
\begin{figure}[H]
\centering
\begin{subfigure}[t]{\linewidth}
\centering
\includegraphics[width=\linewidth]{./Experiments/out/metrics_aligned_se3/pg_rpe_50m_se3/rmse/slices_pg_rpe_50m_se3_rmse.png}
\caption{Slices}
\end{subfigure}

\vspace{0.8em}

\begin{subfigure}[t]{\linewidth}
\centering
\includegraphics[width=0.8\linewidth]{./Experiments/out/metrics_aligned_se3/pg_rpe_50m_se3/rmse/surface_delta_vs_zed_rpe_50m_se3_pg_rpe_50m_se3_rmse.png}
\caption{3D surface (vs. ZED baseline)}
\end{subfigure}

\vspace{0.8em}

\begin{subfigure}[t]{0.49\linewidth}
\centering
\vspace{0pt}
\includegraphics[width=\linewidth]{./Experiments/out/metrics_aligned_se3/pg_rpe_50m_se3/rmse/heatmap_delta_vs_zed_rpe_50m_se3_pg_rpe_50m_se3_rmse.png}
\caption{$\Delta$ vs baseline}
\end{subfigure}
\hfill
\begin{subfigure}[t]{0.49\linewidth}
\centering
\vspace{0pt}
\includegraphics[width=\linewidth]{./Experiments/out/metrics_aligned_se3/pg_rpe_50m_se3/rmse/heatmap_ratio_vs_zed_rpe_50m_se3_pg_rpe_50m_se3_rmse.png}
\caption{Relative gain (\%)}
\end{subfigure}

\vspace{0.8em}

\begin{subfigure}[t]{0.49\linewidth}
\centering
\vspace{0pt}
\includegraphics[width=\linewidth]{./Experiments/out/metrics_aligned_se3/pg_rpe_50m_se3/rmse/heatmap_gamma_pg_rpe_50m_se3_rmse.png}
\caption{Interaction heatmap ($\Gamma$)}
\end{subfigure}
\hfill
\begin{subfigure}[t]{0.49\linewidth}
\centering
\vspace{2.0em}
\begin{tabular}{rrrrr}
\toprule
I \textbackslash H & 1 & 2 & 5 & 10 \\
\midrule
0 & 0.000 & 0.000 & 0.000 & 0.000 \\
10 & 0.000 & -0.028 & -0.054 & -0.058 \\
33 & 0.000 & -0.173 & -0.174 & -0.154 \\
100 & 0.000 & -0.025 & 0.006 & -0.023 \\
\bottomrule
\end{tabular}
\vspace{1.5em}
\caption{Interaction matrix ($\Gamma$)}
\end{subfigure}

\caption{RPE 50m RMSE improvement field over \((I,H)\) for PG, relative to the ZED baseline (aligned SE(3)).}
\label{fig:results-rpe-50m-rmse}
\end{figure}

\subsection{Yaw Error: Orientation Sensitivity}
\label{subsec:yaw_results}

Yaw error is influenced by environmental observability and SLAM graph constraints in addition to preprocessing.
Nevertheless, the PG sweep provides measurable regions of improvement vs the baseline, particularly where \(H\) is sufficiently large to stabilize depth cues.

% ----------------- FIGURE: Yaw 50m RMSE (inline) -----------------
\begin{figure}[H]
\centering
\begin{subfigure}[t]{\linewidth}
\centering
\includegraphics[width=\linewidth]{./Experiments/out/metrics_aligned_se3/pg_yaw_50m_se3/rmse/slices_pg_yaw_50m_se3_rmse.png}
\caption{Slices}
\end{subfigure}

\vspace{0.8em}

\begin{subfigure}[t]{\linewidth}
\centering
\includegraphics[width=0.8\linewidth]{./Experiments/out/metrics_aligned_se3/pg_yaw_50m_se3/rmse/surface_delta_vs_zed_yaw_50m_se3_pg_yaw_50m_se3_rmse.png}
\caption{3D surface (vs. ZED baseline)}
\end{subfigure}

\vspace{0.8em}

\begin{subfigure}[t]{0.49\linewidth}
\centering
\vspace{0pt}
\includegraphics[width=\linewidth]{./Experiments/out/metrics_aligned_se3/pg_yaw_50m_se3/rmse/heatmap_delta_vs_zed_yaw_50m_se3_pg_yaw_50m_se3_rmse.png}
\caption{$\Delta$ vs baseline}
\end{subfigure}
\hfill
\begin{subfigure}[t]{0.49\linewidth}
\centering
\vspace{0pt}
\includegraphics[width=\linewidth]{./Experiments/out/metrics_aligned_se3/pg_yaw_50m_se3/rmse/heatmap_ratio_vs_zed_yaw_50m_se3_pg_yaw_50m_se3_rmse.png}
\caption{Relative gain (\%)}
\end{subfigure}

\vspace{0.8em}

\begin{subfigure}[t]{0.49\linewidth}
\centering
\vspace{0pt}
\includegraphics[width=\linewidth]{./Experiments/out/metrics_aligned_se3/pg_yaw_50m_se3/rmse/heatmap_gamma_pg_yaw_50m_se3_rmse.png}
\caption{Interaction heatmap ($\Gamma$)}
\end{subfigure}
\hfill
\begin{subfigure}[t]{0.49\linewidth}
\centering
\vspace{2.0em}
\begin{tabular}{rrrrr}
\toprule
I \textbackslash H & 1 & 2 & 5 & 10 \\
\midrule
0 & 0.000 & 0.000 & 0.000 & 0.000 \\
10 & 0.000 & -0.038 & -0.006 & -0.026 \\
33 & 0.000 & -0.106 & -0.129 & -0.117 \\
100 & 0.000 & 0.028 & 0.016 & 0.027 \\
\bottomrule
\end{tabular}
\vspace{1.5em}
\caption{Interaction matrix ($\Gamma$)}
\end{subfigure}

\caption{Yaw 50m RMSE improvement field over \((I,H)\) for PG, relative to the ZED baseline (aligned SE(3)).}
\label{fig:results-yaw-50m-rmse}
\end{figure}

\subsection{Interaction Analysis: Superposition Deviation \(\Gamma(I,H)\)}
\label{subsec:interaction_results}

To test whether the effects of \(I\) and \(H\) are approximately additive, we analyze the superposition deviation \(\Gamma(I,H)\)
(using both heatmaps and the embedded \(\Gamma\) tables shown in Figures~\ref{fig:results-ape-rmse}--\ref{fig:results-yaw-50m-rmse}).
In general, \(\Gamma(I,H)\) remains small over large regions of the sweep, indicating that iteration and history effects often combine near-additively.
However, localized regions show non-negligible interaction: negative \(\Gamma\) indicates synergy (combined improvement stronger than additive expectation),
while positive \(\Gamma\) indicates diminishing returns.
This behavior is metric-dependent: short-scale consistency metrics often exhibit weaker interaction, while longer-scale metrics may show stronger non-additive structure.

\subsection{Summary of Findings}
\label{subsec:results_summary}

The parameter sweep demonstrates that the proposed PG-based fusion pipeline can systematically reduce trajectory errors relative to the stereo-only baseline.
The dominant driver of improvement is the temporal history size \(H\), while the PG iteration count \(I\) provides secondary refinements with diminishing returns.
Interaction analysis shows that the parameters are frequently near-additive, with metric-dependent localized synergies or saturation effects.
In the next section, these findings are consolidated into configuration-level comparisons (e.g., best-performing \((I,H)\) per metric) and discussed in terms of practical deployment.
