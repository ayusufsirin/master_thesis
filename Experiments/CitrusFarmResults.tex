\section{Deneysel Sonuçlar | CitrusFarm Veri Seti}
\label{sec:results-citrusfarm}

Bu bölümde, önerilen Papoulis--Gerchberg (PG) tabanlı LiDAR--stereo füzyon hattının CitrusFarm Veriseti verileri ile deneysel değerlendirmesi iki parametreli bir tarama (sweep) altında sunulmaktadır: PG iterasyon sayısı \(I\) ve zamansal geçmiş (history) boyutu \(H\).
Tüm yörüngeler, hizalama sonrasında EVO kullanılarak RTK-GNSS odometrisi (yer gerçeği) ile karşılaştırılmıştır.
Stereo-only RTAB-Map çıktısı (ZED) taban çizgisi (baseline) olarak alınmış; PG hattı ise \((I,H)\) ile indekslenen
bir çıktı ailesi üretmiştir.

ZED hattında \((I,H)\) parametreleri bulunmadığından, karşılaştırma tarama uzayı üzerinde taban-çizgisine göre
tanımlanmış bir iyileşme alanı olarak formüle edilmiştir:
\[
    \Delta(I,H)=M_{\text{ZED}}-M_{\text{PG}}(I,H),
    \qquad
    R(I,H)=\frac{M_{\text{PG}}(I,H)}{M_{\text{ZED}}}.
\]
Bu formülasyonda \(\Delta>0\) PG’nin ZED taban çizgisine göre iyileşme sağladığını, \(R<1\) ise hatada göreli bir
azalma olduğunu gösterir.
\(I\) ve \(H\) arasındaki toplamsal olmayan (non-additive) davranış ise
Bölüm~\ref{sec:math_analysis_framework}’te tanımlandığı şekilde süperpozisyon sapması \(\Gamma(I,H)\) ile
nicemlenmiştir.

\paragraph*{Taban-çizgisine göre haritalar (\(\Delta\) ve göreli kazanç) nasıl okunmalı?}
ZED (stereo-only) hattında tarama parametreleri olmadığından tüm karşılaştırmalar sabit bir taban metrik değeri
\(M_{\mathrm{ZED}}\) referans alınarak ifade edilir.
Her bir \((I,H)\) konfigürasyonu için PG metriği \(M_{\mathrm{PG}}(I,H)\) hesaplanır ve iyileşme iki tamamlayıcı
biçimde görselleştirilir:

\begin{align}
\Delta(I,H) &= M_{\mathrm{ZED}} - M_{\mathrm{PG}}(I,H), \\
G(I,H) &= 100 \cdot \frac{M_{\mathrm{ZED}} - M_{\mathrm{PG}}(I,H)}{M_{\mathrm{ZED}}}
      = 100 \cdot \frac{\Delta(I,H)}{M_{\mathrm{ZED}}}.
\end{align}

\noindent Burada \(\Delta(I,H)\), metriğin \emph{doğal birimlerinde} (örn. APE/RPE için metre, yaw için derece)
raporlanırken; \(G(I,H)\) farklı metrikler ve ölçekler arasında doğrudan karşılaştırmaya olanak veren \emph{birimsiz
normalize yüzde} ölçüsüdür.

\begin{itemize}
    \item \textbf{\(\Delta>0\)}: PG’nin taban çizgisine göre iyileştirdiğini (hata azalması) gösterir; çünkü
    \(M_{\mathrm{PG}}(I,H) < M_{\mathrm{ZED}}\).
    \item \textbf{\(\Delta=0\)}: taban çizgisine göre değişim olmadığını gösterir.
    \item \textbf{\(\Delta<0\)}: taban çizgisine göre bozulmayı (hata artışı) gösterir; çünkü
    \(M_{\mathrm{PG}}(I,H) > M_{\mathrm{ZED}}\).
\end{itemize}

\noindent \textbf{Göreli kazanç (\%)} haritası \(G(I,H)\), farklı metriklerin büyüklük aralıkları farklı olduğunda
özellikle faydalıdır.
Örneğin 1\,m’lik mutlak iyileşme, kısa ölçekli bir RPE için çok anlamlıyken büyük ölçekli bir RPE için daha mütevazı
kalabilir.
\(M_{\mathrm{ZED}}\) ile normalizasyon yaparak kazanç haritası, iyileşmenin \emph{oransal} olarak en anlamlı olduğu
bölgeleri vurgular.

\paragraph*{Pratik yorum.}
Isı haritalarında her bir hücre bir \((I,H)\) konfigürasyonuna karşılık gelir.
\(\Delta>0\) (veya \(G>0\)) koşulunu sağlayan geniş ve bitişik bölgeler, iyileşmenin tek bir ince ayarlı parametre
seçimine bağlı kalmayıp \emph{sistematik ve sağlam} olduğunu gösterir.
Buna karşılık, etrafı negatif bölgelerle çevrili izole pozitif hücreler kırılgan (fragile) bir ayar/tuning
davranışına işaret eder.

\paragraph*{Ölçek ve kararlılığa ilişkin notlar.}
\(G(I,H)\) pratik bir normalizasyon sağlasa da belirli bir metrik için \(M_{\mathrm{ZED}}\) çok küçük olduğunda
küçük değişimleri görsel olarak büyütebilir.
Bu nedenle kazanç haritalarının yorumu, metriğin birimlerini koruyan \(\Delta(I,H)\) ile birlikte yapılmalıdır.
Bu çalışmada, yalnızca normalizasyondan kaynaklı yanıltıcı çıkarımları önlemek için her iki harita birlikte
raporlanmıştır.

\subsection{Küresel Yörünge Doğruluğu (APE): Güçlü ve Tutarlı İyileşme}
\label{subsec:results_ape}

Mutlak Poz Hatası (APE), RTK-GNSS’e göre küresel yörünge tutarlılığını yansıtır ve bu nedenle uzun ufuk (long-horizon)
sürüklenmenin (drift) temel göstergesidir.
Şekiller~\ref{fig:results-ape-rmse-1}, ~\ref{fig:results-ape-rmse-2}, ~\ref{fig:results-ape-mean-1} ve
~\ref{fig:results-ape-mean-2}, tüm parametre taraması boyunca APE iyileşme alanını göstermektedir.
Buna ilişkin iki tamamlayıcı gözlem elde edilmiştir.

İlk olarak, iyileşme tek bir izole konfigürasyonla sınırlı değildir: \(\Delta\) ve oran haritaları, PG’nin ZED taban
çizgisini geçtiği geniş bölgeler ortaya koyar; bu da önerilen ön-işleme/iyileştirme yaklaşımının kırılgan bir ayara
dayanmak yerine sistematik biçimde faydalı olduğunu gösterir.
İkinci olarak, iyileşme yapısı zamansal geçmiş ekseni \(H\) boyunca belirgin biçimde organize olmaktadır; bu durum,
zamansal biriktirmenin LiDAR kılavuz sinyalini kararlı hâle getirdiği yorumunu destekler.

% ----------------- FIGURE: APE RMSE -----------------
\begin{figure}[H]
    \centering
    \begin{subfigure}[t]{\linewidth}
        \centering
\includegraphics[width=\linewidth]{./Experiments/out/metrics_aligned_se3/pg_ape_se3/rmse/slices_pg_ape_se3_rmse.png}
        \caption{Dilimler}
    \end{subfigure}
    \vspace{0.6em}
    \begin{subfigure}[t]{\linewidth}
        \centering
\includegraphics[width=0.50\linewidth]{./Experiments/out/metrics_aligned_se3/pg_ape_se3/rmse/surface_delta_vs_zed_ape_se3_pg_ape_se3_rmse.png}
        \caption{3B yüzey}
    \end{subfigure}
    \caption{ZED taban çizgisine göre PG için \((I,H)\) üzerinde APE RMSE iyileşme alanı (SE(3) hizalı)
    (Dilimler ve Yüzey).}
    \label{fig:results-ape-rmse-1}
\end{figure}
\begin{figure}[H]
    \centering
    \begin{subfigure}[t]{0.49\linewidth}
        \centering
        \vspace{0pt}
\includegraphics[width=\linewidth]{./Experiments/out/metrics_aligned_se3/pg_ape_se3/rmse/heatmap_delta_vs_zed_ape_se3_pg_ape_se3_rmse.png}
        \caption{\(\Delta\) (taban çizgisine göre)}
    \end{subfigure}
    \hfill
    \begin{subfigure}[t]{0.49\linewidth}
        \centering
        \vspace{0pt}
\includegraphics[width=\linewidth]{./Experiments/out/metrics_aligned_se3/pg_ape_se3/rmse/heatmap_ratio_vs_zed_ape_se3_pg_ape_se3_rmse.png}
        \caption{Göreli kazanç (\%)}
    \end{subfigure}
    \vspace{0.8em}
    \begin{subfigure}[t]{0.49\linewidth}
        \centering
        \vspace{0pt}
\includegraphics[width=\linewidth]{./Experiments/out/metrics_aligned_se3/pg_ape_se3/rmse/heatmap_gamma_pg_ape_se3_rmse.png}
        \caption{Etkileşim ısı haritası (\(\Gamma\))}
    \end{subfigure}
    \hfill
    \begin{subfigure}[t]{0.49\linewidth}
        \centering
        \vspace{2.0em}
        \begin{tabular}{rrrrr}
            \toprule
            I \textbackslash H & 1     & 2     & 5     & 10    \\
            \midrule
            0                  & 0.000 & 0.000 & 0.000 & 0.000 \\
            10                 & 0.000 & 8.663 & 16.041 & 6.786 \\
            33                 & 0.000 & 9.286 & 15.424 & 7.796 \\
            100                & 0.000 & 9.001 & 15.780 & 7.600 \\
            \bottomrule
        \end{tabular}
        \vspace{1.5em}
        \caption{Etkileşim matrisi (\(\Gamma\))}
    \end{subfigure}
    \caption{ZED taban çizgisine göre PG için \((I,H)\) üzerinde APE RMSE iyileşme alanı (SE(3) hizalı)
    (Metrik karşılaştırmaları).}
    \emph{Pozitif \(\Delta\) ve pozitif kazanç (\%) ZED taban çizgisine göre daha iyi performansı (daha düşük hata)
    ifade eder.}
    \label{fig:results-ape-rmse-2}
\end{figure}

% ----------------- FIGURE: APE MEAN -----------------
\begin{figure}[H]
    \centering
    \begin{subfigure}[t]{\linewidth}
        \centering
\includegraphics[width=\linewidth]{./Experiments/out/metrics_aligned_se3/pg_ape_se3/mean/slices_pg_ape_se3_mean.png}
        \caption{Dilimler}
    \end{subfigure}
    \vspace{0.6em}
    \begin{subfigure}[t]{\linewidth}
        \centering
\includegraphics[width=0.50\linewidth]{./Experiments/out/metrics_aligned_se3/pg_ape_se3/mean/surface_delta_vs_zed_ape_se3_pg_ape_se3_mean.png}
        \caption{3B yüzey}
    \end{subfigure}
    \caption{ZED taban çizgisine göre PG için \((I,H)\) üzerinde APE ortalama (Mean) iyileşme alanı (SE(3) hizalı)
    (Dilimler ve Yüzey).}
    \label{fig:results-ape-mean-1}
\end{figure}
\begin{figure}[H]
    \centering
    \begin{subfigure}[t]{0.49\linewidth}
        \centering
        \vspace{0pt}
\includegraphics[width=\linewidth]{./Experiments/out/metrics_aligned_se3/pg_ape_se3/mean/heatmap_delta_vs_zed_ape_se3_pg_ape_se3_mean.png}
        \caption{\(\Delta\) (taban çizgisine göre)}
    \end{subfigure}
    \hfill
    \begin{subfigure}[t]{0.49\linewidth}
        \centering
        \vspace{0pt}
\includegraphics[width=\linewidth]{./Experiments/out/metrics_aligned_se3/pg_ape_se3/mean/heatmap_ratio_vs_zed_ape_se3_pg_ape_se3_mean.png}
        \caption{Göreli kazanç (\%)}
    \end{subfigure}
    \vspace{0.8em}
    \begin{subfigure}[t]{0.49\linewidth}
        \centering
        \vspace{0pt}
\includegraphics[width=\linewidth]{./Experiments/out/metrics_aligned_se3/pg_ape_se3/mean/heatmap_gamma_pg_ape_se3_mean.png}
        \caption{Etkileşim ısı haritası (\(\Gamma\))}
    \end{subfigure}
    \hfill
    \begin{subfigure}[t]{0.49\linewidth}
        \centering
        \vspace{2.0em}
        \begin{tabular}{rrrrr}
            \toprule
            I \textbackslash H & 1     & 2     & 5     & 10    \\
            \midrule
            0                  & 0.000 & 0.000 & 0.000 & 0.000 \\
            10                 & 0.000 & 7.300 & 13.171 & 6.054 \\
            33                 & 0.000 & 7.800 & 12.598 & 6.885 \\
            100                & 0.000 & 7.503 & 12.870 & 6.702 \\
            \bottomrule
        \end{tabular}
        \vspace{1.5em}
        \caption{Etkileşim matrisi (\(\Gamma\))}
    \end{subfigure}
    \caption{ZED taban çizgisine göre PG için \((I,H)\) üzerinde APE ortalama (Mean) iyileşme alanı (SE(3) hizalı)
    (Metrik karşılaştırmaları).}
    \emph{Pozitif \(\Delta\) ve pozitif kazanç (\%) ZED taban çizgisine göre daha iyi performansı (daha düşük hata)
    ifade eder.}
    \label{fig:results-ape-mean-2}
\end{figure}

\subsection{Kısa Ölçekli Tutarlılık (RPE 1m): Özellikle \(H\)’ye Duyarlı}
\label{subsec:results_rpe_1m}

RPE 1\,m, yerel (local) tutarlılığı ölçer ve özellikle kısa zaman/mesafe aralıklarında gürültü ve izleme (tracking)
kalitesine duyarlıdır.
Şekiller~\ref{fig:results-rpe-1m-rmse-1} ve ~\ref{fig:results-rpe-1m-rmse-2}, kısa ölçekli RPE’de iyileşmenin
büyük ölçüde \(H\) tarafından sürüklendiğini göstermektedir:
daha büyük tarihçe değerleri daha yoğun kümülatif LiDAR desteği sağlayarak PG’nin stereo eşleştirmeye uyguladığı
geometrik kılavuzu güçlendirir.

% ----------------- FIGURE: RPE 1m RMSE -----------------
\begin{figure}[H]
    \centering
    \begin{subfigure}[t]{\linewidth}
        \centering
\includegraphics[width=\linewidth]{./Experiments/out/metrics_aligned_se3/pg_rpe_1m_se3/rmse/slices_pg_rpe_1m_se3_rmse.png}
        \caption{Dilimler}
    \end{subfigure}
    \vspace{0.6em}
    \begin{subfigure}[t]{\linewidth}
        \centering
\includegraphics[width=0.80\linewidth]{./Experiments/out/metrics_aligned_se3/pg_rpe_1m_se3/rmse/surface_delta_vs_zed_rpe_1m_se3_pg_rpe_1m_se3_rmse.png}
        \caption{3B yüzey}
    \end{subfigure}
    \caption{ZED taban çizgisine göre PG için \((I,H)\) üzerinde RPE 1m RMSE iyileşme alanı (SE(3) hizalı)
    (Dilimler ve Yüzeyler).}
    \label{fig:results-rpe-1m-rmse-1}
\end{figure}
\begin{figure}[H]
    \centering
    \begin{subfigure}[t]{0.49\linewidth}
        \centering
        \vspace{0pt}
\includegraphics[width=\linewidth]{./Experiments/out/metrics_aligned_se3/pg_rpe_1m_se3/rmse/heatmap_delta_vs_zed_rpe_1m_se3_pg_rpe_1m_se3_rmse.png}
        \caption{\(\Delta\) (taban çizgisine göre)}
    \end{subfigure}
    \hfill
    \begin{subfigure}[t]{0.49\linewidth}
        \centering
        \vspace{0pt}
\includegraphics[width=\linewidth]{./Experiments/out/metrics_aligned_se3/pg_rpe_1m_se3/rmse/heatmap_ratio_vs_zed_rpe_1m_se3_pg_rpe_1m_se3_rmse.png}
        \caption{Göreli kazanç (\%)}
    \end{subfigure}
    \vspace{0.8em}
    \begin{subfigure}[t]{0.49\linewidth}
        \centering
        \vspace{0pt}
\includegraphics[width=\linewidth]{./Experiments/out/metrics_aligned_se3/pg_rpe_1m_se3/rmse/heatmap_gamma_pg_rpe_1m_se3_rmse.png}
        \caption{Etkileşim ısı haritası (\(\Gamma\))}
    \end{subfigure}
    \hfill
    \begin{subfigure}[t]{0.49\linewidth}
        \centering
        \vspace{2.0em}
        \begin{tabular}{rrrrr}
            \toprule
            I \textbackslash H & 1     & 2     & 5     & 10    \\
            \midrule
            0                  & 0.000 & 0.000 & 0.000 & 0.000 \\
            10                 & 0.000 & 0.004 & 0.014 & 0.008 \\
            33                 & 0.000 & 0.007 & 0.013 & 0.010 \\
            100                & 0.000 & 0.004 & 0.009 & 0.008 \\
            \bottomrule
        \end{tabular}
        \vspace{1.5em}
        \caption{Etkileşim matrisi (\(\Gamma\))}
    \end{subfigure}
    \caption{ZED taban çizgisine göre PG için \((I,H)\) üzerinde RPE 1m RMSE iyileşme alanı (SE(3) hizalı)
    (Metrik karşılaştırmaları) (Metrik karşılaştırmaları).}
    \emph{Pozitif \(\Delta\) ve pozitif kazanç (\%) ZED taban çizgisine göre daha iyi performansı (daha düşük hata)
    ifade eder.}
    \label{fig:results-rpe-1m-rmse-2}
\end{figure}

\subsection{Orta Ölçekli Tutarlılık (RPE 100m): SLAM Grafik Kısıtları Altında İyileşme}
\label{subsec:results_rpe_100m}

100\,m’de RPE, orta ölçekli sürüklenme davranışını yakalar ve kısa ölçekli RPE’ye kıyasla birikimli (accumulated)
backend kısıtlarını daha fazla yansıtır.
Şekiller~\ref{fig:results-rpe-100m-rmse-1} ve ~\ref{fig:results-rpe-100m-rmse-2}, bu ölçekte de iyileşmenin mümkün
olduğunu göstermektedir; bu da PG’nin yalnızca yerel izleme kalitesine değil, aynı zamanda orta ufukta daha kararlı bir
yörünge tutarlılığına katkı verdiğini düşündürür.

% ----------------- FIGURE: RPE 100m RMSE -----------------
\begin{figure}[H]
    \centering
    \begin{subfigure}[t]{\linewidth}
        \centering
\includegraphics[width=\linewidth]{./Experiments/out/metrics_aligned_se3/pg_rpe_100m_se3/rmse/slices_pg_rpe_100m_se3_rmse.png}
        \caption{Dilimler}
    \end{subfigure}
    \vspace{0.6em}
    \begin{subfigure}[t]{\linewidth}
        \centering
\includegraphics[width=0.60\linewidth]{./Experiments/out/metrics_aligned_se3/pg_rpe_100m_se3/rmse/surface_delta_vs_zed_rpe_100m_se3_pg_rpe_100m_se3_rmse.png}
        \caption{3B yüzey}
    \end{subfigure}
    \caption{ZED taban çizgisine göre PG için \((I,H)\) üzerinde RPE 100m RMSE iyileşme alanı (SE(3) hizalı)
    (Dilimler ve Yüzeyler).}
    \label{fig:results-rpe-100m-rmse-1}
\end{figure}
\begin{figure}[H]
    \centering
    \begin{subfigure}[t]{0.49\linewidth}
        \centering
        \vspace{0pt}
\includegraphics[width=\linewidth]{./Experiments/out/metrics_aligned_se3/pg_rpe_100m_se3/rmse/heatmap_delta_vs_zed_rpe_100m_se3_pg_rpe_100m_se3_rmse.png}
        \caption{\(\Delta\) (taban çizgisine göre)}
    \end{subfigure}
    \hfill
    \begin{subfigure}[t]{0.49\linewidth}
        \centering
        \vspace{0pt}
\includegraphics[width=\linewidth]{./Experiments/out/metrics_aligned_se3/pg_rpe_100m_se3/rmse/heatmap_ratio_vs_zed_rpe_100m_se3_pg_rpe_100m_se3_rmse.png}
        \caption{Göreli kazanç (\%)}
    \end{subfigure}
    \vspace{0.8em}
    \begin{subfigure}[t]{0.49\linewidth}
        \centering
        \vspace{0pt}
\includegraphics[width=\linewidth]{./Experiments/out/metrics_aligned_se3/pg_rpe_100m_se3/rmse/heatmap_gamma_pg_rpe_100m_se3_rmse.png}
        \caption{Etkileşim ısı haritası (\(\Gamma\))}
    \end{subfigure}
    \hfill
    \begin{subfigure}[t]{0.49\linewidth}
        \centering
        \vspace{2.0em}
        \begin{tabular}{rrrrr}
            \toprule
            I \textbackslash H & 1     & 2     & 5     & 10    \\
            \midrule
            0                  & 0.000 & 0.000 & 0.000 & 0.000 \\
            10                 & 0.000 & 0.358 & 0.611 & 0.134 \\
            33                 & 0.000 & 0.073 & 0.011 & 0.024 \\
            100                & 0.000 & 0.458 & 0.612 & 0.311 \\
            \bottomrule
        \end{tabular}
        \vspace{1.5em}
        \caption{Etkileşim matrisi (\(\Gamma\))}
    \end{subfigure}
    \caption{ZED taban çizgisine göre PG için \((I,H)\) üzerinde RPE 100m RMSE iyileşme alanı (SE(3) hizalı)
    (Metrik karşılaştırmaları).}
    \emph{Pozitif \(\Delta\) ve pozitif kazanç (\%) ZED taban çizgisine göre daha iyi performansı (daha düşük hata)
    ifade eder.}
    \label{fig:results-rpe-100m-rmse-2}
\end{figure}

\subsection{Etkileşim Analizi: Süperpozisyon Sapması \(\Gamma(I,H)\)}
\label{subsec:results_interaction}

Şekil~\ref{fig:results-ape-rmse-2}, ~\ref{fig:results-rpe-1m-rmse-2} ve ~\ref{fig:results-rpe-100m-rmse-2} içinde
gömülü olarak verilen \(\Gamma\) ısı haritaları ve tabloları, \(I\) ve \(H\) etkilerinin toplamsal (additive) biçimde
birleşip birleşmediğini test etmeye imkân tanır.
Genel olarak \(\Gamma(I,H)\) geniş bölgelerde orta düzeyde kalmaktadır; bu durum iterasyon ve geçmiş etkilerinin çoğu
zaman toplamsala yakın davrandığını ima eder.
Bununla birlikte \(\Gamma \approx 0\)’dan yerel sapmalar, metriğe bağlı etkileşimleri işaret eder:
negatif \(\Gamma\), toplamsal beklentinin ötesinde sinerjik bir iyileşmeye; pozitif \(\Gamma\) ise azalan getiriye
(diminishing returns) karşılık gelir.

\subsection{Bulguların Özeti}
\label{subsec:results_summary}

Seçilen başlık metrikleri (APE ve kısa/orta ölçekli RPE), PG tabanlı füzyon hattının stereo-only taban çizgisini
tutarlı biçimde geçebildiğini göstermektedir.
İyileşme, tarama uzayında yapılandırılmış (structured) bir biçimde görülmekte; bu da izole ayar etkilerinden ziyade
sistematik bir faydaya işaret etmektedir.
Gözlenen örüntüler, zamansal geçmiş \(H\)’nin birincil bir kararlılık mekanizması olduğunu; iterasyon sayısı \(I\)’nin
ise metriğin ölçeğine bağlı olarak etkileşim davranışı sergileyen ikincil düzeltmeler sağladığını desteklemektedir.

\paragraph*{Sabit iterasyon taramasından çıkarım (\(I=100\)).}
PG iterasyon sayısı \(I=100\) sabitlenerek, Şekil~\ref{fig:pg_iter_100_xy_raw} zamansal geçmiş boyutu \(H\)’nin
taranmasının elde edilen yörünge kestirimi üzerindeki etkisini izole eder.
\(H\) arttıkça füzyon aşaması daha yoğun bir kümülatif LiDAR nokta desteğinden faydalanır; bu da PG yönlendirmeli
iyileştirme sırasında kullanılan geometrik öncülü güçlendirir.
Bu artan geometrik koşullama (conditioning), yörünge bindirmelerinde (overlay) gözlemlenmektedir:
daha büyük \(H\) değerlerine sahip konfigürasyonlar RTK-GNSS referansını daha tutarlı izleme eğiliminde olup,
düşük tarihçe ayarlarına kıyasla daha az sürüklenme sergiler.
Özellikle dikkat çekici nitel (qualitative) bir ipucu, stereo-only (ZED) SLAM yörüngesinde gözlenen belirgin ölçek
büzülmesidir; bu yörünge, yüksek tarihçeli PG ayarına (örn. \(H=10\)) göre daha kısa görünmektedir.
Bu durum, yeterli zamansal bağlam sağlandığında LiDAR destekli iyileştirmenin ölçek tutarsızlığını azalttığını ve
birikimli hareket kestirimini daha kararlı hâle getirdiğini düşündürür.

\begin{figure}[H]
    \centering
    \begin{subfigure}[t]{0.49\linewidth}
        \centering
\includegraphics[width=\textwidth]{./Experiments/traj_plots/by_iter/iter_100_histories_vs_gt_xy_raw_rpy.png}
        \caption{Tüm tarihçeler için iterasyon 100’de PG yörüngeleri (mod: xy, hizalama: raw, görünüm: rpy).}
        \label{fig:pg_iter_100_xy_raw_rpy}
    \end{subfigure}
    \begin{subfigure}[t]{0.49\linewidth}
        \centering
\includegraphics[width=\textwidth]{./Experiments/traj_plots/by_iter/iter_100_histories_vs_gt_xy_raw_xyz.png}
        \caption{Tüm tarihçeler için iterasyon 100’de PG yörüngeleri (mod: xy, hizalama: raw, görünüm: xyz).}
        \label{fig:pg_iter_100_xy_raw_xyz}
    \end{subfigure}
    \begin{subfigure}[t]{0.49\linewidth}
        \centering
\includegraphics[width=\textwidth]{./Experiments/traj_plots/by_iter/iter_100_histories_vs_gt_xy_raw_trajectories.png}
        \caption{Tüm tarihçeler için iterasyon 100’de PG yörüngeleri (mod: xy, hizalama: raw, görünüm: trajectories).}
        \label{fig:pg_iter_100_xy_raw_trajectories}
    \end{subfigure}

    \caption{Tüm tarihçeler için PG iterasyon 100 yörünge karşılaştırması (mod: xy, hizalama: raw).}
    \emph{SLAM odometrisinde gözlenen tutarlı saat yönünün tersine (CCW) sapmanın, hem RTAB-Map arka ucu hem de LiDAR
    upsampling sırasında kullanılan CCW-yanlı (CCW-biased) tekerlek odometrisi girdisinden kaynaklandığı
    düşünülmektedir.}
    \label{fig:pg_iter_100_xy_raw}
\end{figure}

\paragraph*{\(z\) ekseni ve yönelim görünümlerinin yorumu (neden
alt-şekiller~\ref{fig:pg_iter_100_xy_raw_rpy} ve~\ref{fig:pg_iter_100_xy_raw_xyz} nicel değil, nitel kalır?).}
Alt-şekiller~\ref{fig:pg_iter_100_xy_raw_rpy} ve~\ref{fig:pg_iter_100_xy_raw_xyz}, yörüngeyi ayrıca 3B ve RPY
odaklı bir görünümde görselleştirir; ancak bu grafikler yer gerçeğine göre katı bir doğruluk karşılaştırması olarak
değil, \emph{nitel tanılama} (qualitative diagnostics) amaçlı yorumlanmalıdır.
RTK-GNSS referansı güvenilir küresel konum sağlar; fakat karşılaştırılabilir doğrulukta tam 6-DoF duruş (attitude)
yer gerçeğini doğrudan sunmaz (özellikle roll ve pitch; ayrıca GNSS kurulumuna bağlı olarak yaw).
Bu nedenle \(z\) ve roll/pitch sapmaları, düzlemsel konumla aynı biçimde ve aynı referansa karşı puanlanamaz.
Buna rağmen bu görünümler kasten korunmuştur; çünkü konfigürasyona bağlı hata türlerini (örn. dikey salınımlar,
yönelim kararsızlığı veya tutarsız 3B hareket) görünür kılar ve bunlar çoğu zaman öteleme (translational) sürüklenmesi
ile ilişkili olabilir.
Nicel değerlendirme için çalışma, RTK-GNSS referansı ile doğrudan karşılaştırılabilen öteleme bileşenlerinden türetilen
konum tabanlı yörünge hata metriklerine (APE/RPE) dayanmaktadır.
