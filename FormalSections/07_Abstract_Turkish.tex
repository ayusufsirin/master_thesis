\btypeout{Abstract TR}
\pagestyle{fancy} 
\fancyhf{}
\fancyfoot[C]{\footnotesize\thepage}
\renewcommand{\headrulewidth}{0pt}      
\renewcommand{\footrulewidth}{0pt}
\setcounter{page}{1}
\begin{center}
    \setlength{\parskip}{0pt}
    {\fontsize{14}{16.8}{\textbf{ÖZET}}}\\
    ~\\
    ~\\
    {\fontsize{14}{16.8}\bf \TURKISHTITLE} \\
    ~\\
    ~\\
    {\fontsize{14}{16.8}\bf \author \par}
    {\fontsize{14}{16.8}\bf Yüksek Lisans}, {\fontsize{14}{16.8}\bf \department \par} 
    {\fontsize{14}{16.8}\bf Danışman: \supname}\\
    %{\fontsize{14}{16.8}\bf Eş Danışman: Doç. Dr. Adı Soyadı }\\
    {\fontsize{14}{16.8}\bf Ocak 2026, \ref{TotPages} sayfa} \\
    ~\\
    ~\\
\end{center}


{\addtocontents{toc}{\vspace{1em}}
\addtotoc{ÖZET}} 

Otonom sistemlerin çevreyi doğru algılayabilmesi ve güvenilir navigasyon kararları verebilmesi için derinlik bilgisinin yüksek doğrulukta ve yeterli çözünürlükte elde edilmesi kritik bir gereksinimdir. Çevreyi algılamada kullanılan stereo kameralar, yoğun (piksel seviyesinde) derinlik haritaları üretebilmelerine rağmen gürültüye, doku yetersizliğine ve nesne sınırlarında tutarsızlıklara açıktır. LiDAR sensörleri ise daha yüksek doğrulukta fakat seyrek ölçümler sağlar. Bu iki algılayıcının tamamlayıcı yapısı sensör füzyonu için güçlü bir fırsat sunarken, farklı çözünürlükte ve farklı örnekleme dağılımındaki verilerin ortak bir geometri üzerinde birleştirilmesi teknik olarak zorlu bir problemdir. Son yıllarda derin öğrenme tabanlı Stereo-LiDAR füzyon yöntemleri yaygınlaşsa da bu yöntemlerin yüksek hesaplama maliyeti ve etiketli veri gereksinimi, derin öğrenme yöntemlerinin gömülü ve gerçek zamanlı otonom platformlarda uygulanabilirliği sınırlamaktadır.

Bu tezde, eğitim gerektirmeyen ve deterministik bir yapıda çalışan Papoulis–Gerchberg (PG) algoritmasına dayalı yinelemeli bir Stereo–LiDAR füzyon yöntemi önerilmektedir. Önerilen yöntemin işlem maliyeti de derin öğrenme yöntemlerine göre oldukça düşüktür. Literatürde çoğunlukla sinyal tamamlama ve süper çözünürlük problemlerinde kullanılan PG yaklaşımı, bu çalışmada stereo derinlik haritalarının LiDAR ölçümleri ile iyileştirilmesi amacıyla yeniden yapılandırılmıştır. Önerilen çerçevede LiDAR ölçümleri ankraj noktaları olarak kullanılmakta stereo derinlik haritası ise bu noktalara göre yinelemeli biçimde güncellenmektedir. Böylece, LiDAR’ın yüksek doğruluklu derinlik bilgisi ile stereo derinlik haritasının yüksek yoğunluklu yapısı iteratif iyileştirme sürecinde birleştirilmektedir.

Yöntemin başarımı, hem gerçek sensör verisi hem de farklı saha koşullarını temsil eden veri setleri üzerinde kapsamlı biçimde değerlendirilmiştir. Bu amaçla ZED stereo kamera, Velodyne VLP16 LiDAR ve NVIDIA Jetson Orin bileşenlerinden oluşan taşınabilir bir sensör platformu kurulmuştur. Titreşim ve hareket etkilerini gözlemlemek için sensör platformu (SensorSuiteV2) bir otomobil üzerine entegre edilerek saha verisi toplanmıştır. Hem toplanılan hem de özdeş veri tiplerine sahip açık kaynaklı CitrusFarm veri seti  üzerinde PG ön işlemeli ve ön işlemesiz akışlar karşılaştırılmıştır. Derinlik haritası kalitesi ile navigasyon/haritalama performansına yansıyan etkiler çevrim dışı metriklerle analiz edilmiştir. Elde edilen bulgular, önerilen PG tabanlı Stereo–LiDAR füzyon yönteminin derin öğrenmeye ihtiyaç duymadan düşük işlem maliyetiyle çalışabilen ve gerçek zamanlı sistemlere uyarlanabilir bir derinlik iyileştirme yaklaşımı sunduğunu göstermektedir.

\textbf{Keywords:} Stereo–LiDAR Füzyonu, Papoulis–Gerchberg, İteratif İyileştirme, Gerçek Zamanlı Algılama, SLAM, Konumlandırma, Haritalama


\clearpage

