\setstretch{1.5}
\setcounter{page}{0}
\thispagestyle{plain}
\fancyfoot[C]{\footnotesize \thepage}
\begin{center}{\fontsize{14}{16.8}\bf{YAYINLAMA FİKRİ MÜLKİYET HAKLARI BEYANI} \par}\end{center}
{\normalsize}
\vfil\vfil\null



Enstitü tarafından onaylanan lisansüstü tezimin tamamını veya herhangi bir kısmını, basılı (kağıt) ve elektronik formatta arşivleme ve aşağıda verilen koşullarla kullanıma açma iznini Hacettepe üniversitesine verdiğimi bildiririm. Bu izinle Üniversiteye verilen kullanım hakları dışındaki tüm fikri mülkiyet haklarım bende kalacak, tezimin tamamının ya da bir bölümünün gelecekteki çalışmalarda (makale, kitap, lisans ve patent vb.) kullanım hakları bana ait olacaktır. 

Tezin kendi orijinal çalışmam olduğunu, başkalarının haklarını ihlal etmediğimi ve tezimin tek yetkili sahibi olduğumu beyan ve taahhüt ederim. Tezimde yer alan telif hakkı bulunan ve sahiplerinden yazılı izin alınarak kullanması zorunlu metinlerin yazılı izin alarak kullandığımı ve istenildiğinde suretlerini Üniversiteye teslim etmeyi taahhüt ederim. 

Yükseköğretim Kurulu tarafından yayınlanan\textbf{ “Lisansüstü Tezlerin Elektronik Ortamda Toplanması, Düzenlenmesi ve Erişime Açılmasına İlişkin Yönerge” }kapsamında tezim aşağıda belirtilen koşullar haricince YÖK Ulusal Tez Merkezi / H. Ü. Kütüphaneleri Açık Erişim Sisteminde erişime açılır.


 \begin{todolist}
    \item Enstitü yönetim kurulu kararı ile tezimin erişime açılması mezuniyet tarihimden itibaren 2 yıl ertelenmiştir.
    \item Enstitü yönetim kurulu gerekçeli kararı ile tezimin erişime açılması mezuniyet tarihimden itibaren .... ay ertelenmiştir.
    \item Tezim ile ilgili gizlilik kararı verilmiştir.
\end{todolist}

\begin{minipage}{\textwidth}
    \begin{flushright}
        \sdate \ \ \ \ 
    \end{flushright}
    ~\\
    \begin{flushright}
        \author
    \end{flushright}
\end{minipage}

\clearpage